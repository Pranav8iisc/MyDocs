%% This defines the bibliography file (main.bib) and the bibliography style.
%% If you want to create a bibliography file by hand, change the contents of
%% this file to a `thebibliography' environment.  For more information 
%% see section 4.3 of the LaTeX manual.
\begin{singlespace}
\begin{thebibliography}{10000000000}
\bibitem{1} %1
Microsoft Photosynth.\newline 
photosynth.net\newline
Accessed 3 December 2012

\bibitem{2} %2
Projective geometry tutorial.\newline 
http://www.cse.iitd.ernet.in/~suban/vision/tutorial/node1.html \newline
Accessed 4 December 2012

\bibitem{3} %3
Faugeras,Olivier.(1995).Stratification of 3-D vision: projective, affine, and metric representations,Journal of the Optical Society of America A,Vol-12,46548--4 


\bibitem{4} %4
Hartley,Richard I.,Sturm,Peter(1997).Triangulation, 
Computer Vision and Image Understanding,Volume 68, Issue 2, November 1997,Pages 146-157 

\bibitem{5}  %5
Saxena,Ashutosh,Sun,Min,Ng,Andrew Y.(2007).3-D Reconstruction from Sparse Views using Monocular Vision,ICCV workshop on Virtual Representations and Modeling of Large-scale environments (VRML),2007  

\bibitem{6} %#
Flexiscan3D scanner software \newline
http://www.3d3solutions.com/products/flexscan3d/ \newline
Accessed:3 December 2012

\bibitem{7} %#
Pan,B., Kemao,Q.,Huang,L.,and Asundi,A.(2009).Phase error analysis and compensation for nonsinusoidal waveforms in phase-shifting digital fringe projection profilometry, Opt. Lett.  34, 416-418. 

\bibitem{8} %#
Tsai,Roger Y. (1987) :A Versatile Camera Calibration Technique for High- 
Accuracy 3D Machine Vision Metrology Using Off-the-Shelf TV Cameras 
and Lenses,IEEE Journal of Robotics and Automation, Vol. RA-3, No. 4, 
August 1987, pp. 323-344. 

\bibitem{9} %#
Zhang,Song(2005).High-resolution, Real-time 3-D Shape Measurement,Ph.D. dissertation, Stony Brook University, Stony Brook, NY, 2005. 

\bibitem{10} %#
Zhang,Z.(2000).A flexible new technique for camera calibration. IEEE Transactions on Pattern Analysis and Machine Intelligence, 22(11):1330-1334, 2000 

\bibitem{11} %#
Vo,Minh,Wang,Zhaoyang,Pan,Bing,Pan,Tongyan(2012).Hyper-accurate flexible calibration technique for fringe-projection-based three-dimensional imaging,Opt. Express 20, 16926-16941 (2012) 

\bibitem{12} %#
Hartley,R.I.(1994),An algorithm for self calibration from several viewes.In Proceedings of IEEE Conference on Computer Vision and Pattern Recognition,pages 908-912,IEEE Computer Society Press 

\bibitem{13} %#
Luong,Q.~T,Faugeras,O.(1997).Self-calibration of a moving camera from point correspondences and fundamental matrices.The International Journal of Computer Vision,22(3):261-289,1997 

\bibitem{14} %#
Caprile,B,Torre,V.(1990).Using Vanishing Points for Camera Calibration.The International Journal Of Computer Vision,4(2):127-140,Mar.1990 

\bibitem{15} %#
Hartley,R.(1994).Self-calibration from multiple views with a rotating camera.In Proceedings of 3rd ECCV,volume 800-801 of 'Lecture notes in Computer Science' pages 471-478,Stockholm,Sweden,May 1994.Springer-Verlag 

\bibitem{16} %#
Zollner, Helmut and Sablatnig,Robert(2004).Comparison of Methods for Geometric Camera Calibration using Planar Calibration Targets,Proceedings of the 28th Workshop of the Austrian Association of Pattern Recognition, AAPR 2004,pp.237--244,OCG Schriftenreihe, {\"O}sterreichische Arbeitsgemeinschaft f{\"u}r Mustererkennung

\bibitem{17} %#
Heikkil\"{a},J.,Silven,O.(1997).A Four-Step Camera Calibration Procedure with Implicit Image Correction.
In Proc. of IEEE Computer Vision and Pattern Recognition, pp. 1106-1112, 1997

\bibitem{18} %#
Camera Calibration Toolbox For Matlab \newline
http://www.vision.caltech.edu/bouguetj/calib\_doc/index.html \newline
Accessed:3 December 2012 

\bibitem{19} %#
OpenCV documentation on Camera calibration \newline
http://docs.opencv.org/modules/calib3d/doc/calib3d.html \newline
Accessed 5 December 2012

\bibitem{20} %#
Levenberg Marquardt algorithm.\newline
http://en.wikipedia.org/wiki/Levenberg\%E2\%80\%93Marquardt\_algorithm \newline
Accessed 4 December 2012.

\bibitem{21} %# 
Extrinsic parameters estimation in OpenCV\newline 
http://opencv.willowgarage.com/documentation/camera\_calibration\_and\_3d\newline
\_reconstruction.html\#findextrinsiccameraparams2 \newline
Accessed 5 December 2012

\bibitem{22} %#
Bradski,Gary and Kaehler,Adrian(2008).Learning OpenCV,Computer Vision with the OpenCV Library.O'Reilly Media,September 2008 

\bibitem{23} %#
Asla Medeiros e Sa,Esdras Soares de Medeiros Filho,Paulo Cezar Carvalho,Luiz Velho.Coded Structured Light for 3D-photography:an Overview.\newline
www.visgraf.impa.br/Data/RefBib/PS\_PDF/rita-survey/survey.pdf \newline
Accessed 5 December 2012
  

\bibitem{24} %#
Salvi,J.,Pag\`es,J,Tutorial on Coded Light Projection techniques\newline 
http://eia.udg.es/~qsalvi/Tutorial\_Coded\_Light\_Projection\_Techniques\_arc\newline
hivos/v3\_document.html \newline
Accessed 4 December 2012

\bibitem{25} %#
Geng,J.(2011)."Structured-light 3D surface imaging: a tutorial," Adv. Opt. Photon.3, 128-160 .

\bibitem{26} %#
Pag\`es,Jordi,Salvi,Joaquim,Garcia,Rafael,Matabosch,Carles(2003).Overview of coded light projection techniques for automatic 3D profiling 

\bibitem{27} %#
Posdamer,J. L.,Altschuler,M. D.(1982). Surface measurement by 
space-encoded projected beam systems, Comput. Graph. Image Processing 
18(1), 1-17 (1982). 

\bibitem{28} %#
Inokuchi,S.,Sato,K.,Matsuda,F.(1984).Range-imaging for 3-D object recognition, 
in International Conference on Pattern Recognition (International 
Association for Pattern Recognition, 1984), pp. 806-808. 

\bibitem{29} %#
Caspi,D.,Kiryati,N.,Shamir,J.(1998),Range imaging with adaptive color 
structured light, IEEE Trans. Pattern Anal. Mach. Intell. 20(5), 470-480 
(May 1998). 

\bibitem{30} %#
Horn,Eli,Kiryati,Nahum(1997).Toward Optimal Structured Light Patterns,Image and Vision Computing,Vol. 17,pp.87-97 

\bibitem{31} %#
Ghiglia,Dennis C.,Pritt,Mark D..Two-Dimensional Phase Unwrapping: Theory, Algorithms, and Software.Wiley Publications,ISBN: 978-0-471-24935-1

\bibitem{32} %#
MacWilliams,F.J.,Sloane,N.J.A.(1976).Pseudorandom sequences and 
arrays, Proc. IEEE 64(12), 1715-1729. 

\bibitem{33} %#
De Bruijn sequnces\newline 
http://feed-back.be/nick/?page\_id=310\newline
Accessed 12 December 2012

\bibitem{34} %#
Moigne,J.Le,Waxman,A. M.(1985), Multi-resolution grid patterns for 
building range maps,in Vision-85, Applied Machine Vision Conference 
(ASME) (Society of Manufacturing Engineers, 1985), pp. 22-39. 

\bibitem{35} %#
Ulusoy,A. Osman,Calakli,F.,Taubin,G.(2009), One-shot scanning using De 
Bruijn spaced grids, in 2009 IEEE 12th International Conference on Computer 
Vision Workshops (ICCV Workshops) (IEEE, 2009), pp. 1786-1792. 
 
\bibitem{36} %#
Carrihill,B.,Hummel,R.(1985). Experiments with the intensity ratio depth sensor. In Computer Vision, Graphics and Image 
Processing, volume 32, pages 337--358. Academic Press, 1985 

\bibitem{37} %#
Chazan,G.,Kiryati,N.(1995). Pyramidal intensity ­ratio depth sensor. Technical report 121, Center for Communication and Informa­tion Technologies, Department of Electrical Engineering, Technion, Haifa, Israel, October 1995. 

\bibitem{38} %#
Hung,D.C.D.(1993). 3d scene modelling by sinusoid encoded illumination. Image and Vision Computing, 11:251--256, 1993. 

\bibitem{39} %#
Tajima,J.,Iwakawa,M.(1990). 3D data acquisition by rainbow range finder. In International Conference on Pattern Recognition, pages 309--313, 1990 

\bibitem{40} %#
Sato,T.(1999). Multispectral pattern projection range finder. In Proceedings of the Conference on ThreeDimensional Image Capture and Applications II, volume 3640, pages 28--37, San Jose, California, January 1999. SPIE. 

\bibitem{41} %#
Stripe boundary codes for real-time structured-light range scanning of moving 
objects. In The 8th IEEE International Conference on 
Computer Vision, pages II: 359-366, 2001. 

\bibitem{42} %#
Zhang,Song(2010).Recent progresses on real-time 3D shape measurement using digital fringe projection techniques.Optics and Lasers in Engineering, Volume 48, Issue 2, February 2010, Pages 149-158 

\bibitem{43} %#
Gorthi,S.S.,Rastogi,P(2009).Fringe projection technique:Whither we are?.Optics and Lasers in Engineering, 2009 

\bibitem{44} %#
`open-light'-Google code \newline
http://code.google.com/p/open-light/\newline
Accessed 5 December 2012 

\bibitem{45} %#
`Structured-light'-Goole code \newline
http://code.google.com/p/structured-light/ \newline
Accessed 5 December 2012

\bibitem{46} %#
Gupta,Mohit and Agrawal, Amit and Veeraraghavan, Ashok and Narasimhan, Srinivasa,G.(2012).A Practical Approach to 3D Scanning in the Presence of Interreflections,Subsurface Scattering and Defocus.International Journal of Computer Vision,pp.1-23,Springer

\bibitem{47} %#
Ghiglia,Dennis C.,Romero,Louis A.(1996)."Minimum Lp-norm two-dimensional phase unwrapping," J. Opt. Soc. Am. A 13, 1999-2013 (1996)

\bibitem{48} %#
Mundy,J.,Zisserman,A.(1992).Appendix:Projective geometry for machine vision-Geometric invariances in computer vision,MIT Press,1992 
 

\bibitem{49} %#
R.I.,Hartley,Zisserman,A.(2004).Multiple View Geometry in Computer Vision,Second edition,Cambridge University Press, ISBN:0521540518
  
\bibitem{50} %#
Hartley,R.,Gupta,R.,Chang,T.(1992).Stereo from uncalibrated cameras, 
in Proceedings, IEEE Conference on Computer Vision and Pattern Recognition, 1992, pp. 761-764. 

\bibitem{51} %#
Triangulation\newline 
http://en.wikipedia.org/wiki/Triangulation\newline
Accessed 5 December 2012

\bibitem{52} %#   
Brett R.,Jones(2010).AUGMENTING COMPLEX SURFACES WITH PROJECTOR-CAMERA SYSTEMS,M.S.Thesis,University of Illinois at Urbana-Champaign

\bibitem{53} %#
Point cloud library.\newline
pointclouds.org/.\newline
Accessed 24 November 2012

\bibitem{54} %#
Accuracy \& precision.\newline
http://en.wikipedia.org/wiki/Accuracy\_and\_precision\newline
Accessed 24 November 2012

\bibitem{55} %#
OpenCV function for checkerboard corner detection.\newline
http://opencv.willowgarage.com/documentation/camera\_calibration\_and\_3d\_\newline
reconstruction.html\#findchessboardcorners\newline
Accessed 24 November 2012

\bibitem{56} %#
Sansoni,G., Carocci,M., and Rodella,R.(1999). "Three-Dimensional Vision Based on a Combination of Gray-Code and Phase-Shift Light Projection: Analysis and Compensation of the Systematic Errors," Appl. Opt.  38, 6565-6573 .

\bibitem{57} %#
Microsoft Kinect.\newline
http://www.xbox.com/en-US/KINECT.\newline
Accessed 24 November 2012

\bibitem{58} %#
OpenNi home page.\newline
openni.org/.Accessed 24 November 2012

\bibitem{59} %#
OpenKinect home page.\newline
openkinect.org/.Accessed 24 November 2012

\bibitem{60} %#
RGBDemo program for accessing kinect data.\newline
labs.manctl.com/rgbdemo/ \newline
Accessed 24 November 2012

\bibitem{61} %#
Skanect for Windows \& Mac-plateform.\newline
http://manctl.com/products.html.\newline
Accessed 24 November 2012

\bibitem{62} %#
Code laboratory.\newline
http://codelaboratories.com/nui/.\newline 
Accessed 24 November 2012

\bibitem{63} %#
Khoshelham, K., Elberink, S.O.(2012). Accuracy and Resolution of Kinect Depth Data for Indoor Mapping Applications. Sensors, 12, 1437-1454.

\bibitem{64} %#
Freedman, B., Shpunt, A., Machline, M., \& Arieli, Y. (2010).
Patent No. US2010/0018123 A1. United States of America.

\bibitem{65} %#
Smisek, J., Jancosek, M., Pajdla, T., (2011), "3D with Kinect," Computer Vision Workshops (ICCV Workshops),pp.1154-1160, 6-13 Nov. 2011
doi: 10.1109/ICCVW.2011.6130380

\bibitem{66} %#
ifixit Kinect teardown.\newline
http://www.ifixit.com/Teardown/Microsoft+Kinect+Teardown/4066/1.\newline
Accessed 24 November 2012

\bibitem{67} %#
ROS kinect calibration page. \newline
http://www.ros.org/wiki/openni\_launch/Tutorials/IntrinsicCalibration?actio\newline
n=show\&redirect=openni\_camera\%2Fcalibration.\newline
Accessed 24 November 2012

\bibitem{68} %#
Moln\`ar,B., Toth,C.K., and Detrek\"{o}i,A.(2012). ACCURACY TEST OF MICROSOFT KINECT FOR HUMAN MORPHOLOGIC MEASUREMENTS, Int. Arch. Photogramm. Remote Sens. Spatial Inf. Sci., XXXIX-B3, 543-547, doi:10.5194/isprsarchives-XXXIX-B3-543-2012

\bibitem{69} %#
Herrera C.,D.,Kannala, J.,Heikkil\"{a},J.(2012).Joint depth and color camera calibration with distortion correction, TPAMI,.

\bibitem{70} %#
Boehm,J.(2012).NATURAL USER INTERFACE SENSORS FOR HUMAN BODY MEASUREMENT, Int. Arch. Photogramm. Remote Sens. Spatial Inf. Sci., XXXIX-B3, 531-536, doi:10.5194/isprsarchives-XXXIX-B3-531-2012

\bibitem{71} %#
Chow,J.C.K.,Ang,K.D.,Lichti,D.D.,and Teskey,W.F.(2012). 
PERFORMANCE ANALYSIS OF A LOW-COST TRIANGULATION-BASED 3D
CAMERA: MICROSOFT KINECT SYSTEM
International Archives of the Photogrammetry, Remote Sensing and Spatial Information Sciences, Volume XXXIX-B5
XXII ISPRS Congress

\bibitem{72} %#
Salvi,Joaquim,Armangu\`e,Xavier,Batlle,Joan(2002).A comparative review of camera calibrating methods with accuracy evaluation, Pattern Recognition, Volume 35, Issue 7, July 2002, Pages 1617-1635, ISSN 0031-3203, 10.1016/S0031-3203(01)00126-1.

\bibitem{73} %#
Drar\`eni, J. and  Roy, S. and Sturm, P.(2012).Methods for Geometrical Video Projector Calibration,Machine Vision and Applications

\end{thebibliography}
\end{singlespace}

