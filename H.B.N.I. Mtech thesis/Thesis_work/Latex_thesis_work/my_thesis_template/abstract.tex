% $Log: abstract.tex,v $
% Revision 1.1  93/05/14  14:56:25  starflt
% Initial revision
% 
% Revision 1.1  90/05/04  10:41:01  lwvanels
% Initial revision
% 
%
%% The text of your abstract and nothing else (other than comments) goes here.
%% It will be single-spaced and the rest of the text that is supposed to go on
%% the abstract page will be generated by the abstractpage environment.  This
%% file should be \input (not \include 'd) from cover.tex.
\chapter*{Abstract}
In this work, a 3D scanner based on a non-contact active technique is developed which can be used as a metrology equipment.  Further, the factors which affect measurement accuracy of the system is analyzed by studying the sources of systematic errors and an attempt has been made to quantize them.\newline

This work was initiated to address the growing need to acquire expertise in surface metrology required in medical applications and to explore alternatives to conventional contact based metrology devices such as CMM.\newline

Specifically, \textit{Coded phase shift technique} is used in this work, which being a combination of both phase-shift and binary coded technique provides high resolution 3D reconstruction as of phase shift technique and high robustness against noise as of binary coded technique.\newline

Accuracy and repeatability of calibration algorithms used in this work are quantified. For assessing accuracy of stereo-correspondence an approach is proposed to quantify correspondence error and has been demonstrated with real experiments.  Further, measurement accuracy and precision of 3D data acquired from the developed system has been studied and compared with that of popularly used sensor Microsoft Kinect. 3D data acquisition rate of kinect is considerably higher than the current version of developed system. Future versions of developed system are expected to reduce this gap.\newline

Open-source development of this system can provide a low cost, portable and flexible alternative to the commercially available 3D scanners. Further it provides a flexible, research-oriented system whose basic parameters can be experimented with. This will provide us further insights into science of 3D metrology. 

