%%%%%%%%%%%%%%%%%%%%%%%%%%%%%%%%%%%%%%%%%%%%%%%%%%%%%%%%%%%%%%%%%%%%%%%%%%%%%
%
% Copyright 2007/2008 by Christian Feuersaenger.
%
% This program is free software: you can redistribute it and/or modify
% it under the terms of the GNU General Public License as published by
% the Free Software Foundation, either version 3 of the License, or
% (at your option) any later version.
% 
% This program is distributed in the hope that it will be useful,
% but WITHOUT ANY WARRANTY; without even the implied warranty of
% MERCHANTABILITY or FITNESS FOR A PARTICULAR PURPOSE.  See the
% GNU General Public License for more details.
% 
% You should have received a copy of the GNU General Public License
% along with this program.  If not, see <http://www.gnu.org/licenses/>.
%
%
%%%%%%%%%%%%%%%%%%%%%%%%%%%%%%%%%%%%%%%%%%%%%%%%%%%%%%%%%%%%%%%%%%%%%%%%%%%%%

%%%%%%%%%%%%%%%%%%%%%%%%%%%%%%%%%%%%%%%%%%%%%%%%%%%%%%%%%%%%%%%%%%%%%%%%%%%%%
%
% Package pgfplots.sty documentation. 
%
% Copyright 2007/2008 by Christian Feuersaenger.
%
% This program is free software: you can redistribute it and/or modify
% it under the terms of the GNU General Public License as published by
% the Free Software Foundation, either version 3 of the License, or
% (at your option) any later version.
% 
% This program is distributed in the hope that it will be useful,
% but WITHOUT ANY WARRANTY; without even the implied warranty of
% MERCHANTABILITY or FITNESS FOR A PARTICULAR PURPOSE.  See the
% GNU General Public License for more details.
% 
% You should have received a copy of the GNU General Public License
% along with this program.  If not, see <http://www.gnu.org/licenses/>.
%
%
%%%%%%%%%%%%%%%%%%%%%%%%%%%%%%%%%%%%%%%%%%%%%%%%%%%%%%%%%%%%%%%%%%%%%%%%%%%%%
\pdfminorversion=5 % to allow compression
\pdfobjcompresslevel=2
\documentclass[a4paper]{ltxdoc}

\usepackage{makeidx}

% DON't let hyperref overload the format of index and glossary. 
% I want to do that on my own in the stylefiles for makeindex...
\makeatletter
\let\@old@wrindex=\@wrindex
\makeatother

\usepackage{ifpdf}
\usepackage[pdfborder=0 0 0]{hyperref}
	\hypersetup{%
		colorlinks=true,	% use true to enable colors below:
		linkcolor=blue,%red,
		filecolor=blue,%magenta,
		pagecolor=blue,%red,
		urlcolor=blue,%cyan,
		citecolor=blue,
		%frenchlinks=false,	% small caps instead of colors
		pdfborder=0 0 0,	% PDF link-darstellung, falls colorlinks=false. 0 0 0: nix. 0 0 1: default.
		%plainpages=false,	% Das ist notwendig, wenn die Seitenzahlen z.T. in Arabischen und z.T. in römischen Ziffern gemacht werden.
		pdftitle=Package PGFPLOTS manual,
		pdfauthor=Dr. Christian Feuersänger,
		%pdfsubject=,
		pdfkeywords={pgfplots pgf tikz tex latex},
	}

\makeatletter
\let\@wrindex=\@old@wrindex
\makeatother


\makeatletter
% disables colorlinks for all following \ref commands
\def\pgfplotsmanualdisablecolorforref{%
	\pgfutil@ifundefined{pgfplotsmanual@oldref}{%
		\let\pgfplotsmanual@oldref=\ref
	}{}%
	\def\ref##1{%
		\begingroup
		\let\Hy@colorlink=\pgfplots@disabled@Hy@colorlink
		\let\Hy@endcolorlink=\pgfplots@disabled@Hy@endcolorlink
		\pgfplotsmanual@oldref{##1}%
		\endgroup
	}%
}%
\def\pgfplots@disabled@Hy@colorlink#1{\begingroup}%
\def\pgfplots@disabled@Hy@endcolorlink{\endgroup}%
\makeatother

% Formatiere Seitennummern im Index:
\newcommand{\indexpageno}[1]{%
	{\bfseries\hyperpage{#1}}%
}


\newcommand{\C}{\mathbb{C}}
\newcommand{\R}{\mathbb{R}}
\newcommand{\N}{\mathbb{N}}
\newcommand{\Z}{\mathbb{Z}}

\long\def\COMMENTLOWLEVEL#1\ENDCOMMENT{}
\def\ENDCOMMENT{}

\usepackage{textcomp}
\usepackage{booktabs}

\usepackage{calc}
\usepackage[formats]{listings}
%\usepackage{courier} % don't use it - the '^' character can't be copy-pasted in courier

\usepackage{array}
\lstset{%
	basicstyle=\ttfamily,
	language=[LaTeX]tex, % Seems as if \lstset{language=tex} must be invoked BEFORE loading tikz!?
	tabsize=4,
	breaklines=true,
	breakindent=0pt
}

\ifpdf
	\pdfinfo {
		/Author	(Christian Feuersaenger)
	}

\else
%	\def\pgfsysdriver{pgfsys-dvipdfm.def}
\fi
%\def\pgfsysdriver{pgfsys-pdftex.def}
\usepackage{pgfplots}
\usepackage{pgfplotstable}

\ifpdf
	% this allows to disable the clickable lib from command line using
	% \pdflatex '\def\pgfplotsclickabledisabled{1}%%%%%%%%%%%%%%%%%%%%%%%%%%%%%%%%%%%%%%%%%%%%%%%%%%%%%%%%%%%%%%%%%%%%%%%%%%%%%
%
% Package pgfplots.sty documentation. 
%
% Copyright 2007/2008 by Christian Feuersaenger.
%
% This program is free software: you can redistribute it and/or modify
% it under the terms of the GNU General Public License as published by
% the Free Software Foundation, either version 3 of the License, or
% (at your option) any later version.
% 
% This program is distributed in the hope that it will be useful,
% but WITHOUT ANY WARRANTY; without even the implied warranty of
% MERCHANTABILITY or FITNESS FOR A PARTICULAR PURPOSE.  See the
% GNU General Public License for more details.
% 
% You should have received a copy of the GNU General Public License
% along with this program.  If not, see <http://www.gnu.org/licenses/>.
%
%
%%%%%%%%%%%%%%%%%%%%%%%%%%%%%%%%%%%%%%%%%%%%%%%%%%%%%%%%%%%%%%%%%%%%%%%%%%%%%
% SEE pgfplots-macros.tex as well!
%%%%%%%%%%%%%%%%%%%%%%%%%%%%%%%%%%%%%%%%%%%%%%%%%%%%%%%%%%%%%%%%%%%%%%%%%%%%%
%
% Package pgfplots.sty documentation. 
%
% Copyright 2007/2008 by Christian Feuersaenger.
%
% This program is free software: you can redistribute it and/or modify
% it under the terms of the GNU General Public License as published by
% the Free Software Foundation, either version 3 of the License, or
% (at your option) any later version.
% 
% This program is distributed in the hope that it will be useful,
% but WITHOUT ANY WARRANTY; without even the implied warranty of
% MERCHANTABILITY or FITNESS FOR A PARTICULAR PURPOSE.  See the
% GNU General Public License for more details.
% 
% You should have received a copy of the GNU General Public License
% along with this program.  If not, see <http://www.gnu.org/licenses/>.
%
%
%%%%%%%%%%%%%%%%%%%%%%%%%%%%%%%%%%%%%%%%%%%%%%%%%%%%%%%%%%%%%%%%%%%%%%%%%%%%%
\pdfminorversion=5 % to allow compression
\pdfobjcompresslevel=2
\documentclass[a4paper]{ltxdoc}

\usepackage{makeidx}

% DON't let hyperref overload the format of index and glossary. 
% I want to do that on my own in the stylefiles for makeindex...
\makeatletter
\let\@old@wrindex=\@wrindex
\makeatother

\usepackage{ifpdf}
\usepackage[pdfborder=0 0 0]{hyperref}
	\hypersetup{%
		colorlinks=true,	% use true to enable colors below:
		linkcolor=blue,%red,
		filecolor=blue,%magenta,
		pagecolor=blue,%red,
		urlcolor=blue,%cyan,
		citecolor=blue,
		%frenchlinks=false,	% small caps instead of colors
		pdfborder=0 0 0,	% PDF link-darstellung, falls colorlinks=false. 0 0 0: nix. 0 0 1: default.
		%plainpages=false,	% Das ist notwendig, wenn die Seitenzahlen z.T. in Arabischen und z.T. in römischen Ziffern gemacht werden.
		pdftitle=Package PGFPLOTS manual,
		pdfauthor=Dr. Christian Feuersänger,
		%pdfsubject=,
		pdfkeywords={pgfplots pgf tikz tex latex},
	}

\makeatletter
\let\@wrindex=\@old@wrindex
\makeatother


\makeatletter
% disables colorlinks for all following \ref commands
\def\pgfplotsmanualdisablecolorforref{%
	\pgfutil@ifundefined{pgfplotsmanual@oldref}{%
		\let\pgfplotsmanual@oldref=\ref
	}{}%
	\def\ref##1{%
		\begingroup
		\let\Hy@colorlink=\pgfplots@disabled@Hy@colorlink
		\let\Hy@endcolorlink=\pgfplots@disabled@Hy@endcolorlink
		\pgfplotsmanual@oldref{##1}%
		\endgroup
	}%
}%
\def\pgfplots@disabled@Hy@colorlink#1{\begingroup}%
\def\pgfplots@disabled@Hy@endcolorlink{\endgroup}%
\makeatother

% Formatiere Seitennummern im Index:
\newcommand{\indexpageno}[1]{%
	{\bfseries\hyperpage{#1}}%
}


\newcommand{\C}{\mathbb{C}}
\newcommand{\R}{\mathbb{R}}
\newcommand{\N}{\mathbb{N}}
\newcommand{\Z}{\mathbb{Z}}

\long\def\COMMENTLOWLEVEL#1\ENDCOMMENT{}
\def\ENDCOMMENT{}

\usepackage{textcomp}
\usepackage{booktabs}

\usepackage{calc}
\usepackage[formats]{listings}
%\usepackage{courier} % don't use it - the '^' character can't be copy-pasted in courier

\usepackage{array}
\lstset{%
	basicstyle=\ttfamily,
	language=[LaTeX]tex, % Seems as if \lstset{language=tex} must be invoked BEFORE loading tikz!?
	tabsize=4,
	breaklines=true,
	breakindent=0pt
}

\ifpdf
	\pdfinfo {
		/Author	(Christian Feuersaenger)
	}

\else
%	\def\pgfsysdriver{pgfsys-dvipdfm.def}
\fi
%\def\pgfsysdriver{pgfsys-pdftex.def}
\usepackage{pgfplots}
\usepackage{pgfplotstable}

\ifpdf
	% this allows to disable the clickable lib from command line using
	% \pdflatex '\def\pgfplotsclickabledisabled{1}\input{pgfplots.tex}'
	\expandafter\ifx\csname pgfplotsclickabledisabled\endcsname\relax
		\usepgfplotslibrary{clickable}
	\fi
\fi

%\usepackage{fp}
% ATTENTION:
% this requires pgf version NEWER than 2.00 :
%\usetikzlibrary{fixedpointarithmetic}

\usepgfplotslibrary{dateplot,units,groupplots}

\usepackage[a4paper,left=2.25cm,right=2.25cm,top=2.5cm,bottom=2.5cm,nohead]{geometry}
\usepackage{amsmath,amssymb}
\usepackage{xxcolor}
\usepackage{pifont}
\usepackage[latin1]{inputenc}
\usepackage{amsmath}
\usepackage{eurosym}
\usepackage{nicefrac}
\input{pgfplots-macros}

\makeatletter
\@ifpackageloaded{tex4ht}{
}{%
	\IfFileExists{ocg.sty}{%
		\usepackage{ocg}%
	}{%
		\usepackage{pgfplots_ocg_copy}%
	}
}%
\makeatother

\usepackage{nicefrac}

\graphicspath{{figures/}}

\def\preambleconfig{width=7cm,compat=1.7}


\expandafter\pgfplotsset\expandafter{\preambleconfig}


\makeatletter
% And now, invoke
% 	/codeexample/typeset listing/.add={% Preamble:\pgfplotsset{\preambleconfig}}{}}
% since listings are VERBATIM, I need to do some low-level things
% here to get the correct \catcodes:
\pgfkeys{/codeexample/typeset listing/.add code={%
		\ifcode@execute
			\pgfutil@in@{axis}{#1}%
			\ifpgfutil@in@
				{\tiny
					\% Preamble: \pgfmanualpdfref{\textbackslash pgfplotsset}{\pgfmanual@pretty@backslash pgfplotsset}%
						\pgfmanual@pretty@lbrace \expandafter\pgfmanualprettyprintpgfkeys\expandafter{\preambleconfig}\pgfmanual@pretty@rbrace
				}%
			\fi
		\fi
	}{},%
	%/codeexample/typeset listing/.show code,
}%
\makeatother

\pgfplotsset{
	%every axis/.append style={width=7cm},
	filter discard warning=false,
}

\pgfqkeys{/codeexample}{%
	every codeexample/.append style={
		width=8cm,
		/pgfplots/legend style={fill=graphicbackground},
		/pgfplots/contour/every contour label/.append style={
			every node/.append style={fill=graphicbackground}
		},
	},
	tabsize=4,
}

\usetikzlibrary{backgrounds,patterns}
% Global styles:
\tikzset{
  shape example/.style={
    color=black!30,
    draw,
    fill=yellow!30,
    line width=.5cm,
    inner xsep=2.5cm,
    inner ysep=0.5cm}
}

\newcommand{\FIXME}[1]{\textcolor{red}{(FIXME: #1)}}

% fuer endvironment 'sidewaysfigure' bspw
% \usepackage{rotating}

\newcommand\Tikz{Ti\textit kZ}
\newcommand\PGF{\textsc{pgf}}
\newcommand\PGFPlots{\pgfplotsmakefilelinkifuseful{pgfplots}{\textsc{pgfplots}}}
\newcommand\PGFPlotstable{\pgfplotsmakefilelinkifuseful{pgfplotstable}{\textsc{PgfplotsTable}}}

\makeindex

% Fix overful hboxes automatically:
\tolerance=2000
\emergencystretch=10pt

\tikzset{prefix=gnuplot/pgfplots_} % prefix for 'plot function'

\author{%
	Dr.\ Christian Feuers\"anger\\
	{\footnotesize\texttt{cfeuersaenger@users.sourceforge.net}}}%



%\RequirePackage[german,english,francais]{babel}

\def\matlabcolormaptext{This colormap is similar to one shipped with Matlab (\textregistered) under a similar name.}

\IfFileExists{tikzlibraryspy.code.tex}{%
\usetikzlibrary{spy}
}{%
	\message{ERROR: tikz SPY library NOT available. The manual will only compile partially.^^J}%
}%

\usetikzlibrary{decorations.markings}

\usepgfplotslibrary{%
	ternary,
	smithchart,
	patchplots,
	polar,
	colormaps,
}
\pgfqkeys{/codeexample}{%
	every codeexample/.append style={
		/pgfplots/every ternary axis/.append style={
			/pgfplots/legend style={fill=graphicbackground},
		}
	},
	tabsize=4,
}

\pgfplotsmanualenableexternalizationofexpensive

%\usetikzlibrary{external}
%\tikzexternalize[prefix=figures/]{pgfplots}

\title{%
	Manual for Package \PGFPlots\\
	{\small 2D/3D Plots in \LaTeX, Version \pgfplotsversion}\\
	{\small\href{http://sourceforge.net/projects/pgfplots}{http://sourceforge.net/projects/pgfplots}}
	%\\{\small Attention: you are using an unstable development version.}
}

%\includeonly{pgfplots.intro}


\begin{document}

\def\plotcoords{%
\addplot coordinates {
(5,8.312e-02)    (17,2.547e-02)   (49,7.407e-03)
(129,2.102e-03)  (321,5.874e-04)  (769,1.623e-04)
(1793,4.442e-05) (4097,1.207e-05) (9217,3.261e-06)
};

\addplot coordinates{
(7,8.472e-02)    (31,3.044e-02)    (111,1.022e-02)
(351,3.303e-03)  (1023,1.039e-03)  (2815,3.196e-04)
(7423,9.658e-05) (18943,2.873e-05) (47103,8.437e-06)
};

\addplot coordinates{
(9,7.881e-02)     (49,3.243e-02)    (209,1.232e-02)
(769,4.454e-03)   (2561,1.551e-03)  (7937,5.236e-04)
(23297,1.723e-04) (65537,5.545e-05) (178177,1.751e-05)
};

\addplot coordinates{
(11,6.887e-02)    (71,3.177e-02)     (351,1.341e-02)
(1471,5.334e-03)  (5503,2.027e-03)   (18943,7.415e-04)
(61183,2.628e-04) (187903,9.063e-05) (553983,3.053e-05)
};

\addplot coordinates{
(13,5.755e-02)     (97,2.925e-02)     (545,1.351e-02)
(2561,5.842e-03)   (10625,2.397e-03)  (40193,9.414e-04)
(141569,3.564e-04) (471041,1.308e-04) 
(1496065,4.670e-05)
};
}%


\maketitle
\begin{abstract}%
\PGFPlots\ draws high--quality function plots in normal or logarithmic scaling with a user-friendly interface directly in \TeX. The user supplies axis labels, legend entries and the plot coordinates for one or more plots and \PGFPlots\ applies axis scaling, computes any logarithms and axis ticks and draws the plots. It supports line plots, scatter plots, piecewise constant plots, bar plots, area plots, mesh-- and surface plots, patch plots, contour plots, quiver plots, histogram plots, polar axes, ternary diagrams, smith charts and some more. It is based on Till Tantau's package \PGF/\Tikz.
\end{abstract}
\tableofcontents
\section{Introduction}
This package provides tools to generate plots and labeled axes easily. It draws normal plots, logplots and semi-logplots, in two and three dimensions. Axis ticks, labels, legends (in case of multiple plots) can be added with key-value options. It can cycle through a set of predefined line/marker/color specifications. In summary, its purpose is to simplify the generation of high-quality function and/or data plots, and solving the problems of
\begin{itemize}
	\item consistency of document and font type and font size,
	\item direct use of \TeX\ math mode in axis descriptions,
	\item consistency of data and figures (no third party tool necessary),
	\item inter-document consistency using preamble configurations and styles.
\end{itemize}
Although not necessary, separate |.pdf| or |.eps| graphics can be generated using the |external| library developed as part of \Tikz.

You are invited to use \PGFPlots\ for visualization of medium sized data sets in two and three dimensions.


\section[About PGFPlots: Preliminaries]{About {\normalfont\PGFPlots}: Preliminaries}
This section contains information about upgrades, the team, the installation (in case you need to do it manually) and troubleshooting. You may skip it completely except for the upgrade remarks.

\PGFPlots\ is built completely on \Tikz/\PGF. Knowledge of \Tikz\ will simplify the work with \PGFPlots, although it is not required.

However, note that this library requires at least \PGF\ version $2.10$. At the time of this writing, many \TeX-distributions still contain the older \PGF\ version $1.18$, so it may be necessary to install a recent \PGF\ prior to using \PGFPlots.

\subsection{Components}
\PGFPlots\ comes with two components:
\begin{enumerate}
	\item the plotting component (which you are currently reading) and
	\item the \PGFPlotstable\ component which simplifies number formatting and postprocessing of numerical tables. It comes as a separate package and has its own manual \href{file:pgfplotstable.pdf}{pgfplotstable.pdf}.
\end{enumerate}

\subsection{Upgrade remarks}
This release provides a lot of improvements which can be found in all detail in \texttt{ChangeLog} for interested readers. However, some attention is useful with respect to the following changes.

\subsubsection{New Optional Features}
\PGFPlots\ has been written with backwards compatibility in mind: old \TeX\ files should compile without modifications and without changes in the appearance. However, new features occasionally lead to a different behavior. In such a case, \PGFPlots\ will deactivate the new feature\footnote{In case of broken backwards compatibility, we apologize -- and ask you to submit a bug report. We will take care of it.}.

Any new features or bugfixes which cause backwards compatibility problems need to be activated \emph{manually} and \emph{explicitly}. In order to do so, you should use 
\begin{codeexample}[code only]
\usepackage{pgfplots}
\pgfplotsset{compat=1.6}
\end{codeexample}
\noindent in your preamble. This will configure the compatibility layer.

You should have at least |compat=1.3|. The suggested value is printed to the |.log| file after running \TeX.

Here is a list of changes introduced in recent versions of \PGFPlots:
\begin{enumerate}
	\item \PGFPlots\ 1.6 added new options for more accurate scaling and more scaling options for |\addplot3 graphics|. These are enabled with |compat=1.6| or higher.

	\item \PGFPlots\ 1.5.1 interpretes circle- and ellipse radii as \PGFPlots\ coordinates (older versions used \pgfname\ unit vectors which have no direct relation to \PGFPlots). In other words: starting with version 1.5.1, it is possible to write |\draw circle[radius=5]| inside of an axis. This requires |\pgfplotsset{compat=1.5.1}| or higher. 

	Without this compatibility setting, circles and ellipses use low--level canvas units of \pgfname\ as in earlier versions.

	\item \PGFPlots\ 1.5 uses |log origin=0| as default (which influences logarithmic bar plots or stacked logarithmic plots). Older versions keep |log origin=infty|. This requires |\pgfplotsset{compat=1.5}| or higher.

	\item \PGFPlots\ 1.4 has fixed several smaller bugs which might produce differences of about $1$--$2\text{pt}$ compared to earlier releases. This requires |\pgfplotsset{compat=1.4}| or higher.

	\item \PGFPlots\ 1.3 comes with user interface improvements. The technical distinction between ``behavior options'' and ``style options'' of older versions is no longer necessary (although still fully supported).

	This is always activated.

	\item \PGFPlots\ 1.3 has a new feature which allows to \emph{move axis labels tight to tick labels} automatically. This is strongly recommended. It requires |\pgfplotsset{compat=1.3}| or higher.

	Since this affects the spacing, it is not enabled be default.

	\item \PGFPlots\ 1.3 supports reversed axes. It is no longer necessary to use workarounds with negative units.
\pgfkeys{/pdflinks/search key prefixes in/.add={/pgfplots/,}{}}

	Take a look at the |x dir=reverse| key.

	Existing workarounds will still function properly. Use |\pgfplotsset{compat=1.3}| or higher together with |x dir=reverse| to switch to the new version.
\end{enumerate}

\subsubsection{Old Features Which May Need Attention}
\begin{enumerate}
	\item The |scatter/classes| feature produces proper legends as of version 1.3. This may change the appearance of existing legends of plots with |scatter/classes|.

	\item Starting with \PGFPlots\ $1.1$, |\tikzstyle| should \emph{no longer be used} to set \PGFPlots\ options.
	
	Although |\tikzstyle| is still supported for some older \PGFPlots\ options, you should replace any occurance of |\tikzstyle| with |\pgfplotsset{|\meta{style name}|/.style={|\meta{key-value-list}|}}| or the associated |/.append style| variant. See Section~\ref{sec:styles} for more detail.
\end{enumerate}
I apologize for any inconvenience caused by these changes.

\begin{pgfplotskey}{compat=\mchoice{1.6,1.5.1,1.5,1.4,1.3,pre 1.3,default} (initially default)}
	The preamble configuration 
\begin{codeexample}[code only]
\usepackage{pgfplots}
\pgfplotsset{compat=1.6}
\end{codeexample}
	allows to choose between backwards compatibility and most recent features.

	Occasionally, you might want to use different versions in the same document. Then, provide
\begin{codeexample}[code only]
\begin{figure}
	\pgfplotsset{compat=1.4}
	...
	\caption{...}
\end{figure}
\end{codeexample}
	\noindent in order to restrict the compatibility setting to the actual context (in this case, the |figure| environment).

	The the output of your |.log| file to see the suggested value for |compat|.

	Use |\pgfplotsset{compat=default}| to restore the factory settings.

	Although typically unnecessary, it is also possible to activate only selected changes and keep compatibility to older versions in general:
	\begin{pgfplotskeylist}{%
		compat/path replacement=\meta{version},%
		compat/labels=\meta{version},%
		compat/scaling=\meta{version},%
		compat/scale mode=\meta{version},%
		compat/empty line=\meta{version},%
		compat/plot3graphics=\meta{version},%
		compat/general=\meta{version}%
	}
	Let us assume that we have a document with |\pgfplotsset{compat=1.3}| and you want to keep it this way.

	In addition, you realized that version 1.5.1 supports circles and ellipses. Then, use
\begin{codeexample}[]
% preamble:
\pgfplotsset{compat=1.3,compat/path replacement=1.5.1}
\begin{tikzpicture}
\begin{axis}[
	extra x ticks={-2,2},
	extra y ticks={-2,2},
	extra tick style={grid=major}]
	\addplot {x};
	\draw (axis cs:0,0) circle[radius=2];
\end{axis}
\end{tikzpicture}
\end{codeexample}
	
	All of these keys accept the possible values of the |compat| key.

	The |compat/path replacement| key controls how radii of circles and ellipses are interpreted.

	The |compat/labels| key controls how axis labels are aligned: either uses adjacent to ticks or with an absolute offset.

	The |compat/scaling| key controls some bugfixes introduced in version 1.4 and 1.6: they might introduce slight scaling differences in order to improve the accuracy.

	The |compat/plot3graphics| controls new features for |\addplot3 graphics|.

	The |compat/scale mode| allows to enable/disable the warning ``The content of your 3d axis has CHANGED compared to previous versions'' because the |axis equal| and |unit vector ratio| features where broken for all versions before~1.6 and have been fixed in~1.6.

	The |compat/empty line| allows to write empty lines into input files in order to generate a jump. This requires |compat=1.4| or newer. See |empty line| for details.

	The |compat/general| key currently only activates |log origin|.

	The detailed effects can be seen on the beginning of this section.
	\end{pgfplotskeylist}

	The value \meta{version} can be |default|, |pre 1.3|, |1.3|, |1.4|, |1.5|, |1.5.1|, |1.6|, and |newest|. The value |default| is the same as |pre 1.3| (up to insignificant changes). The use of |newest| is strongly \emph{discouraged}: it might cause changes in your document, depending on the current version of \PGFPlots. Please inspect your |.log| file to see suggestions for the best possible version. 
\end{pgfplotskey}

\subsection{The Team}
\PGFPlots\ has been written mainly by Christian Feuersänger with many improvements of Pascal Wolkotte and Nick Papior Andersen as a spare time project. We hope it is useful and provides valuable plots.

If you are interested in writing something but don't know how, consider reading the auxiliary manual \href{file:TeX-programming-notes.pdf}{TeX-programming-notes.pdf} which comes with \PGFPlots. It is far from complete, but maybe it is a good starting point (at least for more literature).

\subsection{Acknowledgements}
I thank God for all hours of enjoyed programming. I thank Pascal Wolkotte and Nick Papior Andersen for their programming efforts and contributions as part of the development team. I thank J\"urnjakob Dugge for his contribution of |hist/density|, matlab scripts for \verbpdfref{\addplot3} |graphics|, excellent user forum help and helpful bug reports. I thank Stefan Tibus, who contributed the |plot shell| feature. I thank Tom Cashman for the contribution of the |reverse legend| feature. Special thanks go to Stefan Pinnow whose tests of \PGFPlots\ lead to numerous quality improvements. Furthermore, I thank Dr.~Schweitzer for many fruitful discussions and Dr.~Meine for his ideas and suggestions. Special thanks go to Markus B\"ohning for proof-reading all the manuals of \PGF, \PGFPlots, and \PGFPlotstable. Thanks as well to the many international contributors who provided feature requests or identified bugs or simply improvements of the manual!

Last but not least, I thank Till Tantau and Mark Wibrow for their excellent graphics (and more) package \PGF\ and \Tikz, which is the base of \PGFPlots.

%
% main=manual.tex

\subsection{Installation and Prerequisites}
\subsubsection{Licensing}
This program is free software: you can redistribute it and/or modify
it under the terms of the GNU General Public License as published by
the Free Software Foundation, either version 3 of the License, or
(at your option) any later version.

This program is distributed in the hope that it will be useful,
but WITHOUT ANY WARRANTY; without even the implied warranty of
MERCHANTABILITY or FITNESS FOR A PARTICULAR PURPOSE.  See the
GNU General Public License for more details.

A copy of the GNU General Public License can be found in the package file
\begin{verbatim}
doc/latex/pgfplots/gpl-3.0.txt
\end{verbatim}
You may also visit~\url{http://www.gnu.org/licenses}.

\subsubsection{Prerequisites}
\PGFPlots\ requires \PGF\ with \textbf{at least version~$2.0$}. It is used with
\begin{verbatim}
\usepackage{pgfplots}
\end{verbatim}
in your preamble (see Section~\ref{sec:tex:dialects} for information about how to use it with Con{\TeX}t and plain \TeX).


%\subsubsection{Installation}
There are several ways how to teach \TeX\ where to find the files. Choose the option which fits your needs best.

\subsubsection{Installation in Windows}
Windows users often use Mik\TeX\ which downloads the latest stable package versions automatically. You do not need to install anything manually here. 

However, Mik\TeX\ provides a feature to install packages locally in its own \TeX-Directory-Structure (TDS). This is the preferred way if you like to install newer version than those of Mik\TeX. The basic idea is to unzip \PGFPlots\ in a directory of your choice and configure the Mik\TeX\ Package Manager to use this specific directory with higher priority than its default paths. If you want to do this, start the Mik\TeX\ Settings using ``Start $\gg$ Programs $\gg$ Mik\TeX\ $\gg$ Settings''. There, use the ``Roots'' menu section. It contains the Mik\TeX\ Package directory as initial configuration. Use ``Add'' to select the directory in which the unzipped \PGFPlots\ tree resides. Then, move the newly added path to the list's top using the ``Up'' button. Then press ``Ok''. For Mik\TeX\ 2.8, you may need to uncheck the ``Show Mik\TeX-maintained root directories'' button to see the newly installed path.

Mik\TeX\ complains if the provided directory is not TDS conform (see Section~\ref{pgfplots:tds} for details), so you can't provide a wrong directory here. This method does also work for other packages, but some packages may need some directory restructuring before Mik\TeX\ accepts them.

\subsubsection{Installation of Linux Packages}
At the time of this writing, I am unaware of \PGFPlots\ packages for recent stable Linux distributions. For Ubuntu, there are unofficial Ubuntu Package Repositories which can be added to the Ubuntu Package Tools. The idea is: add a simple URL to the Ubuntu Package Tool, run update and the installation takes place automatically. These URLs are maintained as PPA on Ubuntu Servers.

The \PGFPlots\ download area on sourceforge contains recent links about Ubuntu Package Repositories, go to 
\url{http://sourceforge.net/projects/pgfplots/files} 
and download the readme files with recent links.


\subsubsection{Installation in Any Directory - the \texttt{TEXINPUTS} Variable}
You can simply install \PGFPlots\ anywhere on your harddrive, for example into
\begin{verbatim}
/foo/bar/pgfplots.
\end{verbatim}
Then, you set the \texttt{TEXINPUTS} variable to
\begin{verbatim}
TEXINPUTS=/foo/bar/pgfplots//:
\end{verbatim}
The trailing~`\texttt{:}' tells \TeX\ to check the default search paths after \lstinline!/foo/bar/pgfplots!. The double slash~`\texttt{//}' tells \TeX\ to search all subdirectories.

If the \texttt{TEXINPUTS} variable already contains something, you can append the line above to the existing \texttt{TEXINPUTS} content.

Furthermore, you should set |TEXDOCS| as well,
\begin{verbatim}
TEXDOCS=/foo/bar/pgfplots//:
\end{verbatim}
so that the \TeX-documentation system finds the files |pgfplots.pdf| and |pgfplotstable.pdf| (on some systems, it is then enough to use |texdoc pgfplots|).

Please refer to your operating systems manual for how to set environment variables.

\subsubsection{Installation Into a Local TDS Compliant \texttt{texmf}-Directory}
\label{pgfplots:tds}
\PGFPlots\ comes in a ``\TeX\ Directory Structure'' (TDS) conforming directory structure, so you can simply unpack the files into a directory which is searched by \TeX\ automatically. Such directories are |~/texmf| on Linux systems, for example.

Copy \PGFPlots\ to a local \texttt{texmf} directory like \lstinline!~/texmf!. You need at least the \PGFPlots\ directories |tex/generic/pgfplots| and |tex/latex/pgfplots|. Then, run \lstinline!texhash! (or some equivalent path-updating command specific to your \TeX\ distribution). 

The TDS consists of several sub directories which are searched separately, depending on what has been requested: the sub directories |doc/latex/|\meta{package} are used for (\LaTeX) documentation, the sub-directories |doc/generic/|\meta{package} for documentation which apply to \LaTeX\ and other \TeX\ dialects (like plain \TeX\ and Con\TeX t which have their own, respective sub-directories) as well.

Similarly, the |tex/latex/|\meta{package} sub-directories are searched whenever \LaTeX\ packages are requested. The |tex/generic/|\meta{package} sub-directories are searched for packages which work for \LaTeX\ \emph{and} other \TeX\ dialects.

Do not forget to run \lstinline!texhash!.

\subsubsection{Installation If Everything Else Fails...}
If \TeX\ still doesn't find your files, you can copy all \lstinline!.sty! and all |.code.tex|-files (perhaps all |.def| files as well) into your current project's working directory. In fact, you need everything which is in the |tex/latex/pgfplots| and |tex/generic/pgfplots| sub directories.

Please refer to \url{http://www.ctan.org/installationadvice/} for more information about package installation.



\subsection{Troubleshooting -- Error Messages}
This section discusses some problems which may occur when using \PGFPlots.
Some of the error messages are shown in the index, take a look at the end of this manual (under ``Errors'').


\subsubsection{Problems with available Dimen-registers}
To avoid problems with the many required \TeX-registers for \PGF\ and \PGFPlots, you may want to include
\begin{verbatim}
\usepackage{etex}
\end{verbatim}
as first package. This avoids problems with ``no room for a new dimen''
\index{Error Messages!No room for a new dimen}%
in most cases. It should work with any modern installation of \TeX\ (it activates the e-\TeX\ extensions).

\subsubsection{Dimension Too Large Errors}
The core mathematical engine of \PGF\ relies on \TeX\ registers to perform fast arithmetics. To compute $50+299$, it actually computes |50pt+299pt| and strips the |pt| suffix of the result. Since \TeX\ registers can only contain numbers up to $\pm 16384$, overflow error messages like ``Dimension too large'' occur if the result leaves the allowed range. Normally, this should never happen -- \PGFPlots\ uses a floating point unit with data range $\pm 10^{324}$ and performs all mappings automatically. However, there are some cases where this fails. Some of these cases are:
\begin{enumerate}
	\item The axis range (for example, for $x$) becomes \emph{relatively} small. It's no matter if you have absolutely small ranges like $[10^{-17},10^{-16}]$. But if you have an axis range like $[1.99999999,2]$, where a lot of significant digits are necessary, this may be problematic.

	I guess I can't help here: you may need to prepare the data somehow before \PGFPlots\ processes it.

	\item This may happen as well if you only view a very small portion of the data range.

	This happens, for example, if your input data ranges from $x\in [0,10^6]$, and you say |xmax=10|.

	Consider using the |restrict x to domain*=|\meta{min}|:|\meta{max} key in such a case, where the \meta{min} and \meta{max} should be (say) four times of your axis limits (see page~\pageref{key:restrict:x:to:domain} for details).
		
	\item The |axis equal| key will be confused if $x$ and $y$ have a very different scale.
	\item You may have found a bug -- please contact the developers.
\end{enumerate}

\subsubsection{Restrictions for DVI-Viewers and \texttt{dvipdfm}}
\label{sec:drivers}%
\PGF\ is compatible with 
\begin{itemize}
	\item \lstinline!latex!/\lstinline!dvips!,
	\item \lstinline!latex!/\lstinline!dvipdfm!,
	\item \lstinline!pdflatex!,
	\item $\vdots$
\end{itemize}
However, there are some restrictions: I don't know any DVI viewer which is capable of viewing the output of \PGF\ (and therefor \PGFPlots\ as well). After all, DVI has never been designed to draw something different than text and horizontal/vertical lines. You will need to view the postscript file or the pdf-file. 

Then, the DVI/pdf combination doesn't support all types of shadings (for example, the |shader=interp| is only available for |dvips| and |pdftex| drivers).

Furthermore, \PGF\ needs to know a \emph{driver} so that the DVI file can be converted to the desired output. Depending on your system, you need the following options:
\begin{itemize}
	\item \lstinline!latex!/\lstinline!dvips! does not need anything special because \lstinline!dvips! is the default driver if you invoke \lstinline!latex!.
	\item \lstinline!pdflatex! will also work directly because \lstinline!pdflatex! will be detected automatically.
	\item \lstinline!latex!/\lstinline!dvipdfm! requires to use
\begin{verbatim}
\def\pgfsysdriver{pgfsys-dvipdfm.def}
%\def\pgfsysdriver{pgfsys-pdftex.def}
%\def\pgfsysdriver{pgfsys-dvips.def}
\usepackage{pgfplots}.
\end{verbatim}
	The uncommented commands could be used to set other drivers explicitly.
\end{itemize}
Please read the corresponding sections in~\cite[Section 7.2.1 and 7.2.2]{tikz} if you have further questions. These sections also contain limitations of particular drivers.

The choice which won't produce any problems at all is |pdflatex|.

\subsubsection{Problems with \TeX's Memory Capacities}
\PGFPlots\ can handle small up to medium sized plots. However, \TeX\ has never been designed for data plots -- you will eventually face the problem of small memory capacities. See Section~\ref{sec:pgfplots:optimization} for how to enlarge them.

\subsubsection{Problems with Language Settings and Active Characters}
Both \PGF\ and \PGFPlots\ use a lot of active characters -- which may lead to incompatibilities with other packages which define active characters. Compatibility is better than in earlier versions, but may still be an issue. The manual compiles with the |babel| package for english and french, the |german| package does also work. If you experience any trouble, let me know. Sometimes it may work to disable active characters temporarily (|babel| provides such a command).

\subsubsection{Other Problems}
Please read the mailing list at

\url{http://sourceforge.net/projects/pgfplots/support}.

\noindent Perhaps someone has also encountered your problem before, and maybe he came up with a solution.

Please write a note on the mailing list if you have a different problem. In case it is necessary to contact the authors directly, consider the addresses shown on the title page of this document.
%
% main=pgfplots.tex

\section{User's Guide: Drawing Axes and Plots}

The user interface of \PGFPlots\ consists of three components: a |tikzpicture| environment, an |axis| and the |\addplot| command.

Each axis is generated as part of a picture environment (which can be used to annotate plots afterwards, for example). The axis environment encapsulates one or more |\addplot| commands and controls axis-wide settings (like limits, legends, and descriptions). The |\addplot| command supports several coordinate input methods (like table input or mathematical expressions) and allows various sorts of visualization options with straight lines as initial configuration. 

The rest of \PGFPlots\ is a huge set of key--value options to modify the initial configuration or to select plot types. The reference manual has been optimized for electronical display: a lot of examples illustrate the features, and reference documentation can be found by clicking into the sourcecode text fields. Note that most pdf viewers also support to jump back from a hyperlink: for Acrobat Reader, open the menu View$\gg$Toolbars$\gg$More Tools and activate the ``Previous View'' and ``Next View'' buttons (which are under ``Page Navigation Toolbar''). Thus, knowledge of all keys is unnecessary; you can learn them when it is necessary.

To learn \PGFPlots, you should learn about the |\addplot| command and its coordinate input methods. The most important input methods are \verbpdfref{\addplot table} and \verbpdfref{\addplot expression}.

The following sections explain the basics of \PGFPlots, namely how to work with the |\addplot| commands and |axis| environments and how line styles are assigned automatically.

\subsection{\TeX-dialects: \LaTeX, Con{\TeX}t, plain \TeX }
\label{sec:tex:dialects}%
The starting point for \PGFPlots\ is an |axis| enviroment like |axis| or the logarithmic variants |semilogxaxis|, |semilogyaxis| or |loglogaxis|.

Each environment is available for \LaTeX, Con{\TeX}t and plain \TeX:
\begin{description}
\def\HEAD{%
	\small
	%\lstset{boxpos=b,breaklines=false,aboveskip=3pt,belowskip=3pt}%
	%\hspace{-1cm}%
	\begin{tabular}{*{2}{p{4cm}}}%
}%
\item[\LaTeX:] |\usepackage{pgfplots}| and

{\HEAD
\begin{codeexample}[code only]
\begin{tikzpicture}
\begin{axis}
...
\end{axis}
\end{tikzpicture}
\end{codeexample}
&
\begin{codeexample}[code only]
\begin{tikzpicture}
\begin{semilogxaxis}
...
\end{semilogxaxis}
\end{tikzpicture}
\end{codeexample}
\\
\end{tabular}%
}

\begin{codeexample}[code only]
\documentclass[a4paper]{article}

% for dvipdfm:
%\def\pgfsysdriver{pgfsys-dvipdfm.def}
\usepackage{pgfplots}
\pgfplotsset{compat=1.6}% <-- moves axis labels near ticklabels (respects tick label widths)

\begin{document}
\begin{figure}
	\centering
	\begin{tikzpicture}
		\begin{loglogaxis}[xlabel=Cost,ylabel=Error]
		\addplot coordinates {
			(5,     8.31160034e-02)
			(17,    2.54685628e-02)
			(49,    7.40715288e-03)
			(129,   2.10192154e-03)
			(321,   5.87352989e-04)
			(769,   1.62269942e-04)
			(1793,  4.44248889e-05)
			(4097,  1.20714122e-05)
			(9217,  3.26101452e-06)
		};
		\addplot coordinates {
			(7,     8.47178381e-02)
			(31,    3.04409349e-02)
			(111,   1.02214539e-02)
			(351,   3.30346265e-03)
			(1023,  1.03886535e-03)
			(2815,  3.19646457e-04)
			(7423,  9.65789766e-05)
			(18943, 2.87339125e-05)
			(47103, 8.43749881e-06)
		};
		\legend{Case 1,Case 2}
		\end{loglogaxis}
	\end{tikzpicture}
	\caption{A larger example}
\end{figure}
\end{document}
\end{codeexample}

\item[Con{\TeX}t:] |\usemodule[pgfplots]| and

{\HEAD
\begin{codeexample}[code only]
\starttikzpicture
\startaxis
...
\stopaxis
\stoptikzpicture
\end{codeexample}
&
\begin{codeexample}[code only]
\starttikzpicture
\startsemilogxaxis
...
\stopsemilogxaxis
\stoptikzpicture
\end{codeexample}
\\
\end{tabular}%
}

A complete Con{\TeX}t--example file can be found in
\begin{codeexample}[code only]
doc/context/pgfplots/pgfplotsexample.tex.
\end{codeexample}

\item[plain \TeX:] |\input pgfplots.tex| and

{\HEAD
\begin{codeexample}[code only]
\tikzpicture
\axis
...
\endaxis
\endtikzpicture
\end{codeexample}
&
\begin{codeexample}[code only]
\tikzpicture
\semilogxaxis
...
\endsemilogxaxis
\endtikzpicture
\end{codeexample}
\\
\end{tabular}%
}

A complete plain--\TeX--example file can be found in
\begin{codeexample}[code only]
doc/plain/pgfplots/pgfplotsexample.tex.
\end{codeexample}
\end{description}

If you use |latex| / |dvips| or |pdflatex|, no further modifications are necessary. For |dvipdfm|, you should use the |\def\pgfsysdriver| line as indicated above in the examples (see also Section~\ref{sec:drivers}).


\subsection{A First Plot}
Plotting is done using |\begin{axis} ... \addplot ...; \end{axis}|, where |\addplot| is the main interface to perform plotting operations.
\begin{codeexample}[]
\begin{tikzpicture}
	\begin{axis}[
		xlabel=Cost,
		ylabel=Error]
	\addplot[color=red,mark=x] coordinates {
		(2,-2.8559703)
		(3,-3.5301677)
		(4,-4.3050655)
		(5,-5.1413136)
		(6,-6.0322865)
		(7,-6.9675052)
		(8,-7.9377747)
	};
	\end{axis}
\end{tikzpicture}
\end{codeexample}


\begin{codeexample}[]
\begin{tikzpicture}
	\begin{axis}[
		xlabel=$x$,
		ylabel={$f(x) = x^2 - x +4$}
	]
	% use TeX as calculator:
	\addplot {x^2 - x +4};
	\end{axis}
\end{tikzpicture}
\end{codeexample}

\begin{codeexample}[]
\begin{tikzpicture}
	\begin{axis}[
		xlabel=$x$,
		ylabel=$\sin(x)$
	]
	% invoke external gnuplot as
	% calculator:
	\addplot gnuplot[id=sin]{sin(x)};
	\end{axis}
\end{tikzpicture}
\end{codeexample}

The |plot coordinates|, |plot expression| and |plot gnuplot| commands are three of the several supported ways to create plots, see Section~\ref{sec:addplot} for more details\footnote{Please note that you need \lstinline{gnuplot} installed to use \lstinline{plot gnuplot}.} and the remaining ones (|plot file|, |plot shell|, |plot table| and |plot graphics|). The options `|xlabel|' and `|ylabel|' define axis descriptions.

\subsection{Two Plots in the Same Axis}
Multiple |\addplot|-commands can be placed into the same axis, and a |cycle list| is used to automatically select different line styles:
	% generated with this statement:
	%\addplot plot[id=filesuffix_noise,domain=-6:5,samples=10] gnuplot{(-x**5 - 242 + (-300 + 600*rand(0)))};
\begin{codeexample}[leave comments]
\begin{tikzpicture}
	\begin{axis}[
		height=9cm,
		width=9cm,
		grid=major,
	]
		
	\addplot {-x^5 - 242};
	\addlegendentry{model}

	\addplot coordinates {
		(-4.77778,2027.60977)
		(-3.55556,347.84069)
		(-2.33333,22.58953)
		(-1.11111,-493.50066)
		(0.11111,46.66082)
		(1.33333,-205.56286)
		(2.55556,-341.40638)
		(3.77778,-1169.24780)
		(5.00000,-3269.56775)
	};
	\addlegendentry{estimate}
	\end{axis}
\end{tikzpicture}
\end{codeexample}
A legend entry is generated if there are |\addlegendentry| commands (or one |\legend| command).

\subsection{Logarithmic Plots}
Logarithmic plots show $\log x$ versus $\log y$  (or just one logarithmic axis). \PGFPlots\ normally uses the natural logarithm, i.e. basis $e\approx2.718$ (see the key |log basis x|). Now, the axis description also contains minor ticks and the labels are placed at $10^i$.
\begin{codeexample}[]
\begin{tikzpicture}
\begin{loglogaxis}[xlabel=Cost,ylabel=Gain]
\addplot[color=red,mark=x] coordinates {
	(10,100)
	(20,150)
	(40,225)
	(80,340)
	(160,510)
	(320,765)
	(640,1150)
};
\end{loglogaxis}
\end{tikzpicture}
\end{codeexample}
A common application is to visualise scientific data. This is often provided in the format $1.42\cdot10^4$, usually written as 1.42e+04. Suppose we have a numeric table named |pgfplots.testtable|, containing
\begin{codeexample}[code only,tabsize=6]
Level Cost  Error
1     7     8.471e-02
2     31    3.044e-02
3     111   1.022e-02
4     351   3.303e-03
5     1023  1.038e-03
6     2815  3.196e-04
7     7423  9.657e-05
8     18943 2.873e-05
9     47103 8.437e-06
\end{codeexample}
\noindent then we can plot |Cost| versus |Error| using
\begin{codeexample}[]
\begin{tikzpicture}
\begin{loglogaxis}[
	xlabel=Cost,
	ylabel=Error]
\addplot[color=red,mark=x] coordinates {
	(5,    8.31160034e-02)
	(17,   2.54685628e-02)
	(49,   7.40715288e-03)
	(129,  2.10192154e-03)
	(321,  5.87352989e-04)
	(769,  1.62269942e-04)
	(1793, 4.44248889e-05)
	(4097, 1.20714122e-05)
	(9217, 3.26101452e-06)
};

\addplot[color=blue,mark=*] 
	table[x=Cost,y=Error] {pgfplots.testtable};

\legend{Case 1,Case 2}
\end{loglogaxis}
\end{tikzpicture}
\end{codeexample}
The first plot employs inline coordinates; the second one reads numerical data from file and plots column `|Cost|' versus `|Error|'.

\noindent
Besides the environment ``|loglogaxis|'' you can use
\begin{itemize}
	\item |\begin{axis}...\end{axis}| for normal plots,
	\item |\begin{semilogxaxis}...\end{semilogxaxis}| for plots which have a normal~$y$ axis and a logarithmic~$x$ axis,
	\item |\begin{semilogyaxis}...\end{semilogyaxis}| the same with $x$~and~$y$ switched,
	\item |\begin{loglogaxis}...\end{loglogaxis}| for double--logarithmic plots.
\end{itemize}
You can also use
\begin{codeexample}[code only]
\begin{axis}[xmode=normal,ymode=log]
...
\end{axis}
\end{codeexample}
which is the same as |\begin{semilogyaxis}...\end{semilogyaxis}|.
\begin{codeexample}[]
\begin{tikzpicture}
	\begin{semilogyaxis}[
		xlabel=Index,ylabel=Value]

	\addplot[color=blue,mark=*] coordinates {
		(1,8)
		(2,16)
		(3,32)
		(4,64)
		(5,128)
		(6,256)
		(7,512)
	};
	\end{semilogyaxis}%
\end{tikzpicture}%
\end{codeexample}

\subsection{Cycling Line Styles}
You can skip the style arguments for |\addplot[...]| to determine plot specifications from a predefined list:
\label{page:plotcoords:src}%
\pgfmanualpdflabel{\textbackslash plotcoords}{}%
\begin{codeexample}[width=4cm]
\begin{tikzpicture}
\begin{loglogaxis}[
	xlabel={Degrees of freedom},
	ylabel={$L_2$ Error}
]
\addplot coordinates {
	(5,8.312e-02)    (17,2.547e-02)   (49,7.407e-03)
	(129,2.102e-03)  (321,5.874e-04)  (769,1.623e-04)
	(1793,4.442e-05) (4097,1.207e-05) (9217,3.261e-06)
};

\addplot coordinates{
	(7,8.472e-02)    (31,3.044e-02)    (111,1.022e-02)
	(351,3.303e-03)  (1023,1.039e-03)  (2815,3.196e-04)
	(7423,9.658e-05) (18943,2.873e-05) (47103,8.437e-06)
};

\addplot coordinates{
	(9,7.881e-02)     (49,3.243e-02)    (209,1.232e-02)
	(769,4.454e-03)   (2561,1.551e-03)  (7937,5.236e-04)
	(23297,1.723e-04) (65537,5.545e-05) (178177,1.751e-05)
};

\addplot coordinates{
	(11,6.887e-02)    (71,3.177e-02)     (351,1.341e-02)
	(1471,5.334e-03)  (5503,2.027e-03)   (18943,7.415e-04)
	(61183,2.628e-04) (187903,9.063e-05) (553983,3.053e-05)
};

\addplot coordinates{
	(13,5.755e-02)     (97,2.925e-02)     (545,1.351e-02)
	(2561,5.842e-03)   (10625,2.397e-03)  (40193,9.414e-04)
	(141569,3.564e-04) (471041,1.308e-04) (1496065,4.670e-05)
};
\legend{$d=2$,$d=3$,$d=4$,$d=5$,$d=6$}
\end{loglogaxis}
\end{tikzpicture}
\end{codeexample}
\noindent
The |cycle list| can be modified, see the reference below.

\subsection{Scaling Plots}
You can use any of the \Tikz\ options to modify the appearance. For example, the ``|scale|'' transformation takes the picture as such and scales it (just like |\includegraphics|):

\begin{codeexample}[]
\begin{tikzpicture}[scale=0.5]
	\begin{loglogaxis}[
		xlabel={Degrees of freedom},
		ylabel={$L_2$ Error}
	]
	\plotcoords
	\legend{$d=2$,$d=3$,$d=4$,$d=5$,$d=6$}
	\end{loglogaxis}
\end{tikzpicture}

\begin{tikzpicture}[scale=1.1]
	\begin{loglogaxis}[
		xlabel={Degrees of freedom},
		ylabel={$L_2$ Error}
	]
	\plotcoords
	\legend{$d=2$,$d=3$,$d=4$,$d=5$,$d=6$}
	\end{loglogaxis}
\end{tikzpicture}
\end{codeexample}
However, you can also scale plots by assigning a |width=5cm| and/or |height=3cm| argument. This only affects the distance of point coordinates, no font sizes or axis descriptions:
\begin{codeexample}[]
\begin{tikzpicture}
	\begin{loglogaxis}[
		width=6cm,
		xlabel={Degrees of freedom},
		ylabel={$L_2$ Error}
	]
	\plotcoords
	\legend{$d=2$,$d=3$,$d=4$,$d=5$,$d=6$}
	\end{loglogaxis}
\end{tikzpicture}

\begin{tikzpicture}
	\begin{loglogaxis}[
		width=8cm,
		xlabel={Degrees of freedom},
		ylabel={$L_2$ Error}
	]
	\plotcoords
	\legend{$d=2$,$d=3$,$d=4$,$d=5$,$d=6$}
	\end{loglogaxis}
\end{tikzpicture}
\end{codeexample}

Use the predefined styles |normalsize|, |small|, |footnotesize| to adopt font sizes and ticks automatically. Use the |/pgfplots/scale| key to rescale the axis without affecting fonts.

\endinput
%
% main=manual.tex

\section{The Reference}
{
\tikzset{external/figure name/.add={}{reference_}}%

\input pgfplots.reference.axis-addplot.tex
\input pgfplots.reference.preliminaryoptions.tex
\input pgfplots.reference.2dplots.tex
\input pgfplots.reference.3dplots.tex
\input pgfplots.reference.markers-meta.tex
\input pgfplots.reference.axisdescription.tex
\input pgfplots.reference.scaling.tex
\input pgfplots.reference.3dconfiguration.tex
\input pgfplots.reference.errorbars.tex
\input pgfplots.reference.numberformatting.tex
\input pgfplots.reference.specifyrange.tex
\input pgfplots.reference.tickoptions.tex
\input pgfplots.reference.gridoptions-axiscoordinates.tex
\input pgfplots.reference.styleoptions.tex
\input pgfplots.reference.alignment.tex
\input pgfplots.reference.closingplots.tex
\input pgfplots.reference.symbolic-transformations.tex
\input pgfplots.reference.coordfiltering.tex
\input pgfplots.reference.transformations.tex
\input pgfplots.reference.linefitting.tex
\input pgfplots.reference.miscellaneous.tex
\input pgfplots.reference.tikzinteroperability.tex
\input pgfplots.reference.layers.tex
\input pgfplots.reference.technicalinternals.tex

}
%
\section{Related Libraries}
This section describes some libraries which come with \PGFPlots, but they are more or less special and need to be activated separately.
\pgfmanualpdflabel{\textbackslash usepgfplotslibrary}{}


\input pgfplots.libs.clickable.tex
\input pgfplots.libs.colormaps.tex
\input pgfplots.libs.dateplot.tex
\input pgfplots.libs.external.tex
\input pgfplots.libs.groupplot.tex
\input pgfplots.libs.patchplots.tex
\input pgfplots.libs.polar.tex
\input pgfplots.libs.smithchart.tex
\input pgfplots.libs.ternary.tex
\input pgfplots.libs.units.tex

%
\section{Memory and Speed considerations}
{
\tikzset{external/figure name/.add={}{memspeed_}}%
\subsection{Memory Limits of \TeX}
\label{sec:pgfplots:optimization}
\PGFPlots\ can typeset plots with several thousand points if memory limits of \TeX\ are configured properly. Its runtime is roughly proportional to the number of input points\footnote{In fact, the runtime is pseudo--linear: starting with about $100{,}000$ points, it will become quadratic. This limitation applies to the path length of \PGF\ paths as well. Furthermore, the linear runtime is not possible yet for stacked plots.}.

\pgfplotsexpensiveexample
\begin{codeexample}[]
\begin{tikzpicture}
\begin{axis}[
	enlargelimits=0.01,
	title style={yshift=5pt},
	title=Scatter plot with $2250$ points]
	
\addplot[blue,
	mark=*,only marks,mark options={scale=0.3}]
	file[skip first]
	{plotdata/pgfplots_scatterdata3.dat};
	
\end{axis}
\end{tikzpicture}
\end{codeexample}

\pgfplotsexpensiveexample
\begin{codeexample}[]
\begin{tikzpicture}
\begin{axis}[
	enlarge x limits=0.03,
	title=Ornstein-Uhlenbeck sample
		($13000$ time steps),
	xlabel=$t$]
	
\addplot[blue] file {plotdata/ou.dat};
\end{axis}
\end{tikzpicture}
\end{codeexample}

\pgfplotsexpensiveexample
\begin{codeexample}[]
% huger graphs are possible; consider lualatex
\begin{tikzpicture}
\begin{axis}[
	title=$80 \times 80$ Smooth Surface,
	xlabel=$x$,
	ylabel=$y$]
\addplot3[surf,samples=80,shader=interp,domain=0:1] 
	{sin(deg(8*pi*x))* exp(-20*(y-0.5)^2) 
	+ exp(-(x-0.5)^2*30 
		- (y-0.25)^2 - (x-0.5)*(y-0.25))};
\end{axis}
\end{tikzpicture}
\end{codeexample}

\PGFPlots\ relies completely on \TeX\ to do all typesetting. It uses the front-end-layer and basic layer of \PGF\ to perform all drawing operations. For complicated plots, this may take some time, and you may want to read Section~\ref{sec:pgfplots:importexport} for how to write single figures to external graphics files. Externalization is the best way to reduce typesetting time.

However, for large scale plots with a lot of points, limitations of \TeX's capacities are reached easily.

\subsection{Memory Limitations}
The default settings of most \TeX-distributions are quite restrictive, so it may be necessary to adjust them. 

Usually, the log--file or the final error message contains a summary about the used resources, giving a hint which parameter needs to be increased.

\subsubsection{LuaLa\TeX}
One solution which works quite well is to switch the La\TeX\ executable: if you have a decent \TeX\ distribution, you will have the |lualatex| executable as well. This, in turn, uses dynamic memory allocation such that it usually has enough memory for any \PGFPlots\ axis.

The LuaLa\TeX\ executable |lualatex| is supposed to be almost compatible with |pdflatex|. 

This approach works for any platform.

\subsubsection{Mik\TeX}
If you are running Mik\TeX\ and you do not want to (or cannot switch) to |lualatex|, you can proceed as follows.

For Mik\TeX, memory limits can be increased in two ways. The first is to use command line switches:
\begin{codeexample}[code only]
pdflatex 
	--stack-size=n --save-size=n 
	--main-memory=n --extra-mem-top=n --extra-mem-bot=n
	--pool-size=n --max-strings=n 
\end{codeexample}
\noindent Experiment with these settings if Mik\TeX\ runs out of memory. Usually, one doesn't invoke |pdflatex| manually: there is a development aid which does all the invocations, so this one needs to be adjusted. 

Sometimes it might be better to adjust the Mik\TeX\ configuration file permanently, for example to avoid reconfiguring the \TeX\ development program. This can be implemented using the command
\begin{codeexample}[code only]
initexmf --edit-config-file=pdflatex
\end{codeexample}
\noindent which can be typed either on a command prompt in Windows or using Start $\gg$ Execute. As a result, an editor will be opened with the correct config file. A sample config file could be
\begin{codeexample}[code only]
main_memory=90000000
save_size=80000
\end{codeexample}
or any of the config file entries which are listed below can be entered. 
Thanks to ``LeSpocky'' for his documentation in

\url{http://blog.antiblau.de/2009/04/21/speicherlimits-von-miktex-erhoehen}.

\subsubsection{\TeX Live or similar installations}
In addition to the option to switch to |lualatex|, you can proceed as follows to keep existing |dvips| or |pdflatex| workflows.

For Unix installations, one needs to adjust config files. This can be done as follows:
\begin{enumerate}
	\item Locate |texmf.cnf| on your system. On my Ubuntu installation, it is in 
	
	|/usr/share/texmf/web2c/texmf.cnf|.
	\item Either change |texmf.cnf| directly, or copy it to some convenient place. If you copy it, here is how to proceed:
		\begin{itemize}
			\item keep only the changed entries in your local copy to reduce conflicts. \TeX\ will always read \emph{all} config files found in its search path.
			\item Adjust the search path to find your local copy. This can be done using the environment variable |TEXMFCNF|. Assuming your local copy is in |~/texmf/mytexcnf/texmf.cnf|, you can write
\begin{codeexample}[code only]
export TEXMFCNF=~/texmf/mytexcnf:
\end{codeexample}
			to search first in your directory, then in all other system directories.
		\end{itemize}
	\item You should change the entries
\begin{codeexample}[code only]
main_memory = n
extra_mem_top = n
extra_mem_bot = n
max_strings = n
param_size = n
save_size = n
stack_size = n
\end{codeexample}
		The log--file usually contains information about the parameter which needs to be enlarged.
\end{enumerate}
An example of this config file thing is shown below. It changes memory limits.
\begin{enumerate}
	\item Create the file |~/texmf/mytexcnf/texmf.cnf| (and possibly the paths as well).
\begin{codeexample}[code only]
% newly created file ~/texmf/mytexcnf/texmf.cnf:
% If you want to change some of these sizes only for a certain TeX
% variant, the usual dot notation works, e.g.,
% main_memory.hugetex = 20000000
main_memory = 230000000 % words of inimemory available; also applies to inimf&mp
extra_mem_top = 10000000     % extra high memory for chars, tokens, etc.
extra_mem_bot = 10000000     % extra low memory for boxes, glue, breakpoints, etc.
save_size = 150000	% for saving values outside current group
stack_size = 150000	% simultaneous input sources

% Max number of characters in all strings, including all error messages,
% help texts, font names, control sequences.  These values apply to TeX and MP.
%pool_size = 1250000
% Minimum pool space after TeX/MP's own strings; must be at least
% 25000 less than pool_size, but doesn't need to be nearly that large.
%string_vacancies = 90000
% Maximum number of strings.
%max_strings = 100000
% min pool space left after loading .fmt
%pool_free = 47500
\end{codeexample}
	\item Run |texhash| such that \TeX\ updates its |~/texmf/ls-R| database.
	\item Create the environment variable |TEXMFCNF| and assign the value `|~/texmf/mytexcnf:|' (including the trailing `|:|'!). For my linux system, this can be done using by adding
\begin{codeexample}[code only]
export TEXMFCNF=~/texmf/mytexcnf:
\end{codeexample}
	to |~/.bashrc|.
\end{enumerate}

Unfortunately, \TeX\ does not allow arbitrary memory limits, there is an upper bound hard coded in the executables.

\subsection{Reducing Typesetting Time}
\PGFPlots\ does a lot of computations ranging from abstract coordinate computations to low level |.pdf| drawing commands (implemented by \PGF). For complex plots, this may take a considerable time -- especially for 3D plots.

One possibility to reduce typesetting time is to tell \PGF\ to generate single, temporary |.pdf| (or |.eps|) documents for a subset (or all) graphics in one run and re-use these temporary images in successive runs. For \PGFPlots, this is the most effective way to reduce typesetting time. It can be accomplished using the |external| library described in Section~\ref{sec:pgfplots:export}.
}
%

\section{Import/Export From Other Formats}
{
\tikzset{external/figure name/.add={}{importexport_}}%
\label{sec:pgfplots:importexport}
This section contains information of how to single pictures into separate \pdf\ graphics files (or \eps\ graphics files). Furthermore, it explains a matlab (\textregistered) script which allows to convert from matlab to \PGFPlots.

\subsection[Export to pdf/eps]{Export to {\normalfont\pdf/\eps}}
\label{sec:pgfplots:export}
It is possible to export images to single \pdf-documents using routines of \pgfname\ and/or \Tikz.

\subsubsection{Using the Automatic Externalization Framework of \Tikz}
\begin{pgfplotslibrary}{external}
\pgfkeys{
	/pdflinks/search key prefixes in/.add={/tikz/external/,}{}
}
	The |external| library offers a convenient method to export every single |tikzpicture| into a separate~|.pdf| (or~|.eps|). Later runs of \LaTeX\ will simply include these graphics, thereby reducing typesetting time considerably.

	The library can also be used to submit documents to authors who do not even have \PGFPlots\ or \Tikz\ installed.

	\paragraph{Technical foreword:}
	The |external| library has been written by Christian Feuers\"anger (author of \PGFPlots). It has been contributed to \Tikz\ as general purpose library, so the reference documentation along with all tweaks can be found in~\cite[Section ``Externalization Library'']{tikz}. The command |\usepgfplotslibrary{external}| is actually just a wrapper which loads |\usetikzlibrary{external}| or, if this library does not yet exist because the installed \pgfname\ has at most version $2.00$, it will load a copy which is shipped with \PGFPlots.

	The |external| library has been designed such that \emph{no changes} to the document as such are necessary. The idea is as follows:
\begin{enumerate}
	\item Every |\begin{tikzpicture}| $\dotsc$ |\end{tikzpicture}| gets a file name. The file name can be assigned manually with |\tikzsetnextfilename|\marg{output file name} or automatically, in which case \meta{tex file name}|-figure|\meta{number} is used with an increasing \meta{number}.
	
	\item The library writes the resulting images using system calls of the form |pdflatex --jobname |\meta{output file name} automatically, using the write18 system call of \TeX. It is the same framework which can be used to call |gnuplot|.
\end{enumerate}
The only steps which are necessary is to use

\pgfmanualpdflabel{\textbackslash tikzexternalize}{}%
|\usepgfplotslibrary{external}|

|\tikzexternalize|

\noindent somewhere in your document's preamble (see below for system-dependent configuration options). No further modification to the document is necessary. Suppose we have a file called |test.tex|:
\begin{codeexample}[code only]
\documentclass{article}

\usepackage{pgfplots}

\usepgfplotslibrary{external}
\tikzexternalize% activate externalization!

\begin{document}
	\begin{figure}
		\begin{tikzpicture}
		\begin{axis}
			\addplot {x^2};
		\end{axis}
		\end{tikzpicture}
	\caption{Our first external graphics example}
	\end{figure}

	\begin{figure}
		\begin{tikzpicture}
		\begin{axis}
			\addplot {x^3};
		\end{axis}
		\end{tikzpicture}
	\caption{A second graphics}
	\end{figure}
\end{document}
\end{codeexample}
\noindent To enable the system calls, we type
\begin{codeexample}[code only]
pdflatex -shell-escape test
\end{codeexample}
\noindent and \LaTeX\ will now generate the required graphics files |test-figure0.pdf| and |test-figure1.pdf| automatically. Any further call to |pdflatex| will simply use |\includegraphics| and the |tikzpicture|s as such are no longer considered (you need a different command line switch for Mik\TeX, see the |shell escape| option).

If a figure shall be remade, one can simply delete all or selected graphics files and regenerate them. Alternatively, one can use the command |\tikzset{external/force remake}| somewhere in the document to remake every following picture automatically.

There are three ways to modify the file names of externalized figures:
\begin{itemize}
	\item Changing the overall file name using a |prefix|,
	\item Changing the file name for a single figure using |\tikzsetnextfilename|,
	\item Changing the file name for a restricted set of figures using |figure name|.
\end{itemize}
\begin{key}{/tikz/external/prefix=\marg{file name prefix} (initially empty)}
	A shortcut for |\tikzsetexternalprefix|\marg{file name prefix}, see below.
\end{key}

\begin{command}{\tikzsetexternalprefix\marg{file name prefix}}
	Assigns a common prefix used by all file names. For example,
\begin{codeexample}[code only]
\tikzsetexternalprefix{figures/}
\end{codeexample}
	will prepend |figures/| to every external graphics file name.
\end{command}

\begin{command}{\tikzsetnextfilename\marg{file name}}
	Sets the file name for the \emph{next} \tikzname\ picture or |\tikz| short command. It will \emph{only} be used for the next picture.

	Pictures for which no explicit file name has been set will get automatically generated file names.

	Please note that |prefix| will still be prepended to \meta{file name}.
\begin{codeexample}[code only]
\documentclass{article}
% main document, called main.tex
\usepackage{tikz}

\usepgfplotslibrary{external}
\tikzexternalize[prefix=figures/]% activate with a name prefix

\begin{document}

\tikzsetnextfilename{firstplot}
\begin{tikzpicture} % will be written to 'figures/firstplot.pdf'
\begin{axis}
	\addplot {x};  	
\end{axis}
\end{tikzpicture}

\begin{tikzpicture} % will be written to 'figures/main-figure0.pdf'
   \draw[help lines] (0,0) grid (5,5);
\end{tikzpicture}
\end{document}
\end{codeexample}
\begin{codeexample}[code only]
pdflatex -shell-escape main
\end{codeexample}
\end{command}

\begin{key}{/tikz/external/figure name=\marg{name}}
	Same as |\tikzsetfigurename|\marg{name}.
\end{key}
\begin{command}{\tikzsetfigurename\marg{name}}
	Changes the names of \emph{all} following figures. It is possible to change |figure name| during the document using |\tikzset{external/figure name|=\marg{name}|}|. A unique counter\footnote{These counters are stored into different \emph{macros}. In other words: no \TeX\ register will be needed.} will be used for each different \meta{name}, and each counter will start at $0$.

	The value of |prefix| will be applied after |figure name| has been evaluated.
\begin{codeexample}[code only]
\documentclass{article}
% main document, called main.tex
\usepackage{tikz}

\usepgfplotslibrary{external}
\tikzexternalize% activate externalization!

\begin{document}

% will be written to 'main-figure0.pdf'
\begin{tikzpicture} 
\begin{semilogyaxis}
	\addplot {exp(x)};
\end{semilogyaxis}
\end{tikzpicture}

{
  \tikzset{external/figure name={subset_}}
  A simple image is \tikz \fill (0,0) circle(5pt);. % will be written to 'subset_0.pdf'

  \begin{tikzpicture} % will be written to 'subset_1.pdf'
     \begin{axis}
	 	\addplot {x^2};
	\end{axis}
  \end{tikzpicture}
}% here, the old file name will be restored:

\begin{tikzpicture} % will be written to 'main-figure1.pdf'
   \begin{axis}
   		\addplot[domain=1e-3:100] {1/x};
	\end{axis}
\end{tikzpicture}
\end{document}
\end{codeexample}
	The scope of |figure name| ends with the next closing brace (as all values set by |\tikzset| do).

	\medbreak
	Remark: Use |\tikzset{external/figure name/.add=|\marg{prefix}\marg{suffix}|}| to prepend  a \meta{prefix} and append a \meta{suffix} to the actual value of |figure name|. Might be useful for something like
\begin{codeexample}[code only]
\tikzset{external/figure name=main}

% uses main_0.pdf, main_1.pdf, ...

\section{The first section}
{\tikzset{external/figure name/.add={}{_firstsection}}
	...
	% uses main_firstsection_0.pdf, main_firstsection_1.pdf, ...
}

\section{The second section}
{\tikzset{external/figure name/.add={}{secondsection_}}
	...
	% uses main_secondsection_0.pdf, main_secondsection_1.pdf, ...
	\subsection{Second subsection}
	{\tikzset{external/figure name/.add={}{sub_}}
		...
		% uses main_secondsection_sub_0.pdf, main_secondsection_sub_1.pdf, ...
	}
	% uses main_secondsection_2.pdf, main_secondsection_3.pdf, ...
}
\end{codeexample}
\end{command}

\begin{command}{\tikzappendtofigurename\marg{suffix}}
	Appends \meta{suffix} to the actual value of |figure name|.

	It is a shortcut for |\tikzset{external/figure name/.add={}|\marg{suffix}|}| (a shortcut which is also supported if \tikzname\ is not installed, see below).
\end{command}


\paragraph{Configuration option for \textsc{eps} output or Mik\TeX:} Since the |external| lib works by means of system calls, it has to be modified to fit the local system. This is necessary for Mik\TeX\ since it uses a different option to enable these system calls. It is also necessary for \textsc{eps} output since this involves a different set of utilities.

Note that the \emph{most important part} is to enable system calls. This is typically done by typesetting your document with |pdflatex -shell-escape| or |pdflatex -enable-write18| (Mik\TeX). These options \emph{need to be configured in your \TeX\ editor}.
Besides this step, one may want to configure the system call:

\begin{key}{/tikz/external/system call=\marg{template}}
\label{extlib:systemcall:option}
	A template string used to generate system calls. Inside of \meta{template}, the macro |\image| can be used as placeholder for the image which is about to be generated while |\texsource| contains the main file name (in truth, it contains |\input|\marg{main file name}, but that doesn't matter).

	The default is 
\begin{codeexample}[code only]
\tikzset{external/system call={pdflatex \tikzexternalcheckshellescape -halt-on-error 
    -interaction=batchmode -jobname "\image" "\texsource"}
\end{codeexample}
	\noindent where \declareandlabel{\tikzexternalcheckshellescape} inserts the value of the configuration key |shell escape|
	if and only if the current document has been typeset with |-shell-escape|\footnote{Note that this is always true for the default configuration. This security consideration applies mainly for \texttt{mode=list and make} which will also work \emph{without} shell escapes.}.

	For |eps| output, you can (and need to) use
\begin{codeexample}[code only]
\tikzset{external/system call={latex \tikzexternalcheckshellescape -halt-on-error
    -interaction=batchmode -jobname "\image" "\texsource" &&
    dvips -o "\image".ps "\image".dvi}}
\end{codeexample}
	
	The argument \meta{template} will be expanded using |\edef|, so any control sequences will be expanded. During this evaluation, `|\\|' will result in a normal backslash, `|\|'. Furthermore, double quotes `|"|', single quotes `|'|', semicolons and dashes `|-|' will be made to normal characters if any package uses them as macros. This ensures compatibility with the |german| package, for example.
\end{key}

\begin{key}{/tikz/external/shell escape=\marg{command-line arg} (initially -shell-escape)}
	Contains the command line option for |latex| which enables the |\write18| feature. 
	
	For \TeX-Live, this is |-shell-escape|. For Mik\TeX, you should use |\tikzexternalize[shell escape=-enable-write18]|.
\end{key}

\paragraph{Support for Labels and References In External Files}
The |external| library comes with extra support for |\label| and |\ref| (and other commands which usually store information in the |.aux| file) inside of external files.

There are, however, some points which need your attention when you try to use 
\begin{enumerate}
	\item[a)] |\ref| to something in the main document inside of an externalized graphics or
	\item[b)] |\label| in the externalized graphics which is referenced in the main document.
\end{enumerate}

For point a), a |\ref| inside of an externalized graphics works \emph{only} if you issue the required system call \emph{manually} or by |make|. The initial configuration |mode=convert with system call| does \emph{not} support |\ref|. But you can copy--paste the system call generated by |mode=convert with system call| and issue it manually. The reason is that |\ref| information is stored in the main |.aux| file -- but this auxiliary file is not completely written when |mode=convert with system call| is invoked (there is a race condition). Note that |\pageref| is not supported (sorry). Thus: if you have |\ref| inside of external graphics, consider using |mode=list and make| or copy--paste the system call for the image(s) and issue it manually.

Point b) is realized automatically by the external library. In detail, a |\label| inside of an externalized graphics causes the external library to generate separate auxiliary files for every external image. These files are called \meta{imagename}|.dpth|. The extension |.dpth| indicates that the file also contains the image's depth (the |baseline| key of \tikzname). Furthermore, anything which would have been written to an |.aux| file will be redirected to the |.dpth| file -- but only things which occur inside of the externalized |tikzpicture| environment. When the main document loads the image, it will copy the |.dpth| file into the main |.aux| file. Then, successive compilations of the main document contain the external |\label| information. In other words, a |\label| in an external graphics needs the following work flow:
\begin{enumerate}
	\item The external graphics needs to be generated together with its |.dpth| (usually automatically by \tikzname).
	\item The main document includes the external graphics and copies the |.dpth| content into its main |.aux| file.
	\item The main document needs to be translated once again to re-read its |.aux| file\footnote{Note that it is not possible to activate the content of an auxiliary file after \texttt{\textbackslash begin\{document\}} in \LaTeX.}.
\end{enumerate}
There is just a special case if a |\label|/|\ref| drawn as a |tikzpicture|. This is, for example, the case for the legend |\ref| images or for the |\pgfplotslegendfromname| feature. In such cases, you need to proceed as for case a) since |mode=convert with system call| can't handle that stuff on its own. 

In other words: a |\label| in an external document works automatically, just translate the main document often enough. A |\ref| might need manual adjustments as described for case a) above.


\paragraph{Operation Modes}
\begin{key}{/tikz/external/mode=\mchoice{convert with system call,list and make,$\dotsc$} (initially convert with system call)}
	This allows to change the default operation mode. There are a handful of choices possible, all of them are described in detail in~\cite[section ``Externalization Library'']{tikz}. The most useful ones are probably the initial configuration |convert with system call| and the specialized choice |list and make|.
	
	The choice |list and make| configures the library to check if there are already external graphics and uses them. If there are no graphics, the library will \emph{skip} the figure. However, it will also generate a |makefile| to generate the graphics, and a list of all required graphics files.

	It is not required to use |make|: the library expects you to generate the images somehow and it doesn't care about the ``how''. Using |make -f |\meta{name-of-tex-file}|.makefile -j 2| allows parallel execution which might, indeed, be an option. Furthermore, the makefile also supports file dependencies: if one of your data tables has been updated, the external graphics will be remade automatically. \PGFPlots\ tells the external library about any file dependencies (input files and tables).

	The two modes have the following characteristics:
	\begin{enumerate}
		\item |convert with system call| is automatic and does everything on--the--fly. However, it \emph{can't} work with |\ref| and/or |\label| information in external pictures.
		\item |list and make| requires either manual (by issuing the system calls manually) or semi--automatic conversion (using the generated \meta{main}|.makefile|), and multiple runs of |pdflatex|. The generated Makefile can be processed in parallel. Furthermore, |list and make| provides \emph{full support} for |\ref| and |\label|: any |\label| defined inside of an externalized graphics is still available for the main document.
		
		If you have legends with |legend to name| or |\label|/|\ref|, you need to generate the graphics defining the |\label| (or |legend to name|), then run |pdflatex| twice on the main document. Afterwards, you can externalize the legend graphics.
	\end{enumerate}
\end{key}

The complete reference documentation and remaining options are documented in~\cite[``Externalization Library'']{tikz}. This reference also contains information about
\begin{itemize}
	\item how to use |\tikzset{external/|\declareandlabel{force remake}|}| and |\tikzset{external/|\declareandlabel{remake next}|}| to remake selected figures,
	\item how to disable the externalization partially with |\tikzset{external/|\declareandlabel{export}|=false}| or completely with |\tikzexternaldisable|,
	\item how to optimize the speed of the conversion process using |\tikzset{external/optimize command away=\myExpensiveMacro}|,
	\item how to add further remake-dependencies with |\tikzpicturedependsonfile|\marg{name} and/or  |\tikzexternalfiledependsonfile|\marg{external file}\marg{name},
	\item examples how to enable |png| export,
	\item how to typeset such a document without \pgfname\ installed or
	\item how to provide work-arounds with |.pdf| images and bounding box restrictions.
\index{External Graphics!Bounding Box Issues}
\index{Bounding Box Control!Image Externalization Problems}
\end{itemize}

\paragraph{Using the Library Without {\normalfont\pgfname} or {\normalfont\PGFPlots} Installed}
There is a small replacement package \declareandlabel{tikzexternal.sty} which can be used once every figure has been exported. The idea is to uncomment |\usepackage{tikz}| and |\usepackage{pgfplots}| and write |\usepackage{tikzexternal}| instead:
\begin{codeexample}[code only]
% \usepackage{tikz}
% \usepackage{pgfplots}
\usepackage{tikzexternal}
\tikzexternalize% activate externalization

\begin{document}
\begin{tikzpicture}
	...
\end{tikzpicture}
...
\end{document}
\end{codeexample}
You do not need \pgfname, \tikzname\ or \PGFPlots\ installed. What you need is |tikzexternal.sty| and all generated figures (consisting of the image files, `|.pdf|' and the `|.dpth|' files containing information of the |baseline| option). The file |tikzexternal.sty| is shipped with \pgfname\ in the directory
\begin{codeexample}[code only]
latex/pgf/utilities/tikzexternal.sty
\end{codeexample}
and a copy is shipped with \PGFPlots\ in
\begin{codeexample}[code only]
tex/generic/pgfplots/oldpgfcompatib/pgfplotsoldpgfsupp_tikzexternal.sty
\end{codeexample}
Just copy the file into your directory and rename it to |tikzexternal.sty|.

\paragraph{Attention:} The small replacement package doesn't support key--value interfaces. Thus, it is necessary to use |\tikzsetexternalprefix| instead of the |prefix| option and |\tikzsetfigurename| instead of the |figure name| option since |\tikzset| is not available in such a context. Also, you may want to define a dummy--macro |\pgfplotsset| if you have used |\pgfplotsset|.
\end{pgfplotslibrary}

\subsubsection[Using the Externalization Framework of PGF By Hand]{Using the Externalization Framework of {\normalfont\pgfname} ``By Hand''}
Another way to export \TeX-pictures to single graphics files is to use the externalization framework of \pgfname, which requires more work but works more generally than the |external| library.
The basic idea is to encapsulate the desired parts with

\declareandlabel{\beginpgfgraphicnamed}\marg{output file name}

\meta{picture contents}

\declareandlabel{\endpgfgraphicnamed}. 

\noindent Furthermore, one needs to tell \pgfname\ the name of the main document using

\declareandlabel{\pgfrealjobname}\marg{the real job's name}

\noindent in the preamble. This enables two different modes: 
\begin{enumerate}
	\item The first is the normal typesetting mode. \LaTeX\ checks whether a file named \meta{output file name} with one of the accepted file extensions exists -- if that is the case, the graphics file is included with |\pgfimage| and the \meta{picture contents} is skipped. If no such file exists, the \meta{picture contents} is typeset normally. This mode is applied if |\jobname| equals \meta{the real job's name}.
	\item The second mode applies if |\jobname| equals \meta{output file name}, it initiates the ``conversion mode'' which is used to write the graphics file \meta{output file name}. In this case, \emph{only} \meta{picture contents} is written to |\jobname|, the complete rest of the \LaTeX\ is processed as normal, but it is silently discarded.

	This mode needs to be started manually with |pdflatex --jobname |\meta{output file name} for every externalized graphics file.
\end{enumerate}
A complete example may look as follows.
\begin{codeexample}[code only]
\documentclass{article}

\usepackage{pgfplots}

\pgfrealjobname{test}

\begin{document}
	\begin{figure}
		\beginpgfgraphicnamed{testfigure}
		\begin{tikzpicture}
		\begin{axis}
			\addplot {x^2};
		\end{axis}
		\end{tikzpicture}
		\endpgfgraphicnamed
	\caption{Our first external graphics example}
	\end{figure}

	\begin{figure}
		\beginpgfgraphicnamed{testfigure2}
		\begin{tikzpicture}
		\begin{axis}
			\addplot {x^3};
		\end{axis}
		\end{tikzpicture}
		\endpgfgraphicnamed
	\caption{A second graphics}
	\end{figure}
\end{document}
\end{codeexample}
\noindent The file is named |test.tex|, and it is processed (for example) with
\begin{codeexample}[code only]
pdflatex test	
\end{codeexample}
\noindent Now, we type
\begin{codeexample}[code only]
pdflatex --jobname testfigure test	
pdflatex --jobname testfigure2 test	
\end{codeexample}
\noindent to enter conversion mode. These last calls will \emph{only} write the contents of our named graphics environments, one for \meta{testfigure} and one for \meta{testfigure2} into the respective output files |testfigure.pdf| and |testfigure2.pdf|.

In summary, one needs |\pgfrealjobname| and calls |pdflatex --jobname |\meta{graphics file} for every externalized graphics environment. Please note that it is absolutely necessary to use the syntax above, \emph{not} |\begin{pgfgraphicnamed}|.

These steps are explained in much more detail in Section``Externalizing Graphics'' of~\cite{tikz}.

\paragraph{Attention:} Do not forget a correct |\pgfrealjobname| statement! If it is missing, externalization simply won't work. If it is wrong, any call to \LaTeX\ will produce empty output files.

It should be noted that this approach of image externalization is not limited to \Tikz\ picture environments. In fact, it collects everything between the begin and end statements into the external file. It is implicitly assumed that the encapsulated stuff is one box, but you can also encapsulate complete paragraphs using something like the \LaTeX\ minipage (or a |\vbox| which is not as powerful but does not affect the remaining document that much).

\begin{key}{/pgf/images/aux in dpth=\mchoice{true,false} (initially false)}
	If this boolean is set to |true|, any |\label| information generated inside of the external image is stored into the already mentioned |.dpth| file. The main document can thus reference label information of externalized parts of the document (although you may need to run |latex| several times). 

	Label support is provided for |\ref|, and probably |\cite|. The |\pageref| command is only partially supported.
\end{key}

\paragraph{Using the Library Without {\normalfont\pgfname} Installed}
Simply uncomment the packages |\usepackage{tikz}| and |\usepackage{pgfplots}| and use
\begin{codeexample}[code only]
\long\def\beginpgfgraphicnamed#1#2\endpgfgraphicnamed{%
	\begingroup
	\setbox1=\hbox{\includegraphics{#1}}%
	\openin1=#1.dpth
	\ifeof1 \box1 
	\else
		\read1 to\pgfincludeexternalgraphicsdp \closein1
		\dimen0=\pgfincludeexternalgraphicsdp\relax
		\hbox{\lower\dimen0 \box1 }%
	\fi
	\endgroup
}
\end{codeexample}
instead. This will include the generated graphics files (and it will respect the |baseline| information stored in |.dpth| files). Consequently, you won't need \pgfname\ or \PGFPlots\ installed. See Section``Externalizing Graphics'' of~\cite{tikz} for details.

\subsection{Importing From Matlab}

\subsubsection{Importing Mesh Data From Matlab To PGFPlots}
While it is easy to write Matlab vectors to files (using |save P.dat data -ASCII|), it is more involved to export mesh data.

The main problem is to communicate the mesh structure to \PGFPlots.

Here is an example how to realize this task: in Matlab, we have mesh data |X|, |Y| and |Z| which are matrices of the same size. For example, suppose we have

\begin{codeexample}[code only]
[X,Y] = meshgrid( linspace(-1,1,5), linspace(4,5,10) );
Z = X + Y;
surf(X,Y,Z)
\end{codeexample}
\noindent as data. Then, we can generate an $N \times 3$ table containing all single elements in column--wise ordering with

\begin{codeexample}[code only]
data = [ X(:) Y(:) Z(:) ]
save P.dat data -ASCII
\end{codeexample}
\noindent where the second command stores the $N \times 3$ table into |P.dat|. Finally, we can use 

|\addplot3[surf,mesh/rows=10,mesh/ordering=colwise,shader=interp] file {P.dat};|

in \PGFPlots\ to read this data. We need to provide either the number of rows ($10$ here) or the number of columns -- and the ordering (which is |colwise| for Matlab matrices).

An alternative which is faster in \PGFPlots\ would be to transpose the matrices in Matlab and tell \PGFPlots\ they are in |rowwise| ordering. So, the last step becomes

\begin{codeexample}[code only]
XX=X'; YY=Y'; ZZ=Z';
data = [ XX(:) YY(:) ZZ(:) ]
save P.dat data -ASCII
\end{codeexample}
\noindent with \PGFPlots\ command

|\addplot3[surf,mesh/cols=10,mesh/ordering=rowwise,shader=interp] file {P.dat};|.

\subsubsection{matlab2pgfplots.m}
This is a Matlab (\textregistered) script which attempts to convert a Matlab figure to \PGFPlots. It requires Matlab version 7.4 (or higher).

\paragraph{Attention:} This script is largely outdated and supports only a very small subset of \PGFPlots. You may want to look at |matlab2tikz|, a conversion script of Nico Schl\"omer available at

\url{http://www.mathworks.com/matlabcentral/fileexchange/22022-matlab2tikz}

\noindent which also uses \PGFPlots\ for the \LaTeX\ conversion.

\medskip
The idea of |matlab2pgfplots.m| is to
\begin{itemize}
	\item use a complete matlab figure as input,
	\item acquire axis labels, axis scaling (log or normal) and legend entries,
	\item acquire all plot coordinates
\end{itemize}
and write an equivalent \texttt{.pgf} file which typesets the plot with \PGFPlots.

The intention is \emph{not} to simulate matlab. It is a first step for a conversion. Type
\begin{lstlisting}
> help matlab2pgfplots
\end{lstlisting}
on your matlab prompt for more information about its features and its limitations.

This script is experimental.

\subsubsection{matlab2pgfplots.sh}
A \texttt{bash}-script which simply starts matlab and runs 
\begin{lstlisting}
	f=hgload( 'somefigure.fig' );
	matlab2pgfplots( 'outputfile.pgf', 'fig', f );
\end{lstlisting}
See matlab2pgfplots.m above.

\subsubsection{Importing Colormaps From Matlab}
Occasionally, you may want to reuse your matlab |colormap| in \PGFPlots. Here is a small Matlab script which converts it to \PGFPlots:
\begin{codeexample}[code only]
C = colormap;  % gets data of the current colormap.
% C = colormap(jet) % gets data of "jet"
eachnth = 1;
I = 1:eachnth:size(C,1); % this is nonsense for eachnth=1 -- but perhaps you don't want each color.
CC = C(I,:);
TeXstring = [ ...
 sprintf('\\pgfplotsset{\n\tcolormap={matlab}{\n') ...
 sprintf('\t\trgb=(%f,%f,%f)\n',CC') ...
 sprintf('\t}\n}\n') ]
\end{codeexample}

\subsection{SVG Output}
It is possible to write every single \Tikz\ picture into a scalable vector graphics (\texttt{.svg}) file. This has nothing to do with \PGFPlots, it is a separate driver of \PGF. Please refer to~\cite[Section ``Producing HTML / SVG Output'']{tikz}.

\subsection{Generate \PGFPlots\ Graphics Within Python}
Mario Orne D\'IAZ ANAD\'ON contributed a small python script |pgfplots.py| which provides a simple interface to generate \PGFPlots\ figures from within python. It can be found in the \PGFPlots\ installation directory, in |pgfplots/scripts/pgfplots/pgfplots.py|; documentation can be found in the file.
}
%
\section{Utilities and Basic Level Commands}
\label{sec:pgfplots:lowlevel}
This section documents commands which provide access to more basic elements of \PGFPlots. Most of them are closely related to the basic level of \pgfname, especially various point commands which are specific to an axis. Some of them are general purpose utilities like loops.

However, most elements in this section are only interesting for advanced users -- and perhaps only for special cases.

\subsection{Utility Commands}

\begin{command}{\foreach \meta{variables} |in| \meta{list} \marg{commands}}
	A powerful loop command provided by \Tikz, see~\cite[Section Utilities]{tikz}.
\begin{codeexample}[]
\foreach \x in {1,2,...,4} {Iterating \x. }%
\end{codeexample}

	A \PGFPlots\ related example could be
\begin{codeexample}[code only]
\foreach \i in {1,2,...,10} {\addplot table {datafile\i}; }%
\end{codeexample}
\end{command}

\begin{command}{\pgfplotsforeachungrouped \meta{variable} |in| \meta{list} \marg{command}}
	A specialised variant of |\foreach| which can do two things: it does not introduce extra groups while executing \meta{command} and it allows to invoke the math parser for (simple!) \meta{$x_0$}|,|\meta{$x_1$}|,...,|\meta{$x_n$} expressions.

\begin{codeexample}[]
\def\allcollected{}
\pgfplotsforeachungrouped \x in {1,2,...,4} {Iterating \x. \edef\allcollected{\allcollected, \x}}%
All collected = \allcollected.
\end{codeexample}

	A more useful example might be to work with tables. The following example is taken from \PGFPlotstable:

\begin{codeexample}[code only]
\pgfplotsforeachungrouped \i in {1,2,...,10} {%
	\pgfplotstablevertcat{\output}{datafile\i} % appends `datafile\i' -> `\output'
}%
% since it was ungrouped, \output is still defined (would not work
% with \foreach)
\end{codeexample}

	\paragraph{Remark: } The special syntax \meta{list}=\meta{$x_0$}|,|\meta{$x_1$}|,...,|\meta{$x_n$}, i.e.\ with two leading elements, followed by dots and a final element, invokes the math parser for the loop. Thus, it allows larger number ranges than any other syntax if |/pgf/fpu| is active.  In all other cases, |\pgfplotsforeachungrouped| invokes |\foreach| and provides the results without \TeX\ groups.
	
	Keep in mind that inside of an axis environment, all loop constructions (including custom loops, |\foreach| and |\pgfplotsforeachungrouped|) need to be handled with care: loop arguments can only be used in places where they are immediately evaluated; but \PGFPlots\ postpones the evaluation of many macros. For example, to loop over something and to generate axis descriptions of the form |\node at (axis cs:\i,0.5)....|, the loop macro |\i| will be evaluated in |\end{axis}| -- but at that time, the loop is over and its value is lost. The correct way to handle such an application is to \emph{expand} the loop variable \emph{explicitly}. For example:
\begin{codeexample}[code only]
\pgfplotsforeachungrouped \i/\j in {
	1 / a,
	2 / b,
	3 / c
}{
	\edef\temp{\noexpand\node at (axis cs: \i,0.5) {\j};}
	% \show\temp % lets TeX show you what \temp contains
	\temp
}
\end{codeexample}
	The example generates three loop iterations: |\i=1|, |\j=a|; then |\i=2|, |j=b|; then |\i=3|, |\j=c|. Inside of the loop body, it expands them and assigns the result to a macro using an ``expanded definition'', |\edef|. The result no longer contains either |\i| or |\j| (since these have been expanded). Then, it invokes the resulting macro. Details about the \TeX\ command |\edef| and expansion control can be found in the document \href{file:TeX-programming-notes.pdf}{TeX-programming-notes.pdf} which comes with \PGFPlots.
\end{command}

\begin{command}{\pgfplotsinvokeforeach\marg{list} \marg{command}}
	A variant of |\pgfplotsforeachungrouped| (and such also of |\foreach|) which replaces any occurrence of |#1| inside of \meta{command} once for every element in \meta{list}. Thus, it actually assumes that \marg{command} is like a |\newcommand| body.

	In other words, \meta{command} is invoked for every element of \meta{list}. The actual element of \meta{list} is available as |#1|.

	As |\pgfplotsforeachungrouped|, this command does \emph{not} introduce extra scopes (i.e.\ it is ungrouped as well).

	The difference to |\foreach \x in |\meta{list}\marg{command} is subtle: the |\x| would \emph{not} be expanded whereas |#1| is. 
\begin{codeexample}[]
\pgfkeys{
  otherstyle a/.code={[a]},
  otherstyle b/.code={[b]},
  otherstyle c/.code={[c]},
  otherstyle d/.code={[d]}}
\pgfplotsinvokeforeach{a,b,c,d}        	
	{\pgfkeys{key #1/.style={otherstyle #1}}}
Invoke them: 
\pgfkeys{key a} \pgfkeys{key b} 
\pgfkeys{key c} \pgfkeys{key d}
\end{codeexample}
The counter example would use a macro (here |\x|) as loop argument:
\begin{codeexample}[]
\pgfkeys{
  otherstyle a/.code={[a]},
  otherstyle b/.code={[b]},
  otherstyle c/.code={[c]},
  otherstyle d/.code={[d]}}
\pgfplotsforeachungrouped \x in {a,b,c,d}        	
	{\pgfkeys{key \x/.style={otherstyle \x}}}
Invoke them: 
\pgfkeys{key a} \pgfkeys{key b}
\pgfkeys{key c} \pgfkeys{key d}
\end{codeexample}

	\paragraph{Restrictions:} you can't nest this command yet (since it does not introduce protection by scopes).
\end{command}

\begin{command}{\pgfmathparse\marg{expression}}
	Invokes the \pgfname\ math parser for \meta{expression} and defines \declareandlabel{\pgfmathresult} to be the result.
\begin{codeexample}[]
\pgfmathparse{1+41}

The result is `\pgfmathresult'.
\end{codeexample}
	\noindent The math engine in \pgfname\ typically uses \TeX's internal arithmetics. That means: it is well suited for numbers in the range $[-16384,16384]$ and has a precision of $5$ digits.

	The number range is typically too small for plotting applications. \PGFPlots\ improves the number range by means of |\pgfkeys{/pgf/fpu}\pgfmathparse{1+41}| to activate the ``floating point unit'' (fpu) and to apply all following operations in floating point. 

	In \PGFPlots, the key |/pgfplots/use fpu| is typically on, which means that any coordinate arithmetics are carried out with the |fpu|. However, all \pgfname\ related drawing operations still use the standard math engine.

	In case you ever need to process numbers of extended precision, you may want to use
\begin{codeexample}[]
\pgfkeys{/pgf/fpu}%
\pgfmathparse{1000*1000}

The result is `\pgfmathprintnumber{\pgfmathresult}'.
\end{codeexample}
	Note that results of the |fpu| are typically not in human-readable format, so |\pgfmathprintnumber| is the preferred way to typeset such numbers.

	Please refer to \cite{tikz} for more details.
\end{command}


\begin{command}{\pgfplotstableread\marg{file}}
	Please refer to the manual of \PGFPlotstable, |pgfplotstable.pdf|, which is part of the \PGFPlots-bundle.
\end{command}
\begin{command}{\pgfplotstabletypeset\marg{\textbackslash macro}}
	Please refer to the manual of \PGFPlotstable, |pgfplotstable.pdf|, which is part of the \PGFPlots-bundle.
\end{command}

\begin{command}{\pgfplotsiffileexists\marg{filename}\marg{true code}\marg{false code}}
	Invokes \meta{true code} if \meta{filename} exists and \meta{false code} if not. Can be used in looping macros, for example to plot every data file until there are no more of them.
\end{command}
\begin{command}{\pgfplotsutilifstringequal\marg{first}\marg{second}\marg{true code}\marg{false code}}
	A simple ``strcmp'' tool which invokes \meta{true code} if \meta{first} $=$\meta{second} and \meta{false code} otherwise. This does not expand macros.
\end{command}


\begin{commandlist}{\pgfkeys,\pgfeov,\pgfkeysvalueof,\pgfkeysgetvalue}
	These commands are part of the \Tikz\ way of specifying options, its sub-package |pgfkeys|. The |\pgfplotsset| command is actually nothing but a wrapper around |\pgfkeys|.

	A short introduction into |\pgfkeys| can be found in~\cite{keyvalintro} whereas the complete reference is, of course, the \Tikz\ manual~\cite{tikz}.

	The key |\pgfkeysvalueof|\marg{key name} expands to the value of a key; |\pgfkeysgetvalue|\marg{key name}\marg{\textbackslash macro} stores the value of \meta{key name} into \meta{\textbackslash macro}. The |\pgfeov| macro is used to delimit arguments for code keys in |\pgfkeys|, please refer to the references mentioned above.
\end{commandlist}

\subsection[Commands Inside Of PGFPlots Axes]{Commands Inside Of {\normalfont\PGFPlots} Axes}
\begin{command}{\autoplotspeclist}
This command should no longer be used, although it will be kept as technical implementation detail. Please use the `|cycle list|' option, Section~\ref{sec:cycle:list}.
\end{command}

\begin{command}{\logten}
Expands to the constant $\log(10)$. Useful for logplots because $\log(10^i) = i\log(10)$. This command is only available inside of a \Tikz-picture.
\end{command}

\begin{command}{\pgfmathprintnumber\marg{number}}
Generates pretty--printed output\footnote{This method was previously \texttt{\textbackslash prettyprintnumber}. Its functionality has been included into \PGF\ and the old command is now deprecated.} for \meta{number}. This method is used for every tick label.

The number is printed using the current number printing options, see the manual of \PGFPlotstable\ which comes with this package for the different number styles, rounding precision and rounding methods.
\end{command}

\begin{command}{\numplots}
	Inside of any of the axis environments, associated style, option or command, |\numplots| expands to the total number of  plots.
\end{command}
\begin{command}{\numplotsofactualtype}
	Like |\numplots|, this macro returns the total number of plots which have the same plot handler. Thus, if you have |sharp plot| active, it returns the number of all |sharp plots|. If you have |ybar| active, it returns the number of |ybar| plots and so on.
\end{command}

\begin{command}{\plotnum}
	Inside of |\addplot| or any associated style, option or command, |\plotnum| expands to the current plot's number, starting with~$0$.
\end{command}

\begin{command}{\plotnumofactualtype}
	Like |\plotnum|, but it returns the number among all plots of the same type. The number of all such plots is available using |\numplotsofactualtype|.	
\end{command}

\begin{command}{\coordindex}
	Inside of an |\addplot| command, this macro expands to the number of the actual coordinate (starting with~$0$).

	It is useful together with |x filter| or |y filter| to (de)select coordinates.
\end{command}

\subsection{Path Operations}

\begin{commandlist}{\path,\draw,\fill,\node,\matrix}
	These commands are \Tikz\ drawing commands all of which are documented in~\cite{tikz}. They are used to draw or fill paths, generate text nodes or aligned text matrices. They are equivalent to 
	\pgfmanualpdflabel{/tikz/draw}{}|\path[draw]|, 
	\pgfmanualpdflabel{/tikz/fill}{}|\path[fill]|, 
	\pgfmanualpdflabel{/tikz/node}{}|\path[node]|, 
	\pgfmanualpdflabel{/tikz/matrix}{}|\path[matrix]|, 
	respectively.
\end{commandlist}
\begin{pathoperation}{--}{\meta{coordinate}}
	A \Tikz\ path operation which connects the current point (the last one before |--|) and \meta{coordinate} with a straight line.
\end{pathoperation}
{\catcode`\|=12
\begin{pathoperation}[noindex]{|-}{\meta{coordinate}}
\pgfmanualpdflabel[\catcode`\|=12 ]{|-}{}%
	A \Tikz\ path operation which connects the current point and \meta{coordinate} with \emph{two} straight lines: first vertical, then horizontal.
\end{pathoperation}

\begin{pathoperation}[noindex]{-|}{\meta{coordinate}}
\pgfmanualpdflabel[\catcode`\|=12 ]{-|}{}%
	A \Tikz\ path operation which connects the current point and \meta{coordinate} with \emph{two} straight lines: first horizontal, then vertical.
\end{pathoperation}
}

\begin{keylist}{/tikz/xshift=\marg{dimension},/tikz/yshift=\marg{dimension}}
	These \Tikz\ keys allow to shift something by \meta{dimension} which is any \TeX\ size (or expression).
\end{keylist}


\begin{command}{\pgfplotsextra\marg{low-level path commands}}
	A command to execute \meta{low-level path commands} in a \PGFPlots\ axis. Since any drawing commands inside of an axis need to be postponed until the axis is complete and the scaling has been initialised, it is not possible to simply draw any paths.
	Instead, it is necessary to draw them as soon as the axis is finished. This is done automatically for every \Tikz\ path -- and it is also done manually if you write |\pgfplotsextra|\marg{commands}.
\begin{codeexample}[]
\begin{tikzpicture}
	\begin{axis}[xmin=0,xmax=3,ymin=0,ymax=5]
	\pgfplotsextra{%
		\pgfpathmoveto{\pgfplotspointaxisxy{1}{2}}%
		\pgfpathlineto{\pgfplotspointaxisxy{2}{4}}%
		\pgfusepath{stroke}%
	}
	\end{axis}
\end{tikzpicture}
\end{codeexample}
	The example above initializes an axis and executes the basic level path commands as soon as the axis is ready. The execution of multiple |\path|, |\addplot| and |\pgfplotsextra| commands is in the same sequence as they occur in the environment\footnote{Except for stacked plots where the sequence may be reverse, see the key \texttt{reverse stack plots}.}.%
\end{command}

\begin{command}{\pgfplotspathaxisoutline}
	Generates a path which resembles the outline of the current axis. This path is used for clip paths and the background paths (if any).
\end{command}

\subsection{Specifying Basic Coordinates}
\label{sec:basic:coordinates}

\begin{commandlist}{%
	\pgfplotspointaxisxy\marg{x coordinate}\marg{y coordinate},%
	\pgfplotspointaxisxyz\marg{x coordinate}\marg{y coordinate}\marg{z coordinate}}
	Point commands like |\pgfpointxy| which take logical, absolute coordinates and return a low--level point. Every transformation from user transformations to logarithms is applied.

	Since the transformations are initialized after the axis is complete, this command needs to be postponed (see |\pgfplotsextra|).

	This command is the basic--level variant of |axis cs:|\meta{x coordinate}|,|\meta{y coordinate}|,|\meta{z coordinate}.
\end{commandlist}

\begin{commandlist}{%
	\pgfplotspointaxisdirectionxy\marg{x coordinate}\marg{y coordinate},%
	\pgfplotspointaxisdirectionxyz\marg{x coordinate}\marg{y coordinate}\marg{z coordinate}}
	Point commands like |\pgfpointxy| which take logical, \emph{relative} coordinates and return a low--level point. Every transformation from user transformations to logarithms is applied. The difference to |\pgfplotspointaxisxy| is that the shift of the linear transformation is skipped here (compare |disabledatascaling|). 

	This command is the basic--level variant of |axis direction cs:|\meta{x coordinate}|,|\meta{y coordinate}|,|\meta{z coordinate}.  Please refer to the documentation of |axis direction cs| for more details.

	Use this command whenever something of \emph{relative} character like directions or lengths need to be supplied. One use-case is to draw ellipses:
\begin{codeexample}[]
\begin{tikzpicture}
\begin{axis}[
	xmin=-3,   xmax=3,
	ymin=-3,   ymax=3,
	extra x ticks={-1,1},
	extra y ticks={-2,2},
	extra tick style={grid=major},
]
	\draw[red] \pgfextra{
	  \pgfpathellipse{\pgfplotspointaxisxy{0}{0}}
		{\pgfplotspointaxisdirectionxy{1}{0}}
		{\pgfplotspointaxisdirectionxy{0}{2}}
	  % see also the documentation of 
	  % 'axis direction cs' which
	  % allows a simpler way to draw this ellipse
	};
	\draw[blue] \pgfextra{
	  \pgfpathellipse{\pgfplotspointaxisxy{0}{0}}
		{\pgfplotspointaxisdirectionxy{1}{1}}
		{\pgfplotspointaxisdirectionxy{0}{2}}
	};
	\addplot [only marks,mark=*] coordinates 
		{ (0,0) };
\end{axis}
\end{tikzpicture}
\end{codeexample}

	Since the transformations are initialized after the axis is complete, this command needs to be provided either inside of a \tikzname\ |\path| command (like |\draw| in the example above) or inside of |\pgfplotsextra|.

\end{commandlist}


\begin{commandlist}{%
	\pgfplotspointrelaxisxy\marg{rel x coordinate}\marg{rel y coordinate},%
	\pgfplotspointrelaxisxyz\marg{rel x coordinate}\marg{rel y coordinate}\marg{rel z coordinate}}
	Point commands which take \emph{relative} coordinates such that $x=0$ is the \emph{lower} $x$ axis limit and $x=1$ the \emph{upper} $x$ axis limit.

	These commands are used for |rel axis cs|.

	Please note that the transformations are only initialised if the axis is complete! This means you need to provide |\pgfplotsextra|.
\end{commandlist}

\begin{commandlist}{%
	\pgfplotspointdescriptionxy\marg{$x$ fraction}\marg{$y$ fraction},%
	\pgfplotsqpointdescriptionxy\marg{$x$ fraction}\marg{$y$ fraction}}%
	Point commands such that |{0}{0}| is the lower left corner of the axis' bounding box and |{1}{1}| the upper right one; everything else is in between. The `|q|' variant is quicker as it doesn't invoke the math parser on its arguments.

	They are used for |axis description cs|, see Section~\ref{pgfplots:sec:axis:description:cs}.
\end{commandlist}

\begin{commandlist}{\pgfplotspointaxisorigin}
	A point coordinate at the origin, $(0,0,0)$. If the origin is not part of the axis limits, the nearest point on the boundary is returned instead.

	This is the same coordinate as returned by the |origin| anchor.
\end{commandlist}

\begin{commandlist}{%
	\pgfplotstransformcoordinatex\marg{x coordinate of an axis},%
	\pgfplotstransformcoordinatey\marg{y coordinate of an axis},%
	\pgfplotstransformcoordinatey\marg{z coordinate of an axis}}
	Defines |\pgfmathresult| to be the low-level \PGF\ coordinate corresponding to the input argument.

	The command applies any |[xyz] coord trafo| keys, data scalings and/or logarithms or whatever \PGFPlots\ does to map input coordinates to internal coordinates.

	The result can be used inside of a |\pgfpointxy| statement (i.e.\ it still needs to be scaled with the respective \PGF\ unit vector).
\begin{codeexample}[]
\begin{tikzpicture}
	\begin{axis}[xmin=0,xmax=2,ymin=0,ymax=5]
	\pgfplotsextra{%
		\pgfplotstransformcoordinatex{1}%
		\let\xcoord=\pgfmathresult
		\pgfplotstransformcoordinatey{1}%
		\let\ycoord=\pgfmathresult
		\pgfpathcircle
			{\pgfqpointxy{\xcoord}{\ycoord}}
			{5pt}%
		\pgfusepath{fill}%
	}%
	\end{axis}
\end{tikzpicture}
\end{codeexample}
	The result of this command is also available as math method |transformcoordinatex| (see the documentation for |axis cs|).

	Please note that the transformations are only initialised if the axis is complete. This means you need to provide |\pgfplotsextra| as is shown in the example above.
\end{commandlist}

\begin{commandlist}{%
	\pgfplotstransformdirectionx\marg{x direction of an axis},%
	\pgfplotstransformdirectiony\marg{y direction of an axis},%
	\pgfplotstransformdirectiony\marg{z direction of an axis}}
	Defines |\pgfmathresult| to be a low-level \PGF\ \emph{direction vector component}.

	A direction vector needs to be \emph{added} to some coordinate in order to get a coordinate, compare the documentation for |\pgfplotspointaxisdirectionxy| and |axis direction cs|.

	The argument \meta{x direction of an axis} is processed in (almost) the same way as for the macro which operates on absolute positions, |\pgfplotstransformcoordinatex|. The only difference is that \emph{directions} need no shifting transformation. 

	The result of this command is also available as math method |transformdirectionx| (see the documentation for |axis direction cs|).

	See |axis direction cs| for details and examples about this command.
\end{commandlist}

\begin{command}{\pgfplotsconvertunittocoordinate\marg{x, y or z}\marg{dimension}}
	Converts a dimension (with unit!) to a corresponding $x$, $y$ or $z$ coordinate. The result will be written to |\pgfmathresult| (without units).

	It is possible to use the result as arguments for the |\pgfpointxyz| commands.

	The effect is to multiply \meta{dimension} with the inverse length of the unit vector for the specified axis. These lengths are precomputed in \PGFPlots\ so the operation is fast.
\begin{codeexample}[code only]
\pgfplotsconvertunittocoordinate{x}{5pt}
% now, the command uses exactly 5pt in x direction:
\pgfqpointxyz{\pgfmathresult}{4}{3}
\end{codeexample}
\end{command}


\begin{commandlist}{%
	\pgfplotspointunitx,%
	\pgfplotspointunity,%
	\pgfplotspointunitz}%
	Low--level point commands which return the canvas $x$, $y$ or $z$ unit vectors.

	The |\pgfplotspointunitx| is the \pgfname\ unit vector in $x$ direction.

	These vectors are essentially the same as |\pgfqpointxyz{1}{0}{0}|, |\pgfqpointxyz{0}{1}{0}|, and |\pgfqpointxyz{0}{0}{1}|, respectively.

	The unit $z$ vector is only defined for three dimensional axes.
\end{commandlist}

\begin{commandlist}{%
	\pgfplotsunitxlength,%
	\pgfplotsunitylength,%
	\pgfplotsunitzlength,%
	\pgfplotsunitxinvlength,%
	\pgfplotsunityinvlength,%
	\pgfplotsunitzinvlength}%
	Macros which expand to the vector length $\lVert x_i \rVert$ of the respective unit vector $x_i$ or the inverse vector length, $1/\lVert x_i \rVert$. These macros can be used inside of |\pgfmathparse|, for example.

	The $x_i$ are the |\pgfplotspointunitx| variants.
\end{commandlist}

\begin{command}{\pgfplotsqpointoutsideofaxis\marg{three-char-string}\marg{coordinate}\marg{normal distance}}
	Provides a point coordinate on one of the available four axes in case of a two dimensional figure or on one of the available twelve axes in case of a three dimensional figure.
	
	The desired axis is uniquely identified by a three character string, provided as first argument to the command. The first of the three characters is `|0|' if the $x$ coordinate of the specified axis passes through the lower axis limit. It is `|1|', if the $x$ coordinate of the specified axis passes through the upper axis limit. Furthermore, it is `|2|' if it passes through the origin. The second character is also either |0|, |1| or |2| and it characterizes the position on the $y$ axis. The third character is for the third dimension, the $z$ axis. It should be left at `|0|' for two dimensional plots. However, \emph{one} of the three characters should be `|v|', meaning the axis \underline varies. For example, |v01| denotes $\{ (x,y_{\text{min}},z_{\text{max}}) \vert x \in \R \}$.
	
	The second argument, \meta{coordinate} is the logical coordinate on that axis. Since two coordinates of the axis are fixed, \meta{coordinate} refers to the \underline varying component of the axis. It must be a number without unit; no math expressions are supported here.

	The third argument \meta{normal distance} is a dimension like |10pt|. It shifts the coordinate away from the designated axis in direction of the outer normal vector. The outer normal vector always points away from the axis. It is computed using
	|\pgfplotspointouternormalvectorofaxis|.

	There are several variants of this command which are documented in the source code. One of them is particularly useful:
\end{command}

\begin{command}{\pgfplotsqpointoutsideofaxisrel\marg{three-char-string}\marg{axis fraction}\marg{normal distance}}
	This point coordinate is a variant of |\pgfplotsqpointoutsideofaxis| which allows to provide an \meta{axis fraction} instead of an absolute coordinate. The fraction is a number between $0$ (lower axis limit) and $1$ (upper axis limit), i.e.\ it is given in percent of the total axis. It is possible to provide negative values or values larger than one.

	The |\pgfplotsqpointoutsideofaxisrel| command is similar in spirit to |rel axis cs|.

	There is one speciality in conjunction with reversed axes: if the axis has been reversed by |x dir=reverse| and, in addition, |allow reversal of rel axis cs| is true, the value $0$ denotes the \emph{upper} limit while $1$ denotes the \emph{lower} limit. The effect is that coordinates won't change just because of axis reversal.
\index{allow reversal of rel axis cs}%
\end{command}

\begin{command}{\pgfplotspointouternormalvectorofaxis\marg{three-char-string}}
	A point command which yields the outer normal vector of the respective axis. The normal vector has length $1$ (computed with |\pgfpointnormalised|). It is the same normal vector used inside of |\pgfplotsqpointoutsideofaxis| and its variants.

	The output of this command will be cached and re-used during the lifetime of an axis. 
\end{command}

\begin{command}{\pgfplotsticklabelaxisspec\marg{x, y or z}}
	Expands to the three-character-identification for the axis containing tick labels for the chosen axis, either \meta{x}, \meta{y} or \meta{z}.
\end{command}

\begin{command}{\pgfplotsvalueoflargesttickdimen\marg{x, y or z}}
	Expands to the largest distance of a tick position to its tick label bounding box in direction of the outer unit normal vector. It does also include the value of the |ticklabel shift| key.

	This value is used for |ticklabel cs|.
\end{command}

\begin{commandlist}{\pgfplotsmathfloatviewdepthxyz\marg{x}\marg{y}\marg{z},
	\pgfplotsmathviewdepthxyz\marg{x}\marg{y}\marg{z}}
	Both macros define |\pgfmathresult| to be the ``depth'' of a three dimensional point $\bar x = (x,y,z)$. The depth is defined to be the scalar product of $\bar x$ with $\vec d$, the view direction of the current axis.

	For |\pgfplotsmathfloatviewdepthxyz|, the arguments are parsed as floating point numbers and the result is encoded in floating point. A fixed point representation can be generated with |\pgfmathfloattofixed{\pgfmathresult}|.

	For |\pgfplotsmathviewdepthxyz|, \TeX\ arithmetics is employed for the inner product and the result is assigned in fixed point. This is slightly faster, but has considerably smaller data range.

	Both commands can only be used \emph{inside} of a three dimensional \PGFPlots\ axis (as soon as the axis is initialised, see |\pgfplotsextra|). 
\end{commandlist}

\begin{texif}{pgfplotsthreedim}
	A \TeX\ |\if| which evaluates the \meta{true code} if the axis is three dimensional and the \meta{else code} if not.
\end{texif}

\subsection{Accessing Axis Limits}
It is also possible to access axis limits during the visualization phase, i.e.\ during |\end{axis}|. Please refer to the reference documentation for |xmin| on page~\pageref{page:access:limits}.

\subsection{Layer Access}
\begin{command}{\pgfplotsonlayer\marg{layer name}}
    A low-level command which will check if the current axis has layer support activated and, if so, calls |\pgfonlayer|\marg{layer name}.

    There must be a |\endpgfplotsonlayer| to delimit the environment.
\end{command}
\begin{command}{\endpgfplotsonlayer}
    The end of |\pgfplotsonlayer|.
\end{command}

\begin{command}{\pgfonlayer\marg{layer name}}
    A low-level command of \PGF\ which will collect everything until the matching |\endpgfonlayer| into layer \meta{layer name}.

    The \meta{layer name} must be active, i.e.\ it must be part of the layer names of |set layers|. 
    
    The only special case is if you call |\pgfdeclarelayer{discard}| somewhere: this special layer has a ``magical name'' which serves as |/dev/null| if it is enabled using |\pgfonlayer{discard}|: it does not need to be active and everything assigned to this layer will be thrown away if it is not part of the layer name configuration.

    There must be a |\endpgfonlayer| to delimit the environment.
\end{command}
\begin{command}{\endpgfonlayer}
    The end of |\pgfonlayer|.
\end{command}


\begin{command}{\pgfsetlayers\marg{layer list}}
	This is a low-level command of \PGF. At the time of this writing, it is the only way to tell \PGF\ which layers it shall use for the current / next picture. It is used implicitly by |set layers|.
\end{command}
%

\printindex

\bibliographystyle{abbrv} %gerapali} %gerabbrv} %gerunsrt.bst} %gerabbrv}% gerplain}
\nocite{pgfplotstable}
\nocite{programmingnotes}
\bibliography{pgfplots}
\end{document}
'
	\expandafter\ifx\csname pgfplotsclickabledisabled\endcsname\relax
		\usepgfplotslibrary{clickable}
	\fi
\fi

%\usepackage{fp}
% ATTENTION:
% this requires pgf version NEWER than 2.00 :
%\usetikzlibrary{fixedpointarithmetic}

\usepgfplotslibrary{dateplot,units,groupplots}

\usepackage[a4paper,left=2.25cm,right=2.25cm,top=2.5cm,bottom=2.5cm,nohead]{geometry}
\usepackage{amsmath,amssymb}
\usepackage{xxcolor}
\usepackage{pifont}
\usepackage[latin1]{inputenc}
\usepackage{amsmath}
\usepackage{eurosym}
\usepackage{nicefrac}

\def\eps{\textsc{eps}}

\input pgfmanual-en-macros.tex


\def\pgfplotsifdocpackageuptodate#1#2{%
	\pgfkeysifdefined{/codeexample/prettyprint/word/.@cmd}{#1}{#2}
}%

\pgfplotsiffileexists{pgfmanual.sty}{%
	\RequirePackage{pgfmanual}
	\pgfplotsifdocpackageuptodate{}{%
		\makeatletter
		\input pgfplotsoldpgfsupp_pgfmanual.code.tex
		\makeatother
	}%
}{%
	\makeatletter
	\input pgfplotsoldpgfsupp_pgfmanual.code.tex
	\makeatother
}%

\makeatletter
\def\pgfplotsmakefilelinkifuseful#1#2{%
	\protect\pgfplotsmakefilelinkifuseful@{#1}{#2}%
}%
\def\pgfplotsmakefilelinkifuseful@#1#2{%
	\edef\temp{#1}%
	\edef\tempb{\jobname}%
	\edef\temp{\meaning\temp}% \meaning normalizes the catcodes.
	\edef\tempb{\meaning\tempb}%
	\ifx\temp\tempb
		% we are processing '#1'. Don't make a link.
		#2%
	\else
		\href{file:#1.pdf}{#2}%
	\fi
}%
\makeatother


\pgfkeys{
	/codeexample/prettyprint/cs arguments/pgfplotscreateplotcyclelist/.initial=2,
	/codeexample/prettyprint/cs/pgfplotscreateplotcyclelist/.code args={#1#2#3}{\pgfmanualpdfref{#1}{#1}\{#2\}\{\pgfmanualprettyprintpgfkeys{#3}\pgfmanualclosebrace},
	/codeexample/prettyprint/cs arguments/tikzset/.initial=1,
	/codeexample/prettyprint/cs/tikzset/.code 2 args={\pgfmanualpdfref{#1}{#1}\{\pgfmanualprettyprintpgfkeys{#2}\pgfmanualclosebrace},
	/codeexample/prettyprint/cs arguments/pgfplotsset/.initial=1,
	/codeexample/prettyprint/cs/pgfplotsset/.code 2 args={\pgfmanualpdfref{#1}{#1}\{\pgfmanualprettyprintpgfkeys{#2}\pgfmanualclosebrace},
	/codeexample/prettyprint/cs arguments/pgfplotstableset/.initial=1,
	/codeexample/prettyprint/cs/pgfplotstableset/.code 2 args={\pgfmanualpdfref{#1}{#1}\{\pgfmanualprettyprintpgfkeys{#2}\pgfmanualclosebrace},
	/codeexample/prettyprint/cs arguments/usepgfplotslibrary/.initial=1,
	/codeexample/prettyprint/cs/usepgfplotslibrary/.code 2 args={\pgfmanualpdfref{#1}{#1}\{\pgfmanualpdfref{#2}{#2}\pgfmanualclosebrace},
	%
	%
	%/codeexample/prettyprint/key value/cycle list/.code 2 args={\pgfmanualprettyprintpgfkeys{#2}},
	/codeexample/prettyprint/key value/xticklabel/.code 2 args={\pgfmanualprettyprintcode{#2}},
	/codeexample/prettyprint/key value/yticklabel/.code 2 args={\pgfmanualprettyprintcode{#2}},
	/codeexample/prettyprint/key value/zticklabel/.code 2 args={\pgfmanualprettyprintcode{#2}},
	/codeexample/prettyprint/key value/includegraphics/.code 2 args={\pgfmanualprettyprintpgfkeys{#2}},
	%
	%
	% whenever an unqualified key is found, the following key prefix
	% list is tried to find a match.
	/pdflinks/search key prefixes in={/pgfplots/table/,/pgfplots/error bars/,/pgfplots/,/pgfplots/plot file/,/tikz/,/pgf/},
	%
	% the link prefix written to the pdf file:
	/pdflinks/internal link prefix=pgfp,
	%
	/pdflinks/warnings=false,
	/pdflinks/codeexample links=true,
	/pdflinks/show labels=false,
}%


% should be used to show something in red which doesn't need to get a
% hyper ref.
%
% Examples are descriptions of key labels.
\def\declaretext#1{\texttt{\declare{#1}}}

% To be used whenever something NEW has been declared.
% In this case, a \pgfmanualpdflabel will be generated using '#1'.
%
% Use '\declaretext' if you only describe something local (for example
% the documentation of key values).
\def\declarelabel#1{%
	\texttt{\declare{#1}}%
	\pgfmanualpdflabel{#1}{}%
}

\def\pgfmanualbar{\char`\|}
\makeatletter

%%%%%%%%%%%%%%%%%%%%%%%%%%%%%%%%%%%%%%%%%%%%%%%%%%%%%%%%%%%%%%%%%%%%%%%%%%%%%%

\usepgfplotslibrary{external}

% use \pgfplotsmanualenableexternalizationofexpensive in the preamble
% to enable externalization of expensive examples.
% This will ONLY externalize expensive examples, i.e. those for which
% \pgfplotsexpensiveexample is written:
%
% \pgfplotsexpensiveexample
% \begin{codeexample}
\def\pgfplotsmanualenableexternalizationofexpensive{%
	\pgfplotsmanual@enable@externalization@for@expensivetrue
	\tikzexternalize[
		prefix=figures/expensiveexampleX,% the 'X' suffix is to avoid confusion in git: previous versions contained the pdfs in git
		figure name={},
		export=false, % needs to be activated for single pictures (i.e. expensive ones)
		mode=list and make,
		verbose=false,
		%xport=true,% FASTER FOR DEBUGGING
	]
	\tikzifexternalizing{%
		\nofiles
		\pgfkeys{/pdflinks/codeexample links=false}%
	}{}%
}%
%\pgfkeys{/pgf/images/include external/.code={\href{file:#1}{\pgfimage{#1}}}} % FIXME : NOT FOR THE FINAL VERSION

\newif\ifpgfplotsmanual@enable@externalization@for@expensive
\newif\ifpgfplots@example@is@expensive

\pgfkeys{
	/codeexample/every codeexample/.append code={%
		\ifpgfplots@example@is@expensive
			\pgfkeys{/tikz/external/export=true}%
			\global\pgfplots@example@is@expensivefalse
		\fi
	}
}

% Write this macro directly in front of \begin{codeexample} (without arguments):
\def\pgfplotsexpensiveexample{%
	\ifpgfplotsmanual@enable@externalization@for@expensive
		\pgfplots@example@is@expensivetrue
	\else
		\message{[NOTE: I am now about to typeset an expensive example. You will need to ENLARGE YOUR TeX MEMORY CAPACITIES if this fails.]}%
	\fi
}%

%%%%%%%%%%%%%%%%%%%%%%%%%%%%%%%%%%%%%%%%%%%%%%%%%%%%%%%%%%%%%%%%%%%%%%%%%%%%%%

\newif\ifpgfplotsmanualhtmlmode

% MACROS FOR HTML OUTPUT:
% HTML will be compiled in a separate folder. See the Makefile
\tikzset{
	external/html export mode/.style={
		system call={%
			pdflatex \tikzexternalcheckshellescape -halt-on-error -interaction=batchmode -jobname "\image" "\\def\\pgfplotsmanualhtlatexmode{1}\texsource" && pdftoppm "\image.pdf" | pnmtopng > "\image.png"%
		},
		/pgf/images/external info,
		/pgf/images/include external/.code={%
			\includegraphics
			   [width=\pgfexternalwidth,height=\pgfexternalheight]
			   {##1.png}%
		},
		prefix=figures/generated/,
		figure name={manual},
		mode=list and make,
		verbose=false,
	},
}

\def\pgfplotsmanual@configure@for@htlatexmode{%
	% disable this; we will externalize everything now:
	\let\pgfplotsmanualenableexternalizationofexpensive=\relax
	%
	\message{^^Jpgfplots manual: initializing external lib for HTML output (png)^^J}%
	\tikzexternalize[
		html export mode,
	]
	%
	\@ifpackageloaded{tex4ht}{%
		\usepackage[html,png,3]{tex4ht}
		\tikzset{external/mode=only graphics}%
		\patches@for@htlatex
	}{%
	}%
	%
	\pgfplotsmanualhtmlmodetrue
	%
	\gdef\pgfsys@imagesuffixlist{.png}
	%
	\tikzifexternalizing{%
		%\nofiles
		\pgfkeys{/pdflinks/codeexample links=false}%
	}{}%
}

% explanation of how to customize tex4ht:
% http://www.cvr.cc/tex4ht-low-level-commands/#more-482
% 
% some reference:
% http://www.cvr.cc/tex4ht-options/
%
% some tex4ht mailing list discussion:
% http://tug.org/pipermail/tex4ht/2010q4/000246.html
\def\patches@for@htlatex{%
 \let\savecolor\color
 \NewConfigure{color}[2]{\def\a at color{##1}\def\b at color{##2}}
 \def\@@tmp##1{\a at color##1\b at color\savecolor{##1}\aftergroup\endspan}
 \let\color\@@tmp
 \def\endspan{\Tg</span>}
 \Configure{color}{\HCode{<span style="color:}}{\HCode{;">}}
}%

\@ifpackageloaded{tex4ht}{
	% always active here:
	\def\pgfplotsmanualhtlatexmode{1}%
}{
}

% the macro \pgfplotsmanualhtlatexmode should be set from command line
% (to any value). If it is known, the manual will be translated to
% HTML.
\pgfutil@ifundefined{pgfplotsmanualhtlatexmode}{%
}{%
	\pgfplotsmanual@configure@for@htlatexmode
}%


%%%%%%%%%%%%%%%%%%%%%%%%%%%%%%%%%%%%%%%%%%%%%%%%%%%%%%%%%%%%%%%%%%%%%%%%%%%%%%



\newenvironment{addplotoperation}[3][]{
  \begin{pgfmanualentry}
  	{%
	\let\ltxdoc@marg=\marg
	\let\ltxdoc@oarg=\oarg
	\let\ltxdoc@parg=\parg
	\let\ltxdoc@meta=\meta
	\def\marg##1{{\normalfont\ltxdoc@marg{##1}}}%
	\def\oarg##1{{\normalfont\ltxdoc@oarg{##1}}}%
	\def\parg##1{{\normalfont\ltxdoc@parg{##1}}}%
	\def\meta##1{{\normalfont\ltxdoc@meta{##1}}}%
    \pgfmanualentryheadline{\textcolor{gray}{{\ttfamily\char`\\addplot\ }}%
      \declare{\texttt{#2}} \texttt{#3;}}%
	  \unskip
	 \nobreak
    \pgfmanualentryheadline{\textcolor{gray}{\texttt{\char`\\addplot}\oarg{options} }%
      \declare{\texttt{#2}} \texttt{#3} \textcolor{gray}{\meta{trailing path commands}}\texttt{;}}%
	  \unskip
	 \nobreak
    \pgfmanualentryheadline{\textcolor{gray}{{\ttfamily\char`\\addplot3}} $\dotsc$}%
    \def\pgfmanualtest{#1}%
    \ifx\pgfmanualtest\@empty%
      \index{#2@\protect\textcolor{gray}{\protect\texttt{plot}}\protect\texttt{ #2}}%
      \index{Plot operations!plot #2@\protect\texttt{plot #2}}%
    \fi%
	\pgfmanualpdflabel{\textbackslash addplot #2}{}%
	\pgfmanualpdflabel{plot #2}{}%
	\pgfmanualpdflabel{#2}{}%
	}%
    \pgfmanualbody
}
{
  \end{pgfmanualentry}
}

\newenvironment{addplot+}{
  \begin{pgfmanualentry}
  	{%
	\let\ltxdoc@marg=\marg
	\let\ltxdoc@oarg=\oarg
	\let\ltxdoc@parg=\parg
	\let\ltxdoc@meta=\meta
	\def\marg##1{{\normalfont\ltxdoc@marg{##1}}}%
	\def\oarg##1{{\normalfont\ltxdoc@oarg{##1}}}%
	\def\parg##1{{\normalfont\ltxdoc@parg{##1}}}%
	\def\meta##1{{\normalfont\ltxdoc@meta{##1}}}%
    \pgfmanualentryheadline{{\ttfamily\declare{\char`\\addplot+}\oarg{options} \textcolor{gray}{\dots};}}%
    \index{addplot+@\protect\texttt{\protect\textbackslash addplot+}}%
	\pgfmanualpdflabel{\textbackslash addplot+}{}%
	}%
    \pgfmanualbody
}
{
  \end{pgfmanualentry}
}
\newenvironment{addplot3generic}{
  \begin{pgfmanualentry}
  	{%
	\let\ltxdoc@marg=\marg
	\let\ltxdoc@oarg=\oarg
	\let\ltxdoc@parg=\parg
	\let\ltxdoc@meta=\meta
	\def\marg##1{{\normalfont\ltxdoc@marg{##1}}}%
	\def\oarg##1{{\normalfont\ltxdoc@oarg{##1}}}%
	\def\parg##1{{\normalfont\ltxdoc@parg{##1}}}%
	\def\meta##1{{\normalfont\ltxdoc@meta{##1}}}%
    \pgfmanualentryheadline{{\ttfamily\declare{\char`\\addplot3}\oarg{options} \meta{input data} \meta{trailing path commands};}}%
    \index{addplot3@\protect\texttt{\protect\textbackslash addplot3}}%
	\pgfmanualpdflabel{\textbackslash addplot3}{}%
	}%
    \pgfmanualbody
}
{
  \end{pgfmanualentry}
}
\newenvironment{addplot3operation}[3][]{
  \begin{pgfmanualentry}
  	{%
	\let\ltxdoc@marg=\marg
	\let\ltxdoc@oarg=\oarg
	\let\ltxdoc@parg=\parg
	\let\ltxdoc@meta=\meta
	\def\marg##1{{\normalfont\ltxdoc@marg{##1}}}%
	\def\oarg##1{{\normalfont\ltxdoc@oarg{##1}}}%
	\def\parg##1{{\normalfont\ltxdoc@parg{##1}}}%
	\def\meta##1{{\normalfont\ltxdoc@meta{##1}}}%
    \pgfmanualentryheadline{\textcolor{gray}{{\ttfamily\char`\\addplot3\ }}%
      \declare{\texttt{#2}} \texttt{#3;}}%
	  \unskip
	 \nobreak
    \pgfmanualentryheadline{\textcolor{gray}{\texttt{\char`\\addplot3}\oarg{options} }%
      \declare{\texttt{#2}} \texttt{#3} \textcolor{gray}{\meta{trailing path commands}}\texttt{;}}%
    \def\pgfmanualtest{#1}%
    \ifx\pgfmanualtest\@empty%
      \index{#2@\protect\texttt{#2}}%
      \index{Plot operations!addplot3 #2@\protect\texttt{#2}}%
    \fi%
	\pgfmanualpdflabel{\textbackslash addplot3 #2}{}%
	\pgfmanualpdflabel{plot3 #2}{}%
	}%
    \pgfmanualbody
}
{
  \end{pgfmanualentry}
}

\newenvironment{codekey}[1]{%
  \begin{pgfmanualentry}
 	\pgfmanualentryheadline{{\ttfamily\declarekey{#1}\textcolor{gray}{/\pgfmanualpdfref{/handlers/.code}{.code}}=\marg{...}}\hfill}%
	\def\mykey{#1}%
	\def\mypath{}%
	\def\myname{}%
	\firsttimetrue%
	\decompose#1/\nil%
    \pgfmanualbody
}
{
  \end{pgfmanualentry}
}
\newenvironment{codeargskey}[2]{%
  \begin{pgfmanualentry}
  	{\toks0={#2}%
  	\xdef\argpattern{\the\toks0 }%
	}%
  	\pgfmanual@command@to@string\argpattern\argpattern
 	\pgfmanualentryheadline{{\ttfamily\declarekey{#1}\textcolor{gray}{/\pgfmanualpdfref{/handlers/.code}{.code args}}=\texttt{\{\argpattern\}}\marg{...}}\hfill}%
	\def\mykey{#1}%
	\def\mypath{}%
	\def\myname{}%
	\firsttimetrue%
	\decompose#1/\nil%
    \pgfmanualbody
}
{
  \end{pgfmanualentry}
}
\def\pgfmanual@command@to@string#1#2{%
	\expandafter\pgfmanual@command@to@string@@\meaning#1\pgfmanual@EOI{#2}%
}%
\xdef\pgfmanual@glob@TMPa{\meaning\pgfutil@empty}%
\expandafter\def\expandafter\pgfmanual@command@to@string@@\pgfmanual@glob@TMPa#1\pgfmanual@EOI#2{%
	\def#2{#1}%
}%

\newenvironment{pgfplotscodekey}[1]{%
	\begin{codekey}{/pgfplots/#1}%
}
{
  \end{codekey}
}
\newenvironment{pgfplotscodetwokey}[1]{%
  \begin{pgfmanualentry}
 	\pgfmanualentryheadline{{\ttfamily\declarekey{/pgfplots/#1}\textcolor{gray}{/\pgfmanualpdfref{/handlers/.code 2 args}{.code 2 args}}=\marg{...}}\hfill}%
	\def\mykey{/pgfplots/#1}%
	\def\mypath{}%
	\def\myname{}%
	\firsttimetrue%
	\decompose/pgfplots/#1/\nil%
    \pgfmanualbody
}
{
  \end{pgfmanualentry}
}

\newenvironment{pgfplotsxycodekeylist}[1]{%
	\begingroup
	\let\oldpgfmanualentryheadline=\pgfmanualentryheadline
	\def\pgfmanualentryheadline##1{%
		\pgfmanualentryheadline@##1\pgfplots@EOI
	}%
	\def\pgfmanualentryheadline@##1\hfill##2\pgfplots@EOI{%
		\oldpgfmanualentryheadline{{\ttfamily\declarekey{##1}\textcolor{gray}{/\pgfmanualpdfref{/handlers/.code}{.code}}=\marg{...}}\hfill}%
	}
	\begin{pgfplotsxykeylist}{#1}%
}
{
	\end{pgfplotsxykeylist}
	\endgroup
}

\newenvironment{pgfplotskey}[1]{%
  \begin{key}{/pgfplots/#1}%
}
{
  \end{key}
}

\def\choicesep{$\vert$}%
\def\choicearg#1{\texttt{#1}}

\newif\iffirstchoice
\newcommand\mchoice[1]{%
	\begingroup
	\let\margold=\marg
	\def\marg##1{{\normalfont\margold{##1}}}%
	\firstchoicetrue
	\foreach \mchoice@ in {#1} {%
		\iffirstchoice
			\global\firstchoicefalse
		\else
			\choicesep
		\fi
		\choicearg{\mchoice@}%
	}%
	\endgroup
}%




% \begin{xykey}{/path/\x label=value}
% \end{xykey}
%
% has same features with 'default', 'initially' etc as key environment
\newenvironment{xykey}[2][]{%
	\begin{pgfmanualentry}
    \def\extrakeytext{}
	\insertpathifneeded{#2}{#1}%
	\expandafter\pgfutil@in@\expandafter=\expandafter{\mykey}%
	\ifpgfutil@in@%
		\expandafter\xykey@eq\mykey\@nil
	\else
		\expandafter\xykey@noeq\mykey\@nil
	\fi
	\pgfmanualbody
}{%
	\end{pgfmanualentry}
}%

% \begin{xystylekey}{/path/\x label=value}
% \end{xystylekey}
%
% has same features with 'default', 'initially' etc as key environment
\newenvironment{xystylekey}[2][]{%
	\begin{pgfmanualentry}
    \def\extrakeytext{style, }
	\insertpathifneeded{#2}{#1}%
	\expandafter\pgfutil@in@\expandafter=\expandafter{\mykey}%
	\ifpgfutil@in@%
		\expandafter\xykey@eq\mykey\@nil
	\else
		\expandafter\xykey@noeq\mykey\@nil
	\fi
	\pgfmanualbody
}{%
	\end{pgfmanualentry}
}%

% \insertpathifneeded{a key}{/pgfplots} -> assign mykey={/pgfplots/a key}
% \insertpathifneeded{/tikz/a key}{/pgfplots} -> assign mykey={/tikz/a key}
%
% #1: the key
% #2: a default path (or empty)
\def\insertpathifneeded#1#2{%
	\def\insertpathifneeded@@{#2}%
	\ifx\insertpathifneeded@@\empty
		\def\mykey{#1}%
	\else
		\insertpathifneeded@#1\@nil
		\ifpgfutil@in@
			\def\mykey{#1}%
		\else
			\def\mykey{#2/#1}%
		\fi
	\fi
}%
\def\insertpathifneeded@#1#2\@nil{%
	\def\insertpathifneeded@@{#1}%
	\def\insertpathifneeded@@@{/}%
	\ifx\insertpathifneeded@@\insertpathifneeded@@@
		\pgfutil@in@true
	\else
		\pgfutil@in@false
	\fi
}%

% \begin{keylist}[default path]
% 	{/path/option 1=value,/path/option 2=value2}
% \end{keylist}
\newenvironment{keylist}[2][]{%
	\begin{pgfmanualentry}
    \def\extrakeytext{}%
	\foreach \xx in {#2} {%
		\expandafter\insertpathifneeded\expandafter{\xx}{#1}%
		\expandafter\extractkey\mykey\@nil%
	}%
	\pgfmanualbody
}{%
  \end{pgfmanualentry}
}%

\newenvironment{pgfplotskeylist}[1]{%
	\begin{keylist}[/pgfplots]{#1}%
}{%
	\end{keylist}%
}

\newenvironment{anchorlist}[1]{
  \begin{pgfmanualentry}
  	\foreach \xx in {#1} {%
		\pgfmanualentryheadline{Anchor {\ttfamily\declare{\xx}}}%
		\index{\xx @\protect\texttt{\xx} anchor}%
		\index{Anchors!\xx @\protect\texttt{\xx}}
		\expandafter\pgfmanualpdflabel\expandafter{\xx}{}
	}%
    \pgfmanualbody
}
{
  \end{pgfmanualentry}
}

\newenvironment{coordinatesystemlist}[1]{
  \begin{pgfmanualentry}
  	\foreach \xx in {#1} {%
		\pgfmanualentryheadline{Coordinate system {\ttfamily\declare{\xx}}}%
		\index{\xx @\protect\texttt{\xx} coordinate system}%
		\index{Coordinate systems!\xx @\protect\texttt{\xx}}
		\expandafter\pgfmanualpdflabel\expandafter{\xx}{}
	}%
    \pgfmanualbody
}
{
  \end{pgfmanualentry}
}
\renewenvironment{coordinatesystem}[1]{
  \begin{pgfmanualentry}
    \pgfmanualentryheadline{Coordinate system {\ttfamily\declare{#1}}}%
    \index{#1@\protect\texttt{#1} coordinate system}%
    \index{Coordinate systems!#1@\protect\texttt{#1}}
	\pgfmanualpdflabel{#1}{}
    \pgfmanualbody
}
{
  \end{pgfmanualentry}
}

% \begin{xykeylist}[default path]
% 	{/path/option \x1=value,/path/option \x2=value2,/path/option \x3=value}
% \end{xykeylist}
\newenvironment{xykeylist}[2][]{%
	\begin{pgfmanualentry}
    \def\extrakeytext{}
	\foreach \xx in {#2} {%
		\expandafter\insertpathifneeded\expandafter{\xx}{#1}%
		\expandafter\pgfutil@in@\expandafter=\expandafter{\mykey}%
		\ifpgfutil@in@%
			\expandafter\xykey@eq\mykey\@nil
		\else
			\expandafter\xykey@noeq\mykey\@nil
		\fi
	}%
	\pgfmanualbody
}{%
  \end{pgfmanualentry}
}%

\makeatother % FIXME this is almost surely a bug in pgfmanual-en-macros
% \begin{commandlist}
% 	{\command1{arg1},\command2{\arg2}}
% \end{commandlist}
\newenvironment{commandlist}[1]{%
	\begin{pgfmanualentry}
	\foreach \xx in {#1} {%
		\expandafter\extractcommand\xx\@@%
	}%
	\pgfmanualbody
}{%
  \end{pgfmanualentry}
}%

\newenvironment{texif}[1]{%
	\begin{pgfmanualentry}
	\pgfmanualentryheadline{\declare{\texttt{\textbackslash if#1}}\meta{true code}\texttt{\textbackslash else}\meta{else code}\texttt{\textbackslash fi}}%
	\index{if#1}%
	\pgfmanualpdflabel{\\if#1}{}%
	\pgfmanualbody
}{%
  \end{pgfmanualentry}
}%
\makeatletter

\newif\ifxykeyfound

\def\pgfmanual@xykey@install@replacements{%
	\def\ { }%
	\def\space{ }%
}%

\def\xykey@eq#1=#2\@nil{%
	\begingroup
	\pgfmanual@xykey@install@replacements
	\def\x{x}%
	\xdef\mykey{#1}%
	\def\xykey@@{#1}%
	\ifx\xykey@@\mykey
		\xykeyfoundfalse
	\else
		\xykeyfoundtrue
	\fi
    \expandafter\extractkey\mykey=#2\@nil%
	\ifxykeyfound
		\def\x{y}%
		\xdef\mykey{#1}%
		\expandafter\extractkey\mykey=#2\@nil%
		\def\x{z}%
		\xdef\mykey{#1}%
		\expandafter\extractkey\mykey=#2\@nil%
	\fi
	\endgroup
}
\def\xykey@noeq#1\@nil{%
	\begingroup
	\pgfmanual@xykey@install@replacements
	\def\x{x}%
	\xdef\mykey{#1}%
	\def\xykey@@{#1}%
	\ifx\xykey@@\mykey
		\xykeyfoundfalse
	\else
		\xykeyfoundtrue
	\fi
    \expandafter\extractkey\mykey\@nil%
	\ifxykeyfound
		\def\x{y}%
		\xdef\mykey{#1}%
		\expandafter\extractkey\mykey\@nil%
		\def\x{z}%
		\xdef\mykey{#1}%
		\expandafter\extractkey\mykey\@nil%
	\fi
	\endgroup
}

% \begin{pgfplotsxykey}{\x label=value}
% \end{pgfplotsxykey}
%
% It introduces the path /pgfplots/ automatically.
%
% has same features with 'default', 'initially' etc as key environment
\newenvironment{pgfplotsxykey}[1]{%
	\begin{xykey}[/pgfplots]{#1}%
}{%
	\end{xykey}%
}


\newenvironment{pgfplotsxykeylist}[1]{%
	\begin{xykeylist}[/pgfplots]{#1}%
}{%
	\end{xykeylist}%
}


% the first, optional argument is the default key path to insert.
\newenvironment{plottype}[2][/tikz]{%
	\begin{keylist}[#1]{#2}%
	\end{keylist}
  \begin{pgfmanualentry}
    \pgfmanualentryheadline{\textcolor{gray}{{\ttfamily\char`\\addplot+[\declare{#2}]}}}%
    \pgfmanualbody
}
{
  \end{pgfmanualentry}
}

\def\index@prologue{\section*{Index}\addcontentsline{toc}{section}{Index}
}

\newenvironment{pgfplotstablecolumnkey}{%
  \begin{pgfmanualentry}
 	\pgfmanualentryheadline{{\ttfamily\textcolor{gray}{/pgfplots/table/}\declare{columns/\meta{column name}}\textcolor{gray}{/.style}=\marg{key-value-list}}\hfill}%
	\pgfplotsmanualkeyindex{/pgfplots/table/columns}%
    \pgfmanualbody
}
{
  \end{pgfmanualentry}
}
\newenvironment{pgfplotstabledisplaycolumnkey}{%
  \begin{pgfmanualentry}
 	\pgfmanualentryheadline{{\ttfamily\textcolor{gray}{/pgfplots/table/}\declare{display columns/\meta{index}}\textcolor{gray}{/.style}=\marg{key-value-list}}\hfill}%
	\pgfplotsmanualkeyindex{/pgfplots/table/display columns}%
    \pgfmanualbody
}
{
  \end{pgfmanualentry}
}
\newenvironment{pgfplotstablealiaskey}{%
  \begin{pgfmanualentry}
 	\pgfmanualentryheadline{{\ttfamily\textcolor{gray}{/pgfplots/table/}\declare{alias/\meta{col name}}\textcolor{gray}{/.initial}=\marg{real col name}}\hfill}%
	\pgfplotsmanualkeyindex{/pgfplots/table/alias}%
    \pgfmanualbody
}
{
  \end{pgfmanualentry}
}


\def\pgfplotsmanualkeyindex#1{%
	\def\mypath{#1}%
	\def\myname{}%
	\firsttimetrue%
	\decompose#1/\nil%
}
\newenvironment{pgfplotstablecreateonusekey}{%
  \begin{pgfmanualentry}
 	\pgfmanualentryheadline{{\ttfamily\textcolor{gray}{/pgfplots/table/}\declare{create on use/\meta{col name}}\textcolor{gray}{/.style}=\marg{create options}}\hfill}%
	\def\mykey{/pgfplots/table/create on use}%
    \pgfmanualbody
	\pgfplotsmanualkeyindex{/pgfplots/table/create on use}%
}
{
  \end{pgfmanualentry}
}

\def\pgfplotsassertcmdkeyexists#1{%
	\pgfkeysifdefined{/pgfplots/#1/.@cmd}\relax{%
		\pgfplots@error{DOCUMENTATION ERROR: command key /pgfplots/#1 does not exist!}%
	}%
}%

{
\catcode`\ =12%
\gdef\makespaceexpandable{\def\ { }}}%

\def\pgfplotsassertXYcmdkeyexists#1{%
	{\makespaceexpandable\def\x{x}\edef\pgfplotsassertXYcmdkeyexists@tmp{#1}%
	\pgfkeysifdefined{/pgfplots/\pgfplotsassertXYcmdkeyexists@tmp/.@cmd}\relax{%
		\pgfplots@error{DOCUMENTATION ERROR: command key /pgfplots/#1 does not exist!}%
	}}%
	{\makespaceexpandable\def\x{y}\edef\pgfplotsassertXYcmdkeyexists@tmp{#1}%
	\pgfkeysifdefined{/pgfplots/\pgfplotsassertXYcmdkeyexists@tmp/.@cmd}\relax{%
		\pgfplots@error{DOCUMENTATION ERROR: command key /pgfplots/#1 does not exist!}%
	}}%
}%

\def\pgfplotsshortstylekey #1=#2\pgfeov{%
	\pgfplotsassertcmdkeyexists{#1}%
	\pgfplotsassertcmdkeyexists{#2}%
	\begin{pgfplotskey}{#1=\marg{key-value-list}}
		An abbreviation for \texttt{\pgfmanualpdfref{#2}{#2}/\pgfmanualpdfref{/handlers/.append style}{.append style}=}\marg{key-value-list}.

		It appends options to the already existing style \texttt{\pgfmanualpdfref{#2}{#2}}.
	\end{pgfplotskey}
}
\def\pgfplotsshortxystylekey #1=#2\pgfeov{%
	\pgfplotsassertXYcmdkeyexists{#1}%
	\pgfplotsassertXYcmdkeyexists{#2}%
	\begin{pgfplotsxykey}{#1=\marg{key-value-list}}
		An abbreviation for {\def\x{x}\texttt{\pgfmanualpdfref{#2}{#2}/\pgfmanualpdfref{/handlers/.append style}{.append style}=}}\marg{key-value-list} 
		(or the respective styles for $y$, 
			{\def\x{y}\texttt{\pgfmanualpdfref{#2}{#2}/\pgfmanualpdfref{/handlers/.append style}{.append style}=}}\marg{key-{}value-{}list},
		and the $z$--axis, 
			{\def\x{z}\texttt{\pgfmanualpdfref{#2}{#2}/\pgfmanualpdfref{/handlers/.append style}{.append style}=}}\marg{key-{}value-{}list}).

		It appends options to the already existing style {\def\x{x}\texttt{\pgfmanualpdfref{#2}{#2}}}.
	\end{pgfplotsxykey}
}
\def\pgfplotsshortstylekeys #1,#2=#3\pgfeov{%
	\pgfplotsassertcmdkeyexists{#1}%
	\pgfplotsassertcmdkeyexists{#2}%
	\pgfplotsassertcmdkeyexists{#3}%
	\begin{pgfplotskeylist}{%
		#1=\marg{key-value-list},
		#2=\marg{key-value-list}}
		Different abbreviations for \texttt{\pgfmanualpdfref{#3}{#3}/\pgfmanualpdfref{/handlers/.append style}{.append style}=}\marg{key-value-list}.
	\end{pgfplotskeylist}
}
\def\pgfplotsshortxystylekeys #1,#2=#3\pgfeov{%
	\pgfplotsassertXYcmdkeyexists{#1}%
	\pgfplotsassertXYcmdkeyexists{#2}%
	\pgfplotsassertXYcmdkeyexists{#3}%
	\begin{pgfplotsxykeylist}{%
		#1=\marg{key-value-list},
		#2=\marg{key-value-list}}
		Different abbreviations for {\def\x{x}\texttt{\pgfmanualpdfref{#3}{#3}/\pgfmanualpdfref{/handlers/.append style}{.append style}=}}\marg{key-value-list}
		(or the respective styles for $y$, 
			{\def\x{y}\texttt{\pgfmanualpdfref{#3}{#3}/\pgfmanualpdfref{/handlers/.append style}{.append style}=}\marg{key-{}value-{}list}}, and $z$,
			{\def\x{z}\texttt{\pgfmanualpdfref{#3}{#3}/\pgfmanualpdfref{/handlers/.append style}{.append style}=}\marg{key-{}value-{}list}}%
			).
	\end{pgfplotsxykeylist}
}


%
% For using the correct form of including libraries in the manual.
% 
\newenvironment{pgfplotslibrary}[1]{%
  \begin{pgfmanualentry}
    \pgfmanualentryheadline{{\ttfamily\char`\\usepgfplotslibrary\char`\{\declare{#1}\char`\}\space\space \char`\%\space\space  \LaTeX\space and plain \TeX}}%
    \index{#1@\protect\texttt{#1} library}%
    \index{Libraries!#1@\protect\texttt{#1}}%
    \pgfmanualentryheadline{{\ttfamily\char`\\usepgfplotslibrary[\declare{#1}]\space \char`\%\space\space Con\TeX t}}%
    \pgfmanualentryheadline{{\ttfamily\char`\\usetikzlibrary\char`\{\declare{pgfplots.#1}\char`\}\space\space \char`\%\space\space \LaTeX\space and plain \TeX}}%
    \pgfmanualentryheadline{{\ttfamily\char`\\usetikzlibrary[\declare{pgfplots.#1}]\space \char`\%\space\space Con\TeX t}}%
	\pgfmanualpdflabel{#1}{}%
    \pgfmanualbody
}
{
  \end{pgfmanualentry}
}


%
% Creates and shows a colormap with specification '#1'.
\def\pgfplotsshowcolormapexample#1{%
	\pgfplotscreatecolormap{tempcolormap}{#1}%
	\pgfplotsshowcolormap{tempcolormap}%
}

% Shows the colormap named '#1'.
\def\pgfplotsshowcolormap#1{%
	\pgfplotscolormapifdefined{#1}{\relax}{%
		\pgfplotsset{colormap/#1}%
	}%
	\pgfplotscolormaptoshadingspec{#1}{8cm}\result
	\def\tempb{\pgfdeclarehorizontalshading{tempshading}{1cm}}%
	\expandafter\tempb\expandafter{\result}%
	\pgfuseshading{tempshading}%
}

\makeatother

\def\decompose/#1/#2\nil{%
  \def\test{#2}%
  \ifx\test\empty%
    % aha.
    \index{#1@\protect\texttt{#1} key}%
	\ifx\mypath\empty
	\else
		\index{\mypath#1@\protect\texttt{#1}}%
	\fi
    \def\myname{#1}%
	%\pgfmanualpdflabel{#1}{}% No, its better to use fully qualified keys and search if necessary!
  \else%
    \iffirsttime
		\begingroup	
			% also make a pdf link anchor with full key path.
			\def\hyperlabelwithoutslash##1/\nil{%
				\pgfmanualpdflabel{##1}{}%
			}%
			\hyperlabelwithoutslash/#1/#2\nil
		\endgroup
%    CF : disabled for /pgfplots/ prefix.
%		\def\mypath{#1@\protect\texttt{/#1/}!}%
%		\firsttimefalse
		\def\pgfplotslocTMPa{pgfplots}%
		\edef\pgfplotslocTMPb{#1}%
		\ifx\pgfplotslocTMPb\pgfplotslocTMPa
			\def\mypath{}%
		\else
			\def\mypath{#1@\protect\texttt{/#1/}!}%
		\fi
		\firsttimefalse
    \else
      \expandafter\def\expandafter\mypath\expandafter{\mypath#1@\protect\texttt{#1/}!}%
    \fi
    \def\firsttime{}
    \decompose/#2\nil%
  \fi%
}
\def\extracthandler#1#2\@nil{%
  \pgfmanualentryheadline{Key handler \meta{key}{\ttfamily/\declare{#1}}#2}%
  \index{\gobble#1@\protect\texttt{#1} handler}%
  \index{Key handlers!#1@\protect\texttt{#1}}
  \pgfmanualpdflabel{/handlers/#1}%
}
\def\extractcommand#1#2\@@{%
  \pgfmanualentryheadline{\declare{\texttt{\string#1}}#2}%
  \removeats{#1}%
  \index{\strippedat @\protect\myprintocmmand{\strippedat}}%
  \pgfmanualpdflabel{\textbackslash\strippedat}{}%
}
\def\extractenvironement#1#2\@@{%
  \pgfmanualentryheadline{{\ttfamily\char`\\begin\char`\{\declare{#1}\char`\}}#2}%
  \pgfmanualentryheadline{{\ttfamily\ \ }\meta{environment contents}}%
  \pgfmanualentryheadline{{\ttfamily\char`\\end\char`\{\declare{#1}\char`\}}}%
  \index{#1@\protect\texttt{#1} environment}%
  \index{Environments!#1@\protect\texttt{#1}}%
  \pgfmanualpdflabel{#1}{}%
}
\renewenvironment{predefinednode}[1]{
  \begin{pgfmanualentry}
    \pgfmanualentryheadline{Predefined node {\ttfamily\declare{#1}}}%
    \index{#1@\protect\texttt{#1} node}%
    \index{Predefined node!#1@\protect\texttt{#1}}
	\pgfmanualpdflabel{#1}{}%
    \pgfmanualbody
}
{
  \end{pgfmanualentry}
}



\makeatletter
\@ifpackageloaded{tex4ht}{
}{%
	\IfFileExists{ocg.sty}{%
		\usepackage{ocg}%
	}{%
		\usepackage{pgfplots_ocg_copy}%
	}
}%
\makeatother

\usepackage{nicefrac}

\graphicspath{{figures/}}

\def\preambleconfig{width=7cm,compat=1.7}


\expandafter\pgfplotsset\expandafter{\preambleconfig}


\makeatletter
% And now, invoke
% 	/codeexample/typeset listing/.add={% Preamble:\pgfplotsset{\preambleconfig}}{}}
% since listings are VERBATIM, I need to do some low-level things
% here to get the correct \catcodes:
\pgfkeys{/codeexample/typeset listing/.add code={%
		\ifcode@execute
			\pgfutil@in@{axis}{#1}%
			\ifpgfutil@in@
				{\tiny
					\% Preamble: \pgfmanualpdfref{\textbackslash pgfplotsset}{\pgfmanual@pretty@backslash pgfplotsset}%
						\pgfmanual@pretty@lbrace \expandafter\pgfmanualprettyprintpgfkeys\expandafter{\preambleconfig}\pgfmanual@pretty@rbrace
				}%
			\fi
		\fi
	}{},%
	%/codeexample/typeset listing/.show code,
}%
\makeatother

\pgfplotsset{
	%every axis/.append style={width=7cm},
	filter discard warning=false,
}

\pgfqkeys{/codeexample}{%
	every codeexample/.append style={
		width=8cm,
		/pgfplots/legend style={fill=graphicbackground},
		/pgfplots/contour/every contour label/.append style={
			every node/.append style={fill=graphicbackground}
		},
	},
	tabsize=4,
}

\usetikzlibrary{backgrounds,patterns}
% Global styles:
\tikzset{
  shape example/.style={
    color=black!30,
    draw,
    fill=yellow!30,
    line width=.5cm,
    inner xsep=2.5cm,
    inner ysep=0.5cm}
}

\newcommand{\FIXME}[1]{\textcolor{red}{(FIXME: #1)}}

% fuer endvironment 'sidewaysfigure' bspw
% \usepackage{rotating}

\newcommand\Tikz{Ti\textit kZ}
\newcommand\PGF{\textsc{pgf}}
\newcommand\PGFPlots{\pgfplotsmakefilelinkifuseful{pgfplots}{\textsc{pgfplots}}}
\newcommand\PGFPlotstable{\pgfplotsmakefilelinkifuseful{pgfplotstable}{\textsc{PgfplotsTable}}}

\makeindex

% Fix overful hboxes automatically:
\tolerance=2000
\emergencystretch=10pt

\tikzset{prefix=gnuplot/pgfplots_} % prefix for 'plot function'

\author{%
	Dr.\ Christian Feuers\"anger\\
	{\footnotesize\texttt{cfeuersaenger@users.sourceforge.net}}}%



\usepackage{array}
\usepackage{colortbl}
\usepackage{booktabs}
\usepackage{eurosym}

\long\def\codeexamplenl{\noexpand\par}%
\pgfqkeys{/codeexample}{%
	every codeexample/.style={
		width=4cm,
		/pgfplots/every axis/.append style={legend style={fill=graphicbackground}}
	},
	narrow/.style={width=7cm},
	tabsize=4,
	%pre={\begin{minipage}{\linewidth}\begingroup},
	%post={\endgroup\end{minipage}},
	vbox,
	newline=\codeexamplenl,
}

\title{Notes On Programming in \TeX}%\\and Library Functions from \PGF\ and \PGFPlots}

\begin{document}
\maketitle
\begin{abstract}%
	This document contains notes which are intended for those who are interested in \TeX\ programming. It is valuable for beginners as a first start with a lot of examples, and it is also valuable for experienced \TeX nicians who are interested in details about \TeX\ programming. However, it is neither a complete reference, nor a complete manual of \TeX.
\end{abstract}
\tableofcontents
\section{Introduction}
This document is intended to provide a direct start with \TeX\ programming (not necessarily \TeX\ typesetting). The addressed audience consists of people interested in package or library writing.

At the time of this writing, this document is far from complete. Nevertheless, it might be a good starting point for interested readers. Consult the literature given below for more details.

\section{Programming in \TeX}
\subsection{Variables in Registers}
\TeX\ provides several different variables and associated registers which can be manipulated freely.

\label{sec:variables}
\begin{command}{\count\meta{num}}
	There are 256 Integer registers which provide 32 Bit Integer arithmetics. The registers can be used for example with |\count0=42 | or |\count7=\macro | where |\macro| expands to a number.

	The value of a register can be typeset using |\the|\meta{register}.
\begin{codeexample}[]
\count0=42
The value is now `\the\count0'. 
\def\macro{-123456}
\count0=\macro 
The value is now `\the\count0'.
\end{codeexample}
	
	The `|=|' sign is optional and can be omitted. One thing is common among the registers: an assignment of the form |\count0=|\meta{$\cdots$} expands everything which follows until the expansion doesn't need more numbers -- even more than one following macro.
\begin{codeexample}[]
\def\firstmacro{123}
\def\secondmacro{456}
\def\thirdmacro{789}
\count0=\firstmacro\secondmacro\thirdmacro
The value is now `\the\count0'.
\end{codeexample}
 The precise rules can be found in~\cite{texbook}, but it should be kept in mind that care needs to be taken here. More than once, my code failed to produce the expected result because \TeX\ kept expanding macros and the registers got unexpected results. Here is the correct method:
\begin{codeexample}[]
1. \count0=42 % a white space after the number aborts the reading process.
The value is now `\the\count0'.
2. The following code will absorb the `3' of '3.':
\def\macro{1234}
\count0=\macro % a white space after a macro will be absorbed by TeX, so this is wrong.
3. The value is now `\the\count0'.
4. Use \textbackslash relax after an assignment to end scanning:
\count0=\macro\relax
5. The value is now `\the\count0'.
\end{codeexample}
	The command |\relax| tells \TeX\ to ``relax'': it stops scanning for tokens, but |\relax| doesn't expand to anything.
	\index{relax@\texttt{\textbackslash relax}}%
\end{command}

\begin{command}{\dimen\meta{num}}
	There are also 255 registers for fixed point numbers which are used pretty much in the same way as the |\count| registers -- but |\dimen| register assignments require a unit like `|cm|' or `|pt|'.

	String access with `|\the|' works in exactly the same way as for |\count| registers.
\begin{codeexample}[]
\dimen0=1pt
The value is now \the\dimen0.
\dimen0=0.0001pt
The value is now \the\dimen0.
\def\macro{1234.5678}
\dimen0=\macro pt
The value is now \the\dimen0.
\end{codeexample}
	The same rules with expansion of macros after assignments apply here as well.

	The |\dimen| registers perform their arithmetics internally with 32 bit scaled integers, so called `scaled point' with unit `|sp|'. It holds |1pt=65536sp|=$2^{16}$|sp|. One of the 32 bits is used as sign. The total number range in |pt| is $[-(2^{30}-1)/2^{16}, (2^{30}-1)/2^{16} ] = [-16383.9998,+16383.9998]$\footnote{Please note that this does not cover the complete range of a 32 bit integer, I do not know why.}.
\end{command}

\begin{command}{\toks\meta{number}}
\label{cmd:toks}
	There are also 255 token registers which can be thought of as special string variables. Of course, every macro assignment |\def\macro|\marg{content} is also some kind of string variable, but token registers are special: their contents won't be expanded when used with |\the\toks|\meta{number}. This can be used for fine grained expansion control, see Section~\ref{sec:expansion:control} below.
\end{command}

\subsubsection{Allocating Registers}

\subsubsection{Using More than 256 Registers}

\subsection{Arithmetics in \TeX}
\begin{command}{\advance\meta{register}\texttt{ by}\meta{quantity}}
\begin{codeexample}[]
\count0=42
\advance\count0 by 10
The value is now \the\count0.
\end{codeexample}

\begin{codeexample}[]
\dimen0=1pt
\advance\dimen0 by 10pt
The value is now \the\dimen0.
\end{codeexample}
\end{command}

\begin{command}{\multiply\meta{register}\texttt{ by}\meta{integer}}
\begin{codeexample}[]
\count0=42
\multiply\count0 by -10
The value is now \the\count0.
\end{codeexample}

\begin{codeexample}[]
\dimen0=0.5pt
\multiply\dimen0 by 20
The value is now \the\dimen0.
\end{codeexample}
\end{command}

\begin{command}{\divide\meta{register}\texttt{ by}\meta{integer}}
	This allows integer division by \meta{integer} with truncation.
\begin{codeexample}[]
\count0=5
\divide\count0 by 2
The value is now \the\count0.
\end{codeexample}

	Scaling of |\dimen| registers:
\begin{codeexample}[]
\dimen0=10pt
\divide\dimen0 by 20
The value is now \the\dimen0.
\end{codeexample}
\end{command}

\begin{command}{\dimen\meta{number}\texttt{=}\meta{fixed point number without unit}\textbackslash dimen\meta{number}}
	This allows fixed point multiplication in |\dimen| registers.
\begin{codeexample}[]
\dimen1=50pt
\dimen0=0.6\dimen1
The value is now \the\dimen0.
\end{codeexample}
\end{command}


\subsection{Expansion Control}
\label{sec:expansion:control}
Expansion is what \TeX\ does all the time. Thus, expansion control is a key concept for understanding how to program in \TeX.

The first thing to know is: \TeX\ deals the input as a long, long sequence of ``tokens''. A token is the smallest unit which is understood by \TeX. Each character becomes a token the first time it is seen by \TeX. Every macro becomes a (single!) token the first time it is seen by \TeX.

The second thing to know is what characters are \emph{before} \TeX\ has seen them. Although this knowledge is rarely needed in every day's life, it is nevertheless important. The characters which are in the input document are nothing but characters at first. Even the characters known to have a special meaning like `|%|', `|\|' or the braces `|{}|' are \emph{not} special -- until they have been converted to a token. This happens when \TeX\ encounters them the first time during its linear processing of the character stream. A token stays a token - and it will remain the same token forever. If you manage to tell \TeX\ that `|\|' is a normal character and \TeX\ sees just one backslash, this backslash will be a normal character token -- even if the meaning of all following backslashes is again special.

Now, we are given a very long list of tokens \meta{token1}\meta{token2}\meta{token3}\meta{token4}\meta{token5}$\cdots$. \TeX\ processes these tokens one-by-one in linear sequence. If \meta{token1} is a character token like `|a|', it is typeset. This is not what I want to write about here now; my main point is how to program in \TeX\footnote{Of course, typesetting is an art in itself and there is a lot to read about it. Just not here in these notes.}. So, the interesting thing in these notes is when \meta{token1} is a macro.

\subsubsection{Macros}
We have already seen some applications of macros above. Actually, most users who are willing to read notes about \TeX\ programming will have seen macros and may have written some on their own -- for example using |\newcommand| (|\newcommand| is a ``more high--level'' version of |\def| used only in \LaTeX).

A macro has a name and is treated as an elementary token in \TeX\ (even if the name is very long). A macro has replacement text. As soon as \TeX\ encounters a macro, it replaces its occurrence with the replacement text. Furthermore, a macro can consume one or more of the following tokens as arguments.
\begin{codeexample}[]
\def\macro{This here is actually the replacement text.}
Executing it: `\macro'.
\end{codeexample}
\begin{codeexample}[]
\def\macro#1{replacement with first argument=#1}
Invoking it: \macro{hello!}.
\end{codeexample}
This here is not really a surprise. What might come as a surprise is that the accepted arguments can be pretty much anything.
\begin{codeexample}[]
\def\macro#1-#2.{replacement with arguments: `#1' and `#2'.}
Invoking it: \macro a-sign.
\end{codeexample}
\noindent The last example |\macro| runs through the token list which follows the occurrence of |\macro|. This token list is ``|a-sign.|''. Macro expansion is greedy, that means the first matching pattern is used. Now, our |\macro| expected something, then a minus sign `|-|', then another (possibly long) argument, then a period `|.|'. The argument between |\macro| and the minus sign is available as |#1| and the tokens between the minus sign and the period as |#2|.

\begin{codeexample}[]
\def\macro(#1,#2,#3){I found arguments `#1', `#2' and `#3'.}
\macro(42,43,44)
\end{codeexample}

As we have seen, macros can be used to manipulate the input tokens by expansion: they take some input arguments (maybe none) away and insert other tokens into the input token list. These tokens will be the next to process. We will soon learn more about that.

There is a command which helps to understand what \TeX\ does here:

\begin{command}{\meaning\meta{macro}}
	This command expands to the contents of \meta{macro} as it is seen by \TeX.
\begin{codeexample}[]
\def\macro{Replacement \textmacro text  \count0=42 \the\count0.}
\message{Debug message: '\meaning\macro'}
\end{codeexample}
As result, the log file and terminal output will contain

|Debug message: 'macro:->Replacement \textmacro text \count 0=42 \the \count 0.'|
\end{command}

The last example already shows something about |\def|: the replacement text can still contain other macros.

\begin{command}{\def\meta{\textbackslash macroname}\meta{argument pattern}\marg{replacement text}}
	 A new macro named \meta{macroname} will be defined (or re-defined). The \marg{replacement text} is the macro body, whenever the macro is executed, it expands to \marg{replacement text}. The \marg{replacement text} is a token list which can contain other macros. On the time of the definition, \TeX\ does \emph{not} process (expand) the \marg{replacement text}.

	 The \marg{replacement text} will only be expanded if the macro is executed. This does also apply to any macros which are inside of \marg{replacement text}.
\begin{codeexample}[]
\def\macroone{This is macro one}
\def\macrotwo{Macro two contains \macroone.}
Now, I execute it: \macrotwo.
\def\macroone{Redefined macroone}
Now, I execute the second macro again: \macrotwo.
\end{codeexample}

	Macros can be defined almost everywhere in a \TeX\ document. They can also be invoked almost everywhere.

	The \meta{argument pattern} is a token list which can contain simple strings or macro parameters `|#|\meta{number}' or other macro tokens. The \meta{number} of the first parameter is always 1, the second must have 2 and so on up to at most 9. Valid argument patterns are `|#1#2#3|', `|(#1,#2,#3)|' or `|---\relax|'. If \TeX\ executes a macro, it searches for \meta{argument pattern} in the input token list until the first match is found. If no match can be found, it aborts with a (more or less helpful) error message.
\begin{codeexample}[]
\def\macroone abc{\macrotwo}
\def\macrotwo def{\macrothree}
\def\macrothree#1{Got `#1'}
\macroone abcdefg
\end{codeexample}
	The last example contains three macro definitions. Then, \TeX\ encounters |\macroone|. The input token list is now
	
	`|\macroone abcdefg|'.

	The space(s) following |\macroone| are ignored by \TeX, they delimit the \meta{\textbackslash macroname}. Now, \TeX\ attempts to find matches for \meta{argument pattern}. It expects `|abc|' -- and it finds `|abc|'. These three tokens are \emph{removed} from the input token list, and \TeX\ inserts the replacement text of |\macroone| which is |\macrotwo|. At that time, the input token list is

	`|\macrotwo defg|'.

	Now, the same game continues with |\macrotwo|: \TeX\ searches for the expected \marg{argument pattern} which is `|def|', erases these tokens from the input token list and inserts the replacement text of |\macrotwo| instead. This yields

	`|\macrothree g|'.

	Finally, |\macrothree| expects one parameter token (or a token list enclosed in parenthesis). The next token is `|g|', which is consumed from the input token list and the replacement text is inserted -- and `|#1|' is replaced by `|g|'. Then, the token list is

	`|Got `g'|'.

	This text is finally typeset (because it doesn't expand further).
\end{command}

What we have seen now is how \TeX\ macros can be used to modify the token list. It should be noted explicitly that macro expansion does is in no way limited to those tokens provided inside of \marg{replacement text} -- if the last argument in \marg{replacement text} is a macro which requires arguments, these arguments will be taken from the following tokens. Using nested macros, one can even process a complete part of the token list, in a manner of loops (but we don't know yet how to influence macro expansion conditionally, that comes later).

Let's try to solve the following task. Suppose you have a macro named |\point| with \meta{argument pattern} `|(#1,#2)|', i.e.

|\def\point(#1,#2){we do something with #1 and #2}|.

\noindent
Suppose furthermore that you want to invoke |\point| with the contents which is stored in another macro. After all, macros are some kind of string variables -- it makes sense to accumulate or generate string variables which will then be used as input for other macros. Let's assume we have |\temp| and |\temp| contains `|(42,1234)|'. A first choice to invoke |\point| would be to use |\point\temp|. But: |\point| searches for an argument pattern which starts with `|(|', not with |\temp|! The invocation fails.

\begin{command}{\expandafter\meta{token}\meta{next token}}
	The |\expandafter| command is an -- at first sight confusing -- method to alter the input token list. But: it solves our problem with |\point\temp|!
\begin{codeexample}[]
\def\point(#1,#2){we do something with #1 and #2}
\def\temp{(42,1234)}
\expandafter\point\temp
\end{codeexample}
	Why did that work!? The command |\expandafter| scans for the token after |\expandafter| in the input token list. This is |\point| in our case. Then, it scans for the next token which is |\temp| in our case (remember: macros are considered to be elementary tokens, just like characters `|a|' or so). The two scanned arguments are removed from the input token list. Then, |\expandafter| \emph{expands} the \meta{next token} one time. In our case, \meta{next token} is |\temp|. The first level of expansion of |\temp| is `|(42,1234)|'.  Then, |\expansion| inserts the (unexpanded) \meta{token} followed by the (expanded) contents of \meta{next token} back into the input token list. In single steps:

	\begin{enumerate}
		\item |\expandafter\point\temp|
		\item Expand |\expandafter|: next two tokens are `|\point\temp|'.
		\item Use |\point| as \meta{token} and |\temp| as \meta{next token}.
		\item Expand |\temp| once, which leads to the tokens `|(42,1234)|'.
		\item re-insert \meta{token} and the expansion of \meta{next token} back into the input token list. The list is then
			
			`|\point(42,1234)|'.
		\item Expand |\point| as next token.
	\end{enumerate}

	A further example: suppose we want to invoke |\theimportantmacro|\marg{argument}. However, \marg{argument} is contained in another macro! Furthermore, |\theimportantmacro| is defined to take exactly one parameter and our desired argument may have more than one token (which means we need to surround it with braces). This can be solved by the listing below.
\begin{codeexample}[]
\def\theimportantmacro#1{I got the pre-assembled argument `#1' here.}
\def\temp{xyz}
\expandafter\theimportantmacro\expandafter{\temp}
\end{codeexample}
	Now, what happens here? Let's apply the rules step by step again:
	\begin{enumerate}
		\item After the initial definitions, the token list is |\expandafter\theimportantmacro\expandafter{\temp}|.
		\item \TeX\ expands |\expandafter|, using |\theimportantmacro| as \meta{token} and the second |\expandafter| as \meta{next token}.
		\item According to the rules, \TeX\ expands \meta{next token} once. But: \meta{next token} is again a macro, namely |\expandafter|! Does that make a difference? No:
			\begin{enumerate}
				\item The token list after the second |\expandafter| is `|{\temp}|' (3 tokens).
				\item The \meta{token} is thus `|{|' and \meta{next token} is `|\temp|'.
				\item The expansion of \meta{next token} is `|xyz|'.
				\item The second |\expandafter| re-inserts its \meta{token} and expanded \meta{next token}, which is
					
					`|{xyz|'.

					Note that the closing brace `|}|' has not been touched at all, \TeX\ hasn't even seen it so far.
			\end{enumerate}
			We come back from the recursion. Remember: \meta{token} is |\theimportantmacro| and the top-level expansion of \meta{next token} is -- as we have seen above -- `|{xyz|'.
		\item \TeX\ re-inserts \meta{token} and the expansion of \meta{next token} to the input token list, which leads to

			`|\theimportantmacro{xyz}|'.

			The closing brace `|}|' has not been touched, it simply resides in the input token list.
		\item \TeX\ expands |\theimportantmacro|.
	\end{enumerate}

	The \meta{next token} is expanded exactly once. We have already seen that if \meta{next token} is a macro which does substitutions on its own, these substitutions will be performed recursively. But what means `once' exactly? We will need to use |\meaning| to check that (or the |\tracingmacros| tools) because we need to see what \TeX\ does.
\begin{codeexample}[]
\def\macroone{This is macro one \macrotwo}
\def\macrotwo{--2--}
\def\macrothree#1{\def\macrofour{4[#1]}}
\expandafter\macrothree\expandafter{\macroone}%
So far, nothing has been typeset. But now: \macrofour.
\message{We have macrofour = \meaning\macrofour}%
\end{codeexample}
	The logfile (and terminal) will now contain

	`|We have macrofour = macro:->4[This is macro one \macrotwo ]|'.

	What happened? We can proceed as in the last example. After the two |\expandafter| expansions, \TeX\ finds the input token list

	`|\macrothree{This is macro one \macrotwo}|'

	which, after execution, defines |\macrofour| to be `|This is macro one \macrotwo|'. The top-level expansion of |\macroone| has not expanded the nested call to |\macrotwo|.


	So, |\expandafter| is a normal macro which can be expanded -- and it is even possible to expand an |\expandafter| by another |\expandafter|.
\end{command}

What we have seen so far is
\begin{enumerate}
	\item the |\def| command which stores \emph{unexpanded} arguments in a macro variable and
	\item the |\expandafter| which allows control over top-level expansion of macros (it expands one time).
\end{enumerate}
\TeX\ provides two more features for expansion control: the |\edef| macro and token registers.

\begin{command}{\edef\meta{\textbackslash macroname}\meta{argument pattern}\marg{replacement text}}
	The |\edef| command is the same as |\def| insofar as it defines a new macro. However, it expands \marg{replacement text} until only unexpandable tokens remain (|\edef| $=$ expanded definition).
\begin{codeexample}[]
\def\a{3}
\def\b{2\a}
\def\c{1\b}
\def\d{value=\c}
\message{Macro `d' is defined to be `\meaning\d'}
\edef\d{value=\c}
\message{Macro `d' is e-defined to be `\meaning\d'}
\expandafter\def\expandafter\d\expandafter{\c}
\message{Macro `d' is defined to be `\meaning\d' using expandafter}
\end{codeexample}
	This listing results in the log-file output

	|Macro `d' is defined to be `macro:->value=\c '|

	|Macro `d' is e-defined to be `macro:->value=123'|

	|Macro `d' is defined to be `macro:->1\b ' using expandafter|

	\noindent So, |\def| does not expand at all, |\edef| expands until it can't expand any further and the |\expandafter| construction expands |\c| one time and defines |\d| to be the result of this expansion.

	Although possible, it might not occur too often to specify \meta{argument pattern} for an |\edef| because the expansion is immediate in contrast to |\def|. But it works in the same way: the positional arguments |#1|, |#2|$,\dotsc,$ |#9| will be replaced with their arguments.

	The expansion of \marg{replacement text} happens in the same way as the expansion the main token list of \TeX.

	Now, what exactly does ``expands until only unexpandable tokens remain'' mean? Our example indicates that the three tokens |1|, |2| and |3| are not expandable while the macros |\c|, |\b| and |\a| could be expanded. There is one large class of \TeX\ commands which can't be expanded: any assignment operation. The example
\begin{codeexample}[]
\edef\d{\count0=42}
\message{Macro `d' is defined to be `\meaning\d'}
\def\a{1234}
\edef\d{\advance\count0 by\a}
\message{Macro `d' is defined to be `\meaning\d'}
\end{codeexample}
\noindent yields the log-messages 

|Macro `d' is defined to be `macro:->\count 0=42'| and

|Macro `d' is defined to be `macro:->\advance \count 0 by1234'|.

So, assignment and arithmetics operations are \emph{not} expandable, they remain as executable tokens in the newly defined macro. This does also hold for |\let| and other assignment operations.

	Interestingly, conditional expressions using |\if| $\dotsb$ |\fi| \emph{are} expandable, but we will come to that later.

	There is also a method to convert a macro temporarily into an unexpandable token: the |\noexpand| macro.
\end{command}

\begin{command}{\noexpand\meta{expandable token}}
	The |\noexpand| command is only useful inside of the \marg{replacement text} of an |\edef| command. As soon as |\edef| encounters the |\noexpand|, the |\noexpand| will be removed and the \meta{expandable token} will be converted into an unexpandable token. Thus, the code
\begin{codeexample}[]
\edef\d{Invoke \noexpand\a another macro}
\message{Macro `d' is defined to be `\meaning\d'}
\end{codeexample}
\noindent yields the terminal output

|Macro `d' is defined to be `macro:->Invoke \a another macro'|

because |\noexpand\a| yields the token `|\a|' (unexpanded)\footnote{The \texttt{\textbackslash noexpand} key is actually used to implement the \LaTeX\ command \texttt{\textbackslash protect}: \LaTeX's concept of moveable arguments is implemented with \texttt{\textbackslash edef}.}.
\end{command}

\subsubsection{Token Registers}
Now, we turn to token registers. As we have already seen in Section~\ref{cmd:toks}, a token register stores a token list. A macro does also store a token list in its \marg{replacement text}, so where is the difference? There are two differences:
\begin{enumerate}
	\item Token registers are faster.
	\item The contents of token registers will \emph{never} be expanded.
\end{enumerate}
I can't give numbers for the first point -- I have just read it in~\cite{texbook}. But the second point allows expansion control. While |\edef| allows ``infinite'' expansion, token registers allow only top--level expansion, just like |\expandafter|. But they can be used in a more flexible (and often more efficient) way than |\expandafter|.

The following examples demonstrates the second point.
\begin{codeexample}[]
\toks0={A \token list \a \b \count0=42 will never be expanded}
\edef\d{\the\toks0 }% the space token is important!
\message{Macro `d' is defined to be `\meaning\d'}
\end{codeexample}
\noindent Executing this code fragment yields the log output

|Macro `d' is defined to be `macro:->A \token list \a \b \count 0=42 will never be expanded'|.

So, the contents of |\toks0| has been copied unexpanded into |\d|, although we have just |\edef|. Note that the space token after |\the\toks0| is indeed important! \TeX\ uses it to delimit the integer |0|. Without the space token, it would have continued scanning, even beyond the boundaries of the replacement text of |\edef| (see Section~\ref{sec:variables} for details about this scanning).

The example is very simple, and we could have done the same with |\expandafter| as before. But let's try something more difficult: we want to assemble a new macro which consists of different pieces. Each piece is stored in a macro, and for whatever reason, we only want top-level expansion of the single pieces. And: the pieces won't be adjacent to each other. We can assemble the target macro using the following example listing.

\begin{codeexample}[]
\def\piecea{\a{xyz}}
\def\pieceb{\count0=42 }
\def\piecec{string \b}
\toks0=\expandafter{\piecea}
\toks1=\expandafter{\pieceb}
\toks2=\expandafter{\piecec}
\edef\d{I have \the\toks0 and \the\toks1 and \the\toks2}
\message{Macro `d' is defined to be `\meaning\d'}
\end{codeexample}

The first three lines define our pieces. Each of the macros |\piecea|, |\pieceb| and |\piecec| contains tokens which should not be expanded during the definition of |\d|. The three following lines assign the top-level expansion of our pieces into token registers. Since |\toks0={\piecea}| would have stored `|\piecea|' into the token register, we need to use |\expandafter| here\footnote{We could have eliminated the \texttt{\textbackslash piece*} macros by writing everything into token registers directly. But I think this example is more realistic.}. Then, we use |\the\toks|\meta{number} to insert the contents of a token list somewhere -- in our case, into the expanded replacement text of our macro |\d|. Thus, the complete example yields the log--output

|Macro `d' is defined to be `macro:->I have \a {xyz}and \count 0=42 and string \b '|.

\noindent It \emph{is} possible to get exactly the same result using (a lot of) |\expandafter|s. Don't try it.


\subsubsection{Summary of macro definition commands}
Besides |\def| and |\edef|, there are some more commands which allow to define macros (although the main functionality is covered by |\def| and |\edef|). Here are the remaining definition commands.
\begin{command}{\def\meta{\textbackslash macroname}\meta{argument pattern}\marg{replacement text}}
	Defines a new macro named |\macroname| without expanding \marg{replacement text}, see above.	
\end{command}
\begin{command}{\edef\meta{\textbackslash macroname}\meta{argument pattern}\marg{replacement text}}
	Defines a new macro named |\macroname|, expanding \marg{replacement text} completely (see above).
\end{command}
\begin{command}{\let\meta{\textbackslash newmacro}=\meta{token}}
	Defines or redefines |\newmacro| to be an equivalent to \meta{token}. For example, |\let\a=\b| will create a new copy of macro |\b|. The copy is named |\a|, and it will have exactly the same \marg{replacement text} and \meta{argument pattern} as |\b|.

	It is also possible that \meta{token} is something different than a macro, for example a named register or a single character.
\end{command}

\begin{command}{\gdef\meta{\textbackslash macroname}\meta{argument pattern}\marg{replacement text}}
	A shortcut for |\global\def|. It defines |\macroname| globally, independent of the current scope.

	You should avoid macros which exist in both, the global namespace and a local scope, with different meanings. Section~\ref{sec:scopes} explains more about scoping.
\end{command}
\begin{command}{\xdef\meta{\textbackslash macroname}\meta{argument pattern}\marg{replacement text}}
	A shortcut for |\global\edef|. It defines |\macroname| globally, independent of the current scope.

	You should avoid macros which exist in both, the global namespace and a local scope, with different meanings. Section~\ref{sec:scopes} explains more about scoping.
\end{command}

\begin{command}{\csname\meta{expandable tokens}\textbackslash endcsname}
	This command is not a macro definition, it is a definition of a macro's \emph{name}. The ``cs'' means ``control sequence''. The |\csname|, |\endcsname| pair defines a control sequence name (a macro name) using \meta{expandable tokens}. The control sequence character `|\|' will be prepended automatically by |\csname|.\footnote{In fact, the contents of \texttt{\textbackslash escapechar} will be used here. If its value is -1, no character will be prepended. The same holds for any occurrence where a backslash would be inserted by \TeX\ commands.}
\begin{codeexample}[]
\def\macro{Content}
This here is normal usage: `\macro'.
This here uses csname: `\csname macro\endcsname'.
\end{codeexample}
	\noindent The example demonstrates that |\csname|\meta{expandable tokens}|\endcsname| is actually the same as if you had written |\|\meta{expandable tokens} directly -- but the |\csname| construction allows much more tokens inside of macro names:
\begin{codeexample}[]
\expandafter\def\csname a01macro with.strange.chars\endcsname{Content}
I use a strange macro. Here is it: `\csname a01macro with.strange.chars\endcsname'.
\end{codeexample}
	\noindent The example uses |\expandafter| to expand |\csname| one time. The top--level expansion of |\csname| is a single token, namely the control sequence name. Then, |\def| is used to define a macro with the prepared macro name.

	When |\csname| is expanded, it parses all tokens up to the next |\endcsname|. Those tokens will be expanded until only unexpandable tokens remain (as in |\edef|). The resulting string will be used to define a macro name (with the control sequence character `|\|' prepended). The fact that \meta{expandable tokens} is expanded allows to use ``indirect'' macro names:
\begin{codeexample}[]
\def\macro{onetwothree}
\expandafter\def\csname macro\macro\endcsname{Content}
I have just defined \expandafter\string\csname macro\macro\endcsname
with replacement text `\csname macro\macro\endcsname'.
\end{codeexample}
	\noindent I suppose the example is self-explaining, up to the |\string| command which is described below.

	Due do this flexibility, |\csname| is used to implement all (?) of the available key--value packages in \TeX.
\end{command}

\begin{command}{\string\meta{\textbackslash macro}}
	This command does not define a macro. Instead, it returns a macro's name as a sequence of separate tokens, including the control sequence token `|\|'.
\begin{codeexample}[]
\def\macro{Content}
I have just defined `\string\macro' using `\string\def'.
\end{codeexample}

	You can also use |\string| on other tokens -- for example characters. That doesn't hurt, the character will be returned as-is.
\end{command}

\subsubsection{Debugging Tools -- Understanding and Tracing What \TeX\ Does}
\begin{command}{\message\marg{tokens}}
\end{command}
\begin{command}{\meaning\meta{\textbackslash macro}}
\end{command}
\begin{command}{\tracingmacros=2}
\end{command}
\begin{command}{\tracingcommands=2}
\end{command}
\begin{command}{\tracingrestores=1}
\end{command}

\subsection{The Scope of a Variable}
\label{sec:scopes}
Each programming language knows the concept of a scope: they limit the effect of variables or routines. However, \TeX's scoping mechanisms have not been designed for programming -- \TeX\ is a typesetting language. Many programming languages like |C|, |C++|, |java| or a lot of scripting languages define the scope of a variable using the place where the variable has been defined. For example, the |C| fragment
\begin{codeexample}[code only]
int i = 42;

{
	++i;
	int i = 5;
}
\end{codeexample}
\noindent changes the value of the outer |i| to |43|. The inner |i| is |5|, but it will be deleted as soon as the closing brace is encountered. It may even be possible to access both, the value of the inner |i| variable and the value of the outer |i| variable, at the same time.

In \TeX, braces are also used for scopes. But: while \TeX\ will also destroy any variables (macros) defined inside of a scope at the end of that scope, it will \emph{also} undo any change which has been applied inside of that scope.
\begin{codeexample}[]
\def\i{42}
{
	\def\i{43}
	\def\b{2}
}
The value of \textbackslash i is now \i.
\end{codeexample}
\noindent The listing above defines |\i|, enters a local scope (a \TeX\ ``group'') and changes |\i|. However, due to \TeX's scoping rules, the old program state will be restored \emph{completely} after returning from the local group! Neither the change to |\i| nor the definition of |\b| will survive. The same holds for register changes or other assignments.

\TeX\ groups can be created in one of three ways: using curly braces\footnote{Or other tokens with the correct category code, compare~\cite{texbook}.}, using |\begingroup| or using |\bgroup|. Curly braces are seldom used to delimit \TeX\ groups because the other commands are more flexible. If one uses curly braces, they need to match up -- it is forbidden to have unmatched curly braces.
\begin{command}{\begingroup}
	Starts a new \TeX\ group (a local scope). The scope will be active until it will be closed by |\endgroup|. The |\endgroup| command can occur later in the main token list.
\end{command}
\begin{command}{\endgroup}
	Ends a \TeX\ group which has been opened with |\begingroup|.
\end{command}
\begin{command}{\bgroup}
	A special variant of |\begingroup| which can also be used to delimit arguments to |\hbox| or |\vbox| (i.e. it avoids the necessity to provide matched curly braces in this context).

	The |\bgroup| macro is also useful to test whether the next following character is an opening brace (see |\futurelet|).

	If one just needs to open a \TeX\ group, one should prefer |\begingroup|.
\end{command}
\begin{command}{\egroup}
	Closes a preceding |\bgroup|.
\end{command}

\TeX\ does not know how to write into macros of an outer scope -- except for the topmost (global) scope. This restriction is quite heavy if one needs to write complex structures: local variables should be declared inside of local groups, but changes to the structure should be written to the outer group. There is no direct possibility to do such a thing (except global variables).

\subsubsection{Global Variables}
\TeX\ knows only ``global'' variables and ``local'' variables. A local variable will be deleted at the end of the group in which it has been declared. All values assigned locally will also be restored to their old value at the end of the group.

A global variable, on the other hand, maintains the same value throughout \emph{every} scope. Usually, the topmost scope is the same as the one used for global variables: if you define anything in your \TeX\ document, you add commands on global scope. It is also possible to explicitly make assignments or definitions in the global scope.

\begin{command}{\global\meta{definition or assignment}}
	The definition which follows |\global| immediately will be done globally.
\begin{codeexample}[code only]
{
	\global\def\a{123}
	\global\advance\count0 by3
	\global\toks0={34}
}
\end{codeexample}
\end{command}

\begin{command}{\globaldefs=\mchoice{-1,0,1} (initially 0)}
	I cite from~\cite{texbook}: ``If the |\globaldefs| parameter is positive at the time of an assignment, a prefix of |\global| is automatically implied; but if |\globaldefs| is negative at the time of the assignment, a prefix of |\global| is ignored. If |\globaldefs| is zero (which it usually is), the appearance of nonappearance of |\global| determines whether or not a global assignment is made.''
\end{command}


\subsubsection{Transporting Changes to an Outer Group}
There are a couple of methods to ``transport'' changes to an outer scope. Some are copy operations, some require to redo the changes again after the end of the scope. All of them can be realized using expansion control.

Let's start with macro definitions which should be carried over the end of the group. I see the following methods:
\begin{itemize}
	\item Copy the macro into a global, temporary variable (or even token register) and get that value after the scope.
\begin{codeexample}[code only]
\def\initialvalue{0}
{
	% do something:
	\def\initialvalue{42}
	\global\let\myglobaltemporary=\initialvalue
}
\let\initialvalue=\myglobaltemporary
\end{codeexample}
	The idea is that |\myglobaltemporary| is only used temporary; its value is always undefined and can be overwritten at any time. This allows to use a local variable |\initialvalue|.

	Please note that you should not use variables both globally and locally. This confuses \TeX\ and results in a slow-down at runtime.

	\item ``Smuggle'' the result outside of the current group. I know this idea from the implementation of~\cite{tikz} written by Mark Wibrow and Till Tantau. The idea is to use several |\expandafter|s and a |\def| to redefine the macro directly after the end of the group:
\begin{codeexample}[code only]
\def\smuggle#1\endgroup{%
	\expandafter\endgroup\expandafter\def\expandafter#1\expandafter{#1}%
}

\begingroup
	\def\variable{12}
	\edef\variable{\variable34}
	\edef\variable{\variable56}
	\smuggle\variable
\endgroup
\end{codeexample}
	The technique relies on groups started with |\begingroup| and ended with |\endgroup| because unmatched braces are not possible with |\def|. The effect is that after all those |\expandafter|s, \TeX\ encounters the token list

	|\endgroup\def\variable{123456}|

	at the end of the group.

	\item Use the aftergroup stack. \TeX\ has a special token stack of limited size which can be used to re-insert tokens after the end of a group. However, this does only work efficiently if the number of tokens which need to be transported is small and constant (say, at most three). It works by prefixing every token with |\aftergroup|, compare~\cite{texbook} for details.
\end{itemize}

Sometimes one needs to copy other variables outside of a scope. The trick with a temporary global variable works always, of course. But it is also possible to define a macro which contains commands to apply any required changes and transport that macro out of the scope.

\subsection{Branching}
\label{sec:branching}

Here we discuss some of the available branching constructions of \TeX, with emphasis on conditions involving numbers and tokens.

\begin{command}{ifnum\meta{count/integer number}=\meta{count/integer number}\meta{true-block}\textbackslash else\meta{false-block}\textbackslash fi}
|\ifnum| compare integer numbers or integer registers (|\count| registers) and contains two branches, one is executed in the true case, the other in the case of false:
\begin{codeexample}[]
\ifnum1=2 % this space is important.
This is shown if above were true.
\else
This is shown if above results to false.
\fi
\end{codeexample}

	Note that the |\else| with its \meta{false-block} is optional.
\end{command}

\begin{command}{ifdim\meta{dimen/fixed point number}=\meta{dimen/fixed point number}\meta{true-block}\textbackslash else\meta{false-block}\textbackslash fi}
	Similar to |\ifnum|, |\ifdim| compares two fixed point numbers or |\dimen| registers. The numbers must have a unit.
\begin{codeexample}[]
\ifdim1pt=2pt % this space is important.
This is shown if above were true.
\else
This is shown if above results to false.
\fi
\end{codeexample}
\end{command}

\begin{command}{ifx\meta{token1}\meta{token2}\meta{true-block}\textbackslash else\meta{false-block}\textbackslash fi}
|\ifx| is a bit more complex: It compares two \emph{tokens} up to their first-level expansion.
\begin{codeexample}[]
\def\empty{\empty}
\ifx\empty\empty %
This is shown if the two tokens have equal expansion.
\else
This is shown if the two tokens expand to something different.
\fi
\end{codeexample}

Here, we have defined a token |\empty| to be a replacement for |\empty| and subsequently have compared whether these two tokens are equal in first-level expansion. Note that the definition is actually nonsense. If \TeX{} ever were to go through the whole expansion -- i.e. we would put |\empty| somewhere else -- it would do so indefinitely. However, with |\ifx| only first-level expansion is done and compared. Hence, the statement evaluates to true.

Have a look at the following example:
\begin{codeexample}[]
\def\empty{\relax}
\ifx\empty\relax %
This is shown if the two tokens have equal expansion.
\else
This is shown if the two tokens expand to something different.
\fi
\end{codeexample}

On first glance, this should evaluate to true: |\empty| is defined as a replacement for |\relax|. But it does not. Why? 

|\empty| is expanded to |\relax|, however |\relax| expanded has a different meaning, namely stop scanning and not |\relax| anymore. Hence, they are different and the statement is false! If the expansion in |\ifx| were to be taken till maximum, both would be equal but not in the case of a comparison on first-level expansion only.
\end{command}

\begin{command}{if\meta{token1}\meta{token2}\meta{true-block}\textbackslash else\meta{false-block}\textbackslash fi}
	The |\if| comparison is closely related to the |\ifx| conditional, with one major exception: it expands tokens until it finds the next two unexpandable tokens. If these two tokens are the same, it expands to the \meta{true-block}, otherwise to the \meta{false-block}.

	The |\if| conditional should be handled with care as it might produce undesirable effects. Use it only if you know what you do.

	A useful example is if you \emph{know} that a macro contains at most one character, and you want to test for a particular one:
\begin{codeexample}[]
\def\choice{a}
\if b\choice
  This is shown for the `b' choice.
\else
  This is shown for all other choices.
\fi
\end{codeexample}
\end{command}

\begin{command}{iftrue\meta{true-block}\textbackslash else\meta{false-block}\textbackslash fi}
	A ``conditional'' which always invokes the \meta{true-block}.
\end{command}
\begin{command}{iffalse\meta{true-block}\textbackslash else\meta{false-block}\textbackslash fi}
	A ``conditional'' which always invokes the \meta{false-block}.
\end{command}

\subsubsection{Boolean Variables}
\begin{command}{\newif\meta{if-name}}
	You can declare a new ``boolean variable `|\ifsupermanmode| by means of |\newif\ifsupermanmode|. Afterwards, you can use the |\supermanmodetrue| and |\supermanmodefalse| switches to assign the boolean and |\ifsupermanmode| to check it.

	The \meta{if-name} has to start with |\if| (to support scans for nested |\if...\fi| pairs, see below).
\end{command}

\subsubsection{Special Cases for Conditionals}
Whenever you work with |\if|\ldots\ and friends, you should know the following features:
\begin{enumerate}
	\item |\if...\else...\fi| is expandable (including each of the single macros |\if...|, |\else| and |\fi|), which means you can even use it inside of |\edef|:
\begin{codeexample}[]
\def\choice{a}
\edef\temp{The choice is \if a\choice `a'\else not `a'\fi}
We have now \texttt{\string temp=\meaning\temp}.
\end{codeexample}

\begin{codeexample}[]
\def\shownexttoken#1{The next token is `\texttt{\string#1}'.}
\def\mymacro{%
 	\ifnum1=1 %
 		\expandafter\shownexttoken%
 	\fi%
}%
\mymacro 23
\end{codeexample}
	This example is tricky. What would have happened without the |\expandafter|!? Well, |\shownexttoken| would be invoked with |#1=\fi|. This would lead to an error because the |\fi| would be missing, and it would spoil the effect since we do not want the |\fi| to be seen -- we expected |#1=2|. The |\expandafter| first expands |\fi| (which simply removes the |\fi| without further effect) such that |\shownexttoken| will see the |2| token in our example above. This would also have worked if there was an |\else| branch instead of |\fi|.
	
	\item You should generally make sure that the matching |\else| or |\fi| tokens are ``directly reachable'', i.e.\ without token expansion.

	The background here is that \TeX\ works on a token--based level: Whenever it encounters an |\if|\ldots\ statement, it evaluates it and scans tokens to find the matching end part (either an |\else| or an |\fi| token). But it will not expand tokens during this scan, although it will count nested |\if...\fi| pairs! Thus, if you are careless, it might become confused and your conditional will go awry.
\end{enumerate}

\subsection{Loops}
\label{sec:loops}
As you have seen, in \TeX{} we have a very specific control over token expansion. This makes it possible to construct even loops via means of recursion. In essence, a loop consist of the following parts:
\begin{itemize}
 \item counter or, more generally, list of items
 \item incrementor, or more generally, a next item picker
 \item threshold or, more generally, an end list marker
 \item a check of the threshold or end marker, respectively
\end{itemize}

Leafing through the sections above, we realize that all of this is actually in place: We do know about counters, we do know about branching. Only the specifics of how to create these loops is still to be made clear. We will show both cases, the counting loop and the loop over a list of items in the following in detail.

In general, for a loop done via a recursion we need two definitions: One for the loop start and another for the loop step.

\subsubsection{Counting loops}
\label{sec:counting:loops}
For a counting loop, we need a counter |\count0|, an incrementor |\advance|, a threshold |3| and a check |\ifnum\count=10| if the threshold has been reached.


\begin{codeexample}[]
\long\def\countingloop#1 in #2:#3#4{%
	#1=#2 %
	\loopcounter{#1}{#3}{#4}%
}

\long\def\loopcounter#1#2#3{%
	#3%
	\ifnum#1=#2 %
	\else%
		\advance#1 by1 %
		\loopcounter{#1}{#2}{#3}%
	\fi%
}
\countingloop{\count0} in 0:{3}{%
	The current value is `\the\count0'\par
}
\end{codeexample}

There are some subtleties to the above example:
\begin{itemize}
 \item We put a lot of \% in the example. Why? Note that whenever \TeX{} scans for a number -- e.\,g. as in the case of |#1=#2| -- it will continue scanning token by token, that is digit by digit, till he is sure that the number has ended, even over white space, and even expanding macros in case they themselves might not represent numbers again. Hence, \% tells \TeX{} to stop scanning. It is generally good practice to place \% to tell \TeX{} to stop scanning for more digits. However, there are some exceptions to it as well: In case of |\advance#1 by1| one should keep a white space in between, as well as in the case of |\ifnum#1=#2|.
 \item We placed the threshold |3| in |\countingloop{\count0} in 0:{3}| in curly brackets. Why? \TeX{} otherwise will recognize only the token |1| if a threshold of e\,g. |10| is given and stumble over the now remnant `extra' argument |0|. That is because a single letter represents a token to \TeX{}. Hence, two letters are two tokens and -- ungrouped -- become two arguments. Here, we have to group the threshold to make clear what we mean.
 \item One last thing becomes clear first when debugging is activated: As loops are done by recursion, i.\,e. by expansion followed by expansion till some threshold is reached, we will end with a lot of |\fi|s in the above case. If we place |\tracingmacros=2 \tracingcommands=2| before the |\countingloop| call and inspect the log file this will become apparent. This is bad because \TeX{} will keep a stack frame open for each |\if|\ldots|\fi| sequence. If we now have a loop over 10.000 items \ldots
 \item It is not good practice to use one of the system counters, here |\count0|, because one can never be sure that is not used for something else or changed somewhere else. E.\,g. when the page is full, \TeX{} will interrupt the current sequence of tokens to deal with creating a new page and finishing the old one, in this course changing |\count0|. Hence, we should also create our own counter.
\end{itemize}

Hence, we modify the example as follows:
\begin{codeexample}[]
\long\def\countingloop#1 in #2:#3#4{%
	#1=#2 %
	\loopcounter{#1}{#3}{#4}%
}

\long\def\loopcounter#1#2#3{%
	#3%
	\ifnum#1=#2 %
		\let\next=\relax%
	\else
		\advance#1 by1 %
		\def\next{\loopcounter{#1}{#2}{#3}}%
	\fi
	\next
}
\newcount\ourcounter

\countingloop{\ourcounter} in 0:{3}{%
	The current value is `\the\ourcounter'\par
}
\end{codeexample}

Principally, nothing has changed in terms of the output. However, notice that we have introduced the macro |\next| which either recurses into the next level -- but after the |\fi| statement has been given -- or ends the recursion by simply containing |\relax|. Also, we have declared a new counter called |\ourcounter| that is safe from harm.

Finally, let us briefly summarize what happens in detail:
\begin{enumerate}
 \item |\countingloop|\ldots is expanded to an assignment |#1=#2| and another macro |\loopcounter|\ldots.
 \item The assignment is done: |\ourcounter| is set to the starting value |0|.
 \item\label{countingloop:loopstart} The actual loop macro is expanded to the command block -- printing the current value -- and an if statement.
 \item The current value is printed.
 \item |\ourcounter| is compared to the threshold |3| and \ldots
 \begin{itemize}
	 \item False, i.\,e. the if statement is expanded to an |\advance| statement followed by defining |\next| to be another call of the same macro loop.
	 \item True, i.\,e. |\next| is set to be just |\relax|.
 \end{itemize}
 \item The statement is still false: |\advance| will increase |\ourcounter| by one, it is now |1|. |\next| is set to the loop macro.
 \item The loop macro is again expanded, go to step~\ref{countingloop:loopstart}. |\ourcounter| is \ldots |2| \ldots |\ourcounter| is |3|.
 \item Now the statement is true: |\next| is expanded to |\relax| and nothing happens.
\end{enumerate}

\subsubsection{Loops over list of items}
\label{sec:loops:over:list:of:items}
Looping over a list of items is very similar, only we will need |\ifx| in place of |\ifnum| and we need some end marker instead of the threshold value. However, how do we specify the list itself? Let's make some comma-separated list, e.\,g. |{a,b,c,d}| and call the end marker |\listingloopENDMARKER|.

\begin{codeexample}[]
\def\listingloopENDMARKER{\par \listingloopENDMARKER}
\long\def\listingloop#1in#2#3{%
	\looppicker{#1}{#3}#2,\listingloopENDMARKER,%
}%
\long\def\looppicker#1#2#3,{%
	\def\tempitem{#3}%
	\ifx\tempitem\listingloopENDMARKER
		\let\next=\relax%
	\else
		\def#1{#3}%
		#2%
		\def\next{\looppicker{#1}{#2}}%
	\fi
	\next
}%
\listingloop\x in{a,b,c,,d,e}{%
	The current item is `\x'
}
\end{codeexample}

Again, we make clear the subtleties contained therein:
\begin{itemize}
 \item We have defined |\listingloopENDMARKER| to replace itself. This is possible because |\ifx| will only compare first-level expansion, see Section~\ref{sec:branching}.
 \item We seem to miss a white space in \ldots|#1in#2|\ldots. However, tokens are always ending with an additional white space as |\xin| is not equal to |\x in|. Hence, none is needed here and more than one white space would probably get gobbled.
 \item The definition |\looppicker#1#2#3,|\ldots has three arguments but the recursive call |\looppicker{#1}{#2}| only gives two arguments!? This is the actual magic making this type of list possible! \TeX{} is actually scanning beyond the scope of the given token to obtain the third argument. In effect, we are biting off piece by piece, list item by list item off the given list. All because we have stated an additional |,| -- comma being the item separator -- in the definition of the |\looppicker| macro. The expansion of the loop macro will always pick up one more item from the list concatenated to its end until it has reached the |\ENDMARKER|. This is added to the list's very end on the loop's start, and there it stops.
\end{itemize}

\subsection{More On \TeX}
This document is far from complete. I recommend reading about conditional expressions in \cite{schwartz} (German, online version) or \cite{texbook} (bounded book). Hints about loops can be found in the manual of \PGFPlots, \cite{pgfplots} and the manual of \PGF, \cite{tikz}. Moreover, \PGFPlots\ and \PGF\ come with a whole lot of utility functions which are documented in the source |.code.tex| files.

%\subsection{Conditional Expressions}

%\subsection{Loops}

%\subsection{The Problem of Macro--Append Runtime}

%\section{Utility Functions of \PGF}

%\section{Utility Function of \PGFPlots}

\section{Special Tricks}

\subsection{Handling \# in Arguments}
More than once, I encountered the following difficulty: I wanted to collect an argument which contains the hash sign, `|#|'. That's not particularly difficult, but it can lead to a lot of strange error messages when the resulting argument shall be processed! Consider 
\begin{codeexample}[code only]
\def\collectargument#1{%
	\def\collectedcontent{#1}%
	\ifx\collectedcontent\empty
		It is empty.
	\else
		It is not empty, executing it: #1.
	\fi
}%

\collectargument{}% works

\collectargument{something}% works

\collectargument{% does not work!
	\def\something#1{which depends on #1}
}%
\end{codeexample}
The code in this example is relatively simple: the |\collectargument| macro expects one argument and checks if it is empty (using |\ifx|, which is a common and reliable check for emptiness). It is is not empty, it executes it. The |\collectargument| macro works in most circumstances. More precisely: it works as long as there is \emph{no} hash sign in its argument! In our example, the third call fails with ``Illegal parameter number in definition of |\collectedcontent|.'' which occurs during the |\def\collectedcontent{#1}| line (and \TeX\ has reasons for this message due to the special meaning of the parameter expansion).

The cure: redefine the |\collectargument| macro using
\begin{codeexample}[code only]
\def\collectargument#1{%
	\toks0={#1}%
	\edef\collectedcontent{\the\toks0}%
	\ifx\collectedcontent\empty
		It is empty.
	\else
		It is not empty, executing it: #1.
	\fi
}%

\end{codeexample}
\noindent (you may want to allocate a temporary token register for this task). What is the difference? Well, the |\toks0={#1}| assignment introduces no special meaning for the hash sign |#|, and |\the\toks0| neither. Note, however, that this requires |\edef\collectedcontent| instead of |\def\collectedcontent| since the |\the| statement needs to be expanded. Everything works as expected.

\printindex

\bibliographystyle{abbrv} %gerapali} %gerabbrv} %gerunsrt.bst} %gerabbrv}% gerplain}
\nocite{schwartz}
\nocite{pgfplots}
\bibliography{pgfplots}
\end{document}
