%%%%%%%%%%%%%%%%%%%%%%%%%%%%%%%%%%%%%%%%%%%%%%%%%%%%%%%%%%%%%%%%%%%%%%%%%%%%%
%
% This is a helper package with an elementary array datastructure,
% featuring O(1) index access and O(N) creation, deletion, copy.
%
% The following macros are supplied:
%
% \pgfplotsarraynewempty
% \pgfplotsarraynew
% \pgfplotsarrayresize
% \pgfplotsarraycopy
% \pgfplotsarraypushback
% \pgfplotsarraysize
% \pgfplotsarrayselect
% \pgfplotsarrayset
% \pgfplotsarrayletentry
% \pgfplotsarraycheckempty
% \pgfplotsarrayforeach
% \pgfplotsarraysort
%
% and a subset also for global arrays:
% \pgfplotsarraynewemptyglobal
% \pgfplotsarrayresizeglobal
% \pgfplotsarraysetglobal
% \pgfplotsarrayletentryglobal
%
% Copyright 2007/2008 by Christian Feuersänger.
%
% This program is free software: you can redistribute it and/or modify
% it under the terms of the GNU General Public License as published by
% the Free Software Foundation, either version 3 of the License, or
% (at your option) any later version.
% 
% This program is distributed in the hope that it will be useful,
% but WITHOUT ANY WARRANTY; without even the implied warranty of
% MERCHANTABILITY or FITNESS FOR A PARTICULAR PURPOSE.  See the
% GNU General Public License for more details.
% 
% You should have received a copy of the GNU General Public License
% along with this program.  If not, see <http://www.gnu.org/licenses/>.
%
%
%
%%%%%%%%%%%%%%%%%%%%%%%%%%%%%%%%%%%%%%%%%%%%%%%%%%%%%%%%%%%%%%%%%%%%%%%%%%%%%

\newcount\c@pgfplotsarray@tmp
\newif\ifpgfplotsarrayempty

% Creates a new, empty array.
\def\pgfplotsarraynewempty#1{%
	\pgfplotsarray@@def{#1@size}{0}%
}
\def\pgfplotsarraynewemptyglobal#1{%
	\expandafter\gdef\csname\string#1@size\endcsname{0}%
}

% resizes (truncates) array #1 to size #2
%
% the elements won't be initialised. Use 'set' for each element.
\def\pgfplotsarrayresize#1#2{%
	\c@pgfplotsarray@tmp=#2
	\pgfplotsarray@@edef{#1@size}{\the\c@pgfplotsarray@tmp}%
}
\def\pgfplotsarrayresizeglobal#1#2{%
	\c@pgfplotsarray@tmp=#2
	\expandafter\xdef\csname\string#1@size\endcsname{\the\c@pgfplotsarray@tmp}%
}

% Invokes code '#2' if the array named '#1' exists and '#3' if it does
% not exist.
\def\pgfplotsarrayifdefined#1#2#3{%
	\pgfutil@ifundefined{#1@size}{#3}{#2}%
}%

% Creates a new array with an abirtrary number of elements.
% Arguments:
% #1: the array's name (a macro name)
% #2: the elements in one of the following forms
% 	a) a '\\' terminated list:
%     first\\second\\third\\ ...\\
%     like in tabular with one column. The final '\\' necessary!
%   b) a comma-separated list:
%     first,second,third,5,10,last
%     ATTENTION: arrays do not support the '...' syntax of tikz!
%
% Example:
% \pgfplotsarraynew\fooarray{First Element\\Second Element\\Third Element\\}
% \pgfplotsarraynew\fooarray{First Element,Second Element,Third Element}
% \pgfplotsarraynew\fooarray{0,1,2,...,10}
%
%
\long\def\pgfplotsarraynew#1#2{%
	\pgfplots@check@backwards@compatible@list@format #2\\\pgfplots@EOI
	\ifpgfplots@is@old@list@format
		\pgfplotsarraynew@backslash{#1}{#2}%
	\else
		\pgfplotsarraynew@commasep{#1}{#2}%
	\fi
}

% #1 the array's name (a macro name
% #2 a list created by \pgfplotslistnew 
\def\pgfplotsarrayfrompgfplotslist#1#2{%
	\pgfplotsarraynewempty#1%
	\pgfplotslistforeachungrouped#2\as\pgfplotsarray@TMPa{%
		\expandafter\pgfplotsarraypushback\expandafter{\pgfplotsarray@TMPa}\to#1%
	}%
}%

\def\pgfplotsarraynew@backslash#1#2{%
	\pgfplotsarraynewempty{#1}%
	\long\def\pgfplotsarraynew@impl@rest{#2}%
	\pgfutil@loop
	\ifx\pgfplotsarraynew@impl@rest\pgfutil@empty
		\pgfplots@loop@CONTINUEfalse
	\else
		\pgfplots@loop@CONTINUEtrue
	\fi
	\ifpgfplots@loop@CONTINUE
		\expandafter\pgfplotsarraynew@impl\pgfplotsarraynew@impl@rest\toarray#1\relax
	\pgfutil@repeat
}%

% converts a comma-separated list (PGF foreach)  to my internal list
% structure.
\long\def\pgfplotsarraynew@commasep#1#2{%
	\pgfplotsarraynewempty{#1}%
	\pgfplotsutilforeachcommasep{#2}\as\pgfplotsarraynew@elem{%
		\expandafter\pgfplotsarraypushback\expandafter{\pgfplotsarraynew@elem}\to#1%
	}%
}
% helper macro for \pgfplotsarraynew
\long\def\pgfplotsarraynew@impl#1\\#2\toarray#3{%
	\pgfplotsarraypushback#1\to#3\relax%
	\def\pgfplotsarraynew@impl@rest{#2}%
}


\def\pgfplotsarray@@let#1=#2{%
	\def\pgfplotsarray@TMP@@{\expandafter\let\csname\string#1\endcsname}%
	\expandafter\pgfplotsarray@TMP@@\csname\string#2\endcsname
}%

\def\pgfplotsarray@@#1#2{%
	\expandafter#1\csname\string#2\endcsname
}%
\def\pgfplotsarray@@def#1{%
	\expandafter\def\csname\string#1\endcsname
}%
\def\pgfplotsarray@@edef#1{%
	\expandafter\edef\csname\string#1\endcsname
}%

\def\pgfplotsarray@@getsizeto#1#2{%
	\pgfplotsarraysize{#1}\to\c@pgfplotsarray@tmp
	\edef#2{\the\c@pgfplotsarray@tmp}%
}%

% Copies array #1 to array #2.
\def\pgfplotsarraycopy#1\to#2{%
	\pgfplotsarray@@let{#2@size}={#1@size}%
	\pgfplotsarray@@getsizeto{#1}{\pgfplotsarray@TMP}%
	\c@pgfplotsarray@tmp=0
	\pgfutil@loop
	\ifnum\c@pgfplotsarray@tmp<\pgfplotsarray@TMP
	\pgfplotsarray@@let{#2@\the\c@pgfplotsarray@tmp}={#1@\the\c@pgfplotsarray@tmp}%
	\advance\c@pgfplotsarray@tmp by1
	\pgfutil@repeat
}



% #1: the item to append
% #2: the array as macro name
% Example:
% \pgfplotsarraypushback Next last element\to\fooarray
\long\def\pgfplotsarraypushback#1\to#2{%
	\pgfplotsarraysize{#2}\to\c@pgfplotsarray@tmp
	\pgfplotsarray@@def{#2@\the\c@pgfplotsarray@tmp}{#1}%
	\advance\c@pgfplotsarray@tmp by1
	\pgfplotsarray@@edef{#2@size}{\the\c@pgfplotsarray@tmp}%
}
\long\def\pgfplotsarraypushbackglobal#1\to#2{%
	\pgfplotsarraysize{#2}\to\c@pgfplotsarray@tmp
	\pgfplotsarray@@\gdef{#2@\the\c@pgfplotsarray@tmp}{#1}%
	\advance\c@pgfplotsarray@tmp by1
	\pgfplotsarray@@\xdef{#2@size}{\the\c@pgfplotsarray@tmp}%
}

\long\def\pgfplotsarraypushbackglobal#1\to#2{%
	\pgfplotsarraysize{#2}\to\c@pgfplotsarray@tmp
	\pgfplotsarray@@\gdef{#2@\the\c@pgfplotsarray@tmp}{#1}%
	\advance\c@pgfplotsarray@tmp by1
	\pgfplotsarray@@\xdef{#2@size}{\the\c@pgfplotsarray@tmp}%
}

% Counts the number of elements in array #1, storing it into the count
% register #2.
% Example:
% \pgfplotsarraysize\foo\to{\count0}%
% \the\count0
\long\def\pgfplotsarraysize#1\to#2{%
	#2=\csname\string#1@size\endcsname\relax
}
\long\def\pgfplotsarraysizetomacro#1\to#2{%
	\expandafter\let\expandafter#2\csname\string#1@size\endcsname
}

% expands to the size of array #1.
\def\pgfplotsarraysizeof#1{\csname\string#1@size\endcsname}%


% Returns the #1th element of array #2 into macro #3
% Arguments:
% #1: a count 0,...,N-1 where N is the array size.
%     You may specify a number of a count.
% #2: a array
% #3: a macro name
% Example:
%   Element 0:
%   \pgfplotsarrayselect0\of\foo\to\elem
%   \elem
%   Element \count1:
%   \pgfplotsarrayselect\count1\of\foo\to\elem
\def\pgfplotsarrayselect#1\of#2\to#3{%
	\c@pgfplotsarray@tmp=#1\relax
	\expandafter\let\expandafter#3\csname\string#2@\the\c@pgfplotsarray@tmp\endcsname%
	\ifx#3\relax
		\pgfplotsthrow{no such element}{#1}{No such element: \string\pgfplotsarrayselect\the\c@pgfplotsarray@tmp\string\of{\string#2}}\pgfeov%
	\fi
}

% Expands to the value at index #1 of array #2.
% #1: a number (not a register)
% #2: an array
\def\pgfplotsarrayvalueofelem#1\of#2{\csname\string#2@#1\endcsname}%

% Sets element '#1' of array '#2' to '#3'.
\def\pgfplotsarrayset#1\of#2\to#3{%
	\c@pgfplotsarray@tmp=#1\relax
	\pgfutil@namedef{\string#2@\the\c@pgfplotsarray@tmp}{#3}%
}
\def\pgfplotsarraysetglobal#1\of#2\to#3{%
	\c@pgfplotsarray@tmp=#1\relax
	\expandafter\gdef\csname\string#2@\the\c@pgfplotsarray@tmp\endcsname{#3}%
}
\long\def\pgfplotsarrayletentry#1\of#2=#3{%
	\c@pgfplotsarray@tmp=#1\relax
	\expandafter\let\csname\string#2@\the\c@pgfplotsarray@tmp\endcsname=#3\relax
}
\long\def\pgfplotsarrayletentryglobal#1\of#2=#3{%
	\c@pgfplotsarray@tmp=#1\relax
	\expandafter\global\expandafter\let\csname\string#2@\the\c@pgfplotsarray@tmp\endcsname=#3\relax
}

% Defines \pgfplotsretval to be a text-representation of the array.
% It will contain \t to separate cells.
%
\def\pgfplotsarraytotext#1{%
	\begingroup
	\pgfplotsapplistXnewempty\pgfplotsretval@
	\pgfplotsarrayforeachungrouped#1\as\entry{%
		\pgfmathfloatparsenumber\entry
		\pgfmathfloattosci\pgfmathresult
		\edef\entry{\pgfmathresult\noexpand\t}%
		\expandafter\pgfplotsapplistXpushback\entry\to\pgfplotsretval@
	}%
	\pgfplotsapplistXlet\pgfplotsretval=\pgfplotsretval@
	\pgfmath@smuggleone\pgfplotsretval
	\endgroup
}

% Sets the boolean \ifpgfplotsarrayempty depending on whether array #1 is empty
% or not.
% Example:
%
% \pgfplotsarraycheckempty\fooarray
% \ifpgfplotsarrayempty
%    List fooarray is empty!
% \else
%    List is not empty.
% \fi
\def\pgfplotsarraycheckempty#1{%
	\ifnum\csname\string#1@size\endcsname=0
		\pgfplotsarrayemptytrue
	\else
		\pgfplotsarrayemptyfalse
	\fi
}


% Iterates through each array element, names it #2 and calls code #3.
% Example:
% \pgfplotsarraynew\fooarray{Eins\\Zwei\\Drei\\}%
% \pgfplotsarrayforeach\fooarray\as\foo{Element \foo\par}%
% results in
%  Element Eins
%  Element Zwei
%  Element Drei
% Each single element will be grouped with TeX groups.
%
% During the loop, \pgfplotsarrayforeachindex expands to the current index.
\long\def\pgfplotsarrayforeach#1\as#2#3{%
	\pgfplotsarray@@getsizeto{#1}{\pgfplotsarray@TMP}%
	\c@pgfplotsarray@tmp=0\relax
	\def\pgfplotsarrayforeachindex{\the\c@pgfplotsarray@tmp}%
	\long\def\pgfplotsarray@TMPb{#3}%
	\pgfutil@loop
		\ifnum\c@pgfplotsarray@tmp<\pgfplotsarray@TMP\relax
		\begingroup
		\expandafter\let\expandafter#2\expandafter=\csname\string#1@\the\c@pgfplotsarray@tmp\endcsname%
		\pgfplotsarray@TMPb
		\endgroup
		\advance\c@pgfplotsarray@tmp by1
	\pgfutil@repeat
	\let\pgfplotsarrayforeachindex=\relax%
}

% The same but without groups around #3.
\long\def\pgfplotsarrayforeachungrouped#1\as#2#3{%
	\pgfplotsarray@@getsizeto{#1}{\pgfplotsarray@TMP}%
	\c@pgfplotsarray@tmp=0\relax
	\def\pgfplotsarrayforeachindex{\the\c@pgfplotsarray@tmp}%
	\pgfutil@loop
		\ifnum\c@pgfplotsarray@tmp<\pgfplotsarray@TMP\relax
		\expandafter\let\expandafter#2\expandafter=\csname\string#1@\the\c@pgfplotsarray@tmp\endcsname%
		\begingroup
		\toks0={#3}%
		\xdef\pgfplotsarray@glob@TMP{%
			\the\toks0 %
			% restore loop state:
			\c@pgfplotsarray@tmp=\the\c@pgfplotsarray@tmp\space
			\noexpand\def\noexpand\pgfplotsarray@TMP{\pgfplotsarray@TMP}%
		}%
		\endgroup
		\pgfplotsarray@glob@TMP
		\advance\c@pgfplotsarray@tmp by1
	\pgfutil@repeat
	\let\pgfplotsarrayforeachindex=\relax%
}

\long\def\pgfplotsarrayforeachreversed#1\as#2#3{%
	\pgfplotsarray@@getsizeto{#1}{\pgfplotsarray@TMP}%
	\c@pgfplotsarray@tmp=\pgfplotsarray@TMP\relax
	\def\pgfplotsarrayforeachindex{\the\c@pgfplotsarray@tmp}%
	\long\def\pgfplotsarray@TMPb{#3}%
	\pgfutil@loop
		\ifnum\c@pgfplotsarray@tmp>0 %
		\advance\c@pgfplotsarray@tmp by-1
		\begingroup
		\expandafter\let\expandafter#2\expandafter=\csname\string#1@\the\c@pgfplotsarray@tmp\endcsname%
		\pgfplotsarray@TMPb
		\endgroup
	\pgfutil@repeat
	\let\pgfplotsarrayforeachindex=\relax%
}

\long\def\pgfplotsarrayforeachreversedungrouped#1\as#2#3{%
	\pgfplotsarray@@getsizeto{#1}{\pgfplotsarray@TMP}%
	\c@pgfplotsarray@tmp=\pgfplotsarray@TMP\relax
	\def\pgfplotsarrayforeachindex{\the\c@pgfplotsarray@tmp}%
	\long\def\pgfplotsarray@TMPb{#3}%
	\pgfutil@loop
		\ifnum\c@pgfplotsarray@tmp>0 %
		\advance\c@pgfplotsarray@tmp by-1
		\expandafter\let\expandafter#2\expandafter=\csname\string#1@\the\c@pgfplotsarray@tmp\endcsname%
		\pgfplotsarray@TMPb
	\pgfutil@repeat
	\let\pgfplotsarrayforeachindex=\relax%
}

\pgfkeys{%
	/pgfplots/array/unscope pre/.code={%
		% That's more or less efficient (although some runtime factors
		% could be saved).
		\c@pgfplotsarraysort@i=0
		\c@pgfplotsarray@tmp=\pgfplotsarraysizeof{#1}\relax
		\pgfutil@loop
		\ifnum\c@pgfplotsarraysort@i<\c@pgfplotsarray@tmp
			\pgfplotsarrayselect{\c@pgfplotsarraysort@i}\of{#1}\to\pgfplots@loc@TMPa
			\pgfplotsarrayletentryglobal\c@pgfplotsarraysort@i\of{pgfparraytmp}=\pgfplots@loc@TMPa
			\advance\c@pgfplotsarraysort@i by1
		\pgfutil@repeat
	},%
	/pgfplots/array/unscope post/.code={%
		% copy from global temp -> #1
		\c@pgfplotsarray@tmp=0
		\pgfplotsarraysizetomacro{#1}\to\pgfplots@loc@TMPb
		\pgfutil@loop
		\ifnum\c@pgfplotsarray@tmp<\pgfplots@loc@TMPb
			\pgfplotsarrayselect{\c@pgfplotsarray@tmp}\of{pgfparraytmp}\to\pgfplots@loc@TMPa
			\pgfplotsarrayletentry\c@pgfplotsarray@tmp\of{#1}=\pgfplots@loc@TMPa
			\advance\c@pgfplotsarray@tmp by1
		\pgfutil@repeat
	},%
}%

% Sorts array '#1' using N log N time.
% 
% The sort key is provided as argument in
% /pgfplots/iflessthan/.code args={#1#2#3#4}, see docs in
% pgfplotsutil.code.tex.
%
% Remarks:
% - the sorting algorithm is a sub-optimal mergesort currently -
% but it has runtime N log N. 
% - As usual, it is very difficult to move the final result out of the
%   current scope. The initial configuration allocates a *global* copy of #1 
%   to carry results out of one TeX group. This means that memory will NEVER be freed.
%   I don't see a general N log N alternative for doing so (short of discarding
%   the scopes for local groups).
%   BUT:
%   you can redefine the code key pair 
%     |/pgfplots/array/unscope pre|
%   and
%     |/pgfplots/array/unscope post|
%   to define your own routine which may benefit from special needs of
%   your application. The |... pre| command is invoked JUST before the
%   \endgroup and the |... post| JUST after the |\endgroup|.
%
%
% EXAMPLE:
%--------------------------------------------------
% \pgfplotsarraynewempty\testarray
% \pgfplotsarraypushback503\to\testarray
% \pgfplotsarraypushback087\to\testarray
% \pgfplotsarraypushback512\to\testarray
% \pgfplotsarraypushback061\to\testarray
% \pgfplotsarraypushback908\to\testarray
% \pgfplotsarraypushback170\to\testarray
% \pgfplotsarraypushback897\to\testarray
% \pgfplotsarraypushback275\to\testarray
% \pgfplotsarraypushback653\to\testarray
% \pgfplotsarraypushback426\to\testarray
% \pgfplotsarraypushback154\to\testarray
% \pgfplotsarraypushback509\to\testarray
% \pgfplotsarraypushback612\to\testarray
% \pgfplotsarraypushback677\to\testarray
% \pgfplotsarraypushback765\to\testarray
% \pgfplotsarraypushback703\to\testarray
% 
% Unsorted: 
% 
% [\pgfplotsarrayforeach\testarray\as\elem{\elem\space}]
% 
% \pgfplotsarraysort\testarray
% 
% sorted: 
% 
% [\pgfplotsarrayforeach\testarray\as\elem{\elem\space}]
% 
%-------------------------------------------------- 
%
% @see \pgfplotsutilsortthree
% @see \pgfplotsutilsortfour
\def\pgfplotsarraysort{%
	\pgfutil@ifnextchar[{\pgfplotsarraysort@opt}{\pgfplotsarraysort@opt[]}%
}%
\def\pgfplotsarraysort@opt[#1]#2{%
	% I admit, the implementation is more-or less copy-paste of an
	% experiment I had done at university. It does its job, but it is
	% certainly not optimal (that means: it is slow).
	\begingroup
	\pgfqkeys{/pgfplots/array}{#1}%
	\pgfkeysifdefined{/pgfplots/array/iflessthan/.@cmd}{%
		\pgfplots@warning{Warning: /pgfplots/array/iflessthan/.@cmd is deprecated. Please use /pgfplots/iflessthan/.@cmd instead.}%
		\pgfkeysgetvalue{/pgfplots/array/iflessthan/.@cmd}\pgfplotsarraysort@iflt
	}{%
		\pgfkeysgetvalue{/pgfplots/iflessthan/.@cmd}\pgfplotsarraysort@iflt
	}%
	\pgfplotsarraysize{#2}\to\c@pgfplotsarray@tmp
	\pgfplotsarrayresizeglobal{pgfparraytmp}{\the\c@pgfplotsarray@tmp}%
	\countdef\c@pgfplotsarraysort@m=0
	\countdef\c@pgfplotsarraysort@k=1
	\countdef\c@pgfplotsarraysort@i=2
	\countdef\c@pgfplotsarraysort@l=3
	\countdef\c@pgfplotsarraysort@j=4
	\countdef\c@pgfplotsarraysort@q=5
	\def\pgfplotsarray@mergesort@{\pgfplotsarray@mergesort{#2}}%
	\def\pgfplotsarray@mergesort@recurse@{\pgfplotsarray@mergesort@recurse{#2}}%
	\edef\pgfplotsarray@mergesort@@{{0}{\the\c@pgfplotsarray@tmp}}%
	\expandafter\pgfplotsarray@mergesort@\pgfplotsarray@mergesort@@
	%
	% copy the complete, sorted result outside of the current scope.
	\pgfkeysvalueof{/pgfplots/array/unscope pre/.@cmd}#2\pgfeov
	\endgroup
	\pgfkeysvalueof{/pgfplots/array/unscope post/.@cmd}#2\pgfeov
}%

% #1: array name,
% #2: start offset (as string number)
% #3: number of elems (as string number)
\def\pgfplotsarray@mergesort#1#2#3{%
	\ifnum#3<2
		% ready!
	\else
		\c@pgfplotsarraysort@m=#3\relax
		\divide\c@pgfplotsarraysort@m by2
		\c@pgfplotsarraysort@q=#3\relax
		\advance\c@pgfplotsarraysort@q by-\c@pgfplotsarraysort@m
		\c@pgfplotsarraysort@i=\c@pgfplotsarraysort@m
		\advance\c@pgfplotsarraysort@i by#2\relax
		\edef\pgfplotsarray@mergesort@@{%
			{#2}{\the\c@pgfplotsarraysort@m}% start1, N_1
			{\the\c@pgfplotsarraysort@i}{\the\c@pgfplotsarraysort@q}% start2, N_2
		}%
		\expandafter\pgfplotsarray@mergesort@recurse@\pgfplotsarray@mergesort@@
		%
		% Merge:
		% restore 'm':
		\c@pgfplotsarraysort@m=#3\relax
		\divide\c@pgfplotsarraysort@m by2
		\c@pgfplotsarraysort@q=#3
		\advance\c@pgfplotsarraysort@q by-\c@pgfplotsarraysort@m
		\c@pgfplotsarraysort@k=0
		\c@pgfplotsarraysort@l=0
		\c@pgfplotsarraysort@i=0
		%
		% direct comparisons for merge:
		\pgfutil@loop
		\pgfplots@loop@CONTINUEtrue
		\ifnum\c@pgfplotsarraysort@k<\c@pgfplotsarraysort@m
			\ifnum\c@pgfplotsarraysort@l<\c@pgfplotsarraysort@q
			\else
				\pgfplots@loop@CONTINUEfalse
			\fi
		\else
			\pgfplots@loop@CONTINUEfalse
		\fi
		\ifpgfplots@loop@CONTINUE
			\c@pgfplotsarraysort@j=\c@pgfplotsarraysort@m
			\advance\c@pgfplotsarraysort@j by\c@pgfplotsarraysort@l
			\advance\c@pgfplotsarraysort@j by#2
			\pgfplotsarrayselect{\c@pgfplotsarraysort@j}\of{#1}\to\pgfplotsarray@x@mpl
			\c@pgfplotsarraysort@j=\c@pgfplotsarraysort@k
			\advance\c@pgfplotsarraysort@j by#2
			\pgfplotsarrayselect{\c@pgfplotsarraysort@j}\of{#1}\to\pgfplotsarray@x@k
			\begingroup
			\pgfplotsarraysort@iflt
				{\pgfplotsarray@x@mpl}%
				{\pgfplotsarray@x@k}%
				{% x[m+l] < x[k]
					\aftergroup\pgfplots@loc@tmptrue
				}%
				{% x[k] <= x[m+l]
					\aftergroup\pgfplots@loc@tmpfalse
				}%
				\pgfeov
			\endgroup
			\ifpgfplots@loc@tmp
				\pgfplotsarrayletentryglobal\c@pgfplotsarraysort@i\of{pgfparraytmp}=\pgfplotsarray@x@mpl
				\advance\c@pgfplotsarraysort@i by1
				\advance\c@pgfplotsarraysort@l by1
			\else
				\pgfplotsarrayletentryglobal\c@pgfplotsarraysort@i\of{pgfparraytmp}=\pgfplotsarray@x@k
				\advance\c@pgfplotsarraysort@i by1
				\advance\c@pgfplotsarraysort@k by1
			\fi
		\pgfutil@repeat
		%
		% append:
		\pgfutil@loop
		\ifnum\c@pgfplotsarraysort@k<\c@pgfplotsarraysort@m
			\c@pgfplotsarraysort@j=\c@pgfplotsarraysort@k
			\advance\c@pgfplotsarraysort@j by#2
			\pgfplotsarrayselect{\c@pgfplotsarraysort@j}\of{#1}\to\pgfplots@loc@TMPa
			\pgfplotsarrayletentryglobal\c@pgfplotsarraysort@i\of{pgfparraytmp}=\pgfplots@loc@TMPa
			\advance\c@pgfplotsarraysort@i by1
			\advance\c@pgfplotsarraysort@k by1
		\pgfutil@repeat
		\pgfutil@loop
		\ifnum\c@pgfplotsarraysort@l<\c@pgfplotsarraysort@q
			\c@pgfplotsarraysort@j=\c@pgfplotsarraysort@m
			\advance\c@pgfplotsarraysort@j by\c@pgfplotsarraysort@l
			\advance\c@pgfplotsarraysort@j by#2
			\pgfplotsarrayselect{\c@pgfplotsarraysort@j}\of{#1}\to\pgfplots@loc@TMPa
			\pgfplotsarrayletentryglobal\c@pgfplotsarraysort@i\of{pgfparraytmp}=\pgfplots@loc@TMPa
			\advance\c@pgfplotsarraysort@i by1
			\advance\c@pgfplotsarraysort@l by1
		\pgfutil@repeat
		%
		% copy back from temporary to '#1'
		\c@pgfplotsarraysort@i=0
		\c@pgfplotsarraysort@k=#3
		\pgfutil@loop
		\ifnum\c@pgfplotsarraysort@i<\c@pgfplotsarraysort@k
			\c@pgfplotsarraysort@j=\c@pgfplotsarraysort@i
			\advance\c@pgfplotsarraysort@j by#2
			\pgfplotsarrayselect{\c@pgfplotsarraysort@i}\of{pgfparraytmp}\to\pgfplots@loc@TMPa
			\pgfplotsarrayletentry\c@pgfplotsarraysort@j\of{#1}=\pgfplots@loc@TMPa
			\advance\c@pgfplotsarraysort@i by1
		\pgfutil@repeat
%\pgfplotsarraysort@DEBUGTEST{#1}{after mergesort(start=#2,n=#3):}%
	\fi
}%
\long\def\pgfplotsarraysort@DEBUGTEST#1#2{%
	\par#2\par
	[\pgfplotsarrayforeach{#1}\as\elem{\elem\space}]\par%
}
% #1: array name
% #2: start1
% #3: N1
% #4: start2
% #5: N2
\def\pgfplotsarray@mergesort@recurse#1#2#3#4#5{%
	\pgfplotsarray@mergesort{#1}{#2}{#3}%
	\pgfplotsarray@mergesort{#1}{#4}{#5}%
}

	
% A simple, ungrouped insert sort algorithm for small arrays.
%
%	for i := 1; i < N; ++i do
%    begin
%        value := A[i];
%        j := i - 1;
%        done := false;
%        repeat
%            if A[j] > value then
%            begin
%                A[j + 1] := A[j];
%                j := j - 1;
%                if j < 0 then
%                    done := true;
%            end
%            else
%                done := true;
%        until done;
%        A[j + 1] := value;
%    end;
\def\pgfplotsarrayinsertionsort#1{%
	\pgfkeysgetvalue{/pgfplots/iflessthan/.@cmd}\pgfplotsarraysort@iflt
	\pgfplotsarraysizetomacro#1\to\pgfplotsarrayinsertionsort@N
	\c@pgf@countd=1
	\pgfplotsarrayinsertionsort@{#1}%
}%
\def\pgfplotsarrayinsertionsort@#1{%
	\ifnum\c@pgf@countd<\pgfplotsarrayinsertionsort@N\relax
		\pgfplotsarrayselect\c@pgf@countd\of#1\to\pgfplotsarrayinsertionsort@v
		%
		\c@pgf@countb=\c@pgf@countd
		\advance\c@pgf@countb by-1
		%
		\pgfplotsarrayinsertionsort@@{#1}%
		\advance\c@pgf@countb by1
		\pgfplotsarrayletentry\c@pgf@countb\of#1=\pgfplotsarrayinsertionsort@v
		%
		\advance\c@pgf@countd by1
		\expandafter\pgfplotsarrayinsertionsort@\expandafter#1%
	\fi
}
\def\pgfplotsarrayinsertionsort@@#1{%
	\pgfplotsarrayselect\c@pgf@countb\of#1\to\pgfplotsarrayinsertionsort@j
	\pgfplotsarraysort@iflt{\pgfplotsarrayinsertionsort@v}{\pgfplotsarrayinsertionsort@j}
	{%
		\advance\c@pgf@countb by1
		\pgfplotsarrayletentry\c@pgf@countb\of#1=\pgfplotsarrayinsertionsort@j
		\advance\c@pgf@countb by-2
		\ifnum\c@pgf@countb<0
			\let\pgfplotsarrayinsertionsort@@next\relax%
		\else
			\def\pgfplotsarrayinsertionsort@@next{\pgfplotsarrayinsertionsort@@#1}%
		\fi
	}{%
		\let\pgfplotsarrayinsertionsort@@next\relax%
	}\pgfeov%
	\pgfplotsarrayinsertionsort@@next
}%

