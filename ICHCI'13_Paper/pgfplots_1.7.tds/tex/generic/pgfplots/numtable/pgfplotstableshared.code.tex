%--------------------------------------------
%
% Package pgfplotstable
%
% Provides support to read and work with abstact numeric tables of the
% form
%
% COLUMN1	COLUMN2 COLUMN3
% 1 		2		3
% 4			4		552
% 1e124		0.00001	1.2345e-12
% ...
%
% Copyright 2007-2010 by Christian Feuersänger.
%
% This program is free software: you can redistribute it and/or modify
% it under the terms of the GNU General Public License as published by
% the Free Software Foundation, either version 3 of the License, or
% (at your option) any later version.
% 
% This program is distributed in the hope that it will be useful,
% but WITHOUT ANY WARRANTY; without even the implied warranty of
% MERCHANTABILITY or FITNESS FOR A PARTICULAR PURPOSE.  See the
% GNU General Public License for more details.
% 
% You should have received a copy of the GNU General Public License
% along with this program.  If not, see <http://www.gnu.org/licenses/>.
%
%--------------------------------------------

% This file provides a high-level table manipulation and typesetting
% package.
%
%
% ---------------------------------------------
%  Implementation notes for programmers:
% ---------------------------------------------
%
%  The table data structure consists of
%  1. A name which must be a valid TeX macro like '\table'
%  2. A column list in the \pgfplotslist format.
%   In fact, the column list is stored in the table's name:
%   	\pgfplotslistnewempty\table
%  3. A "file name" stored in
%     \csname\string<\namemacro>@@table@name\endcsname
%  4. A "scan line length" number stored in
%  		\csname\string<\namemacro>@@table@scanline\endcsname
%  		It contains the value of \pgfplotsscanlinelength which has
%  		been identified right after \pgfplotstableread.
%  5. foreach column, there is a list of row-values (a row-vector) in
%  the \pgfplotslist format named
%  		\csname\string<\namemacro>@<column name>\endcsname
%   

\newif\ifpgfplotstable@search@header
\newif\ifpgfplotstable@firstline@is@header
\newcount\c@pgfplotstable@counta
\newtoks\t@pgfplotstable@a
\newif\ifpgfplots@addplotimpl@readcompletely
% should always be false; use only in grouped internal macros
\newif\ifpgfplots@table@options@areset
\newif\ifpgfplots@tableread@to@listener
\newif\ifpgfplots@tableread@use@begingroup
\newif\ifpgfplotstable@trimcells

\pgfkeys{%
	/pgfplots/table/every table/.style={},
	/pgfplots/table/.unknown/.code={%
		\pgfplotstable@error@pkg{Sorry, I do not know the key `\pgfkeyscurrentkeyRAW' and I am going to ignore it. Perhaps you need \string\usepackage{pgfplotstable}? (The table typesetting parts are a separate package)}%
	},%
	/pgfplots/table/header/.is choice,
	/pgfplots/table/header/true/.code={\pgfplotstable@firstline@is@headerfalse\pgfplotstable@search@headertrue},
	/pgfplots/table/header/false/.code={\pgfplotstable@firstline@is@headerfalse\pgfplotstable@search@headerfalse},
	/pgfplots/table/header/has colnames/.code={\pgfplotstable@firstline@is@headertrue\pgfplotstable@search@headerfalse},
	/pgfplots/table/header=true,
	/pgfplots/table/x index/.initial=0,
	/pgfplots/table/x/.initial=,
	/pgfplots/table/x expr/.initial=,
	/pgfplots/table/y index/.initial=1,
	/pgfplots/table/y/.initial=,
	/pgfplots/table/y expr/.initial=,
	/pgfplots/table/z index/.initial=2,
	/pgfplots/table/z/.initial=,
	/pgfplots/table/z expr/.initial=,
	/pgfplots/table/meta index/.initial=,
	/pgfplots/table/meta/.initial=,
	/pgfplots/table/meta expr/.initial=,
	/pgfplots/table/x error index/.initial=,
	/pgfplots/table/y error index/.initial=,
	/pgfplots/table/z error index/.initial=,
	/pgfplots/table/x error/.initial=,
	/pgfplots/table/y error/.initial=,
	/pgfplots/table/z error/.initial=,
	/pgfplots/table/x error expr/.initial=,
	/pgfplots/table/y error expr/.initial=,
	/pgfplots/table/z error expr/.initial=,
	/pgfplots/table/ignore chars/.initial=,
	/pgfplots/table/white space chars/.initial=,
	/pgfplots/table/comment chars/.initial=,
	/pgfplots/table/skip first n/.initial=0,
	/pgfplots/table/trim cells/.is if=pgfplotstable@trimcells,
	/pgfplots/table/trim cells/.default=true,
	/pgfplots/table/read completely/.is choice,
	/pgfplots/table/read completely/true/.code=	\pgfplots@addplotimpl@readcompletelytrue\def\pgfplots@addplotimpl@readcompletely@auto{0},
	/pgfplots/table/read completely/false/.code=\pgfplots@addplotimpl@readcompletelyfalse\def\pgfplots@addplotimpl@readcompletely@auto{0},
	/pgfplots/table/read completely/auto/.code=\pgfplots@addplotimpl@readcompletelyfalse\def\pgfplots@addplotimpl@readcompletely@auto{1},
	/pgfplots/table/read completely/.default=true,
	/pgfplots/table/read completely=auto,
	/pgfplots/table/col sep/.is choice,
	/pgfplots/table/col sep/space/.code		= {\def\pgfplotstableread@COLSEP@CASE{0}},
	/pgfplots/table/col sep/comma/.code		= {\def\pgfplotstableread@COLSEP@CASE{1}},
	/pgfplots/table/col sep/semicolon/.code	= {\def\pgfplotstableread@COLSEP@CASE{2}},
	/pgfplots/table/col sep/colon/.code		= {\def\pgfplotstableread@COLSEP@CASE{3}},
	/pgfplots/table/col sep/braces/.code	= {\def\pgfplotstableread@COLSEP@CASE{4}},
	/pgfplots/table/col sep/tab/.code		= {\def\pgfplotstableread@COLSEP@CASE{5}},
	/pgfplots/table/col sep/&/.code			= {\def\pgfplotstableread@COLSEP@CASE{6}\pgfplotstable@trimcellstrue},
	/pgfplots/table/col sep/ampersand/.code	= {\def\pgfplotstableread@COLSEP@CASE{6}\pgfplotstable@trimcellstrue},
	/pgfplots/table/col sep=space,
	/pgfplots/table/format/.is choice,
	/pgfplots/table/format/auto/.code		= {\def\pgfplotstableread@FORMAT@CASE{0}},
	/pgfplots/table/format/inline/.code		= {\def\pgfplotstableread@FORMAT@CASE{1}},
	/pgfplots/table/format/file/.code		= {\def\pgfplotstableread@FORMAT@CASE{2}},
	/pgfplots/table/format=auto,
	/pgfplots/table/row sep/.code={%
		\pgfplotsutilifstringequal{#1}{\\}{%
			\def\pgfplotstableread@ROWSEP@CASE{1}%
		}{%
			\pgfplotsutilifstringequal{#1}{crcr}{%
				\def\pgfplotstableread@ROWSEP@CASE{1}%
			}{%
				\pgfplotsutilifstringequal{#1}{newline}{%
					\def\pgfplotstableread@ROWSEP@CASE{0}%
				}{%
					{%
					\t@pgfplots@tokc={#1}%
					\pgfplotsthrow{invalid argument}{\pgfplots@loc@TMPa}{Sorry, the choice `row sep=\the\t@pgfplots@tokc' is not known. Maybe you misspelled it? Try `\string\\' or `newline'.}\pgfeov%
					}%
				}%
			}%
		}%
	},%
	/pgfplots/table/row sep=newline,
}

\pgfkeys{
	% #1: the argument which should have been assigned.
	% #2: an error message. 
	/pgfplots/exception/non unique colname/.code 2 args={%
		\ifx\pgfplotsexceptionmsg\relax
			\pgfplots@error{#2}%
		\else
			\pgfplots@error{\pgfplotsexceptionmsg}%
		\fi
		\let#1=\pgfutil@empty
	},%
}

% \pgfplotstableread[OPTIONS] {FILE} to \name
%
% This method reads a table from FILE to macro \name.
%
% FILE is something like
% G	Basis	dof	L2	A	Lmax	cgiter	maxlevel	eps
% 5	5	5	8.31160034e-02	0.00000000e+00	1.80007647e-01	2	2	-1
% 17	17	17	2.54685628e-02	0.00000000e+00	3.75580565e-02	5	3	-1
% ...
%
% A number format line is also understood:
% G	Basis	dof	L2	A	Lmax	cgiter	maxlevel	eps
% $flags int	int	int	sci:8	sci:8	sci:8	int	int	std:8
% 5	5	5	8.31160034e-02	0.00000000e+00	1.80007647e-01	2	2	-1
%
% or a three-column-gnuplot file with 2 comment headers like
% #Curve 0, 20 points
% #x y type
% 0.00000 0.00000 i
% 0.52632 0.50235 i
%
% The table data is stored columnwise in lists and can be accessed
% with the other methods of this package.
%
% \pgfplotstableread[<options>]{<file>}{<\macro>}
% \pgfplotstableread[<options>]{<file>} to listener{<\macro>}
% \pgfplotstableread* ...
%
% The '*' does not protect its local variables with TeX groups,
% everything is added to the current scope (useful to the 'to
% listener' thing.
%
% The 'to listener' variant does NOT assemble a table data structure.
% Instead, it processes the input table row-wise and invokes <\macro>
% after each complete row. During the evaluation of <\macro>, the
% following methods can be used to query values:
%   \pgfplotstablereadgetcolindex{<index>}{<\macro>}
%   \pgfplotstablereadgetcolname{<name>}{<\macro>}
%   \pgfplotstablereadvalueofcolname{<name>}
%   \pgfplotstablereadvalueofcolindex{<index>}
%   \thisrow{<name>} (equivalent to \pgfplotstablereadvalueofcolname)
%   \getthisrow{<name>}{<\macro>} (equivalent to \pgfplotstablereadgetcolname)
%   \thisrowno{<index>} (equivalent to \pgfplotstablereadvalueofcolindex)
%   \getthisrowno{<index>}{<\macro>}
%   \pgfplotstableforeachcolumn\as{<\iteratemacro>}{<loop body>}
%   	(the '\as' is required directly after
%   	\pgfplotstableforeachcolumn in this context)
% Attention: 'to listener' is scoped by TeX groups, so any assignments
% need to be done globally (or with aftergroup magic).
%
% More remarks about scoping:
% \pgfplotstableread to listener works as follows:
%     \begingroup
%     // load table
%     while file has more lines:
%     	load line, read every column;
%     	invoke listener;
%     repeat
%     \endgroup
%
%  In short: all invocations of listener have the same level of
%  scoping: they are inside of one TeX group. But single listener
%  invocations as such are not scoped. Make sure you don't
%  accidentally overwrite one of the internals in listener. And: make
%  sure you don't change temporary registers without scoping them!
%
% Note: this command also updates \pgfplotsscanlinelength.
\def\pgfplotstableread{%
	\pgfutil@ifnextchar*{%
		\pgfplots@tableread@use@begingroupfalse
		\pgfplotstableread@impl@star
	}{%
		\pgfplots@tableread@use@begingrouptrue
		\pgfplotstableread@impl@star*%
	}%
}%
\def\pgfplotstableread@impl@star*{%
	\pgfutil@ifnextchar[{%
		\pgfplotstableread@impl
	}{%
		\pgfplotstableread@impl[]%
	}%
}

% BACKWARDS COMPATIBILITY
\let\pgfnumtableread=\pgfplotstableread

% Invokes #2 if '#1' is an already loaded table and #3 if not.
\long\def\pgfplotstable@isloadedtable#1#2#3{%
	\pgfplotsutil@ifdefinedui@withsuffix{#1}{@@table@name}{#2}{#3}%
}%

% If the colum name `#1' is a `create on use' speicifcation, #2 is
% invoked, #3 otherwise.
\long\def\pgfplotstableifiscreateonuse#1#2#3{%
	\pgfkeysifdefined{/pgfplots/table/create on use/#1/.@cmd}{#2}{%
		\pgfutil@in@{create col/}{#1}%
		\ifpgfutil@in@
			#2%
		\else
			#3%
		\fi
	}%
}%

% Defines \pgfplotsretval to be a column name which does not already
% occur in the column names of table #1.
\def\pgfplotstablegenerateuniquecolnamefor#1{%
	\begingroup
	\c@pgf@countd=0
	\pgfplotstableforeachcolumn{#1}\as\pgfplots@loc@TMPa{%
		\def\pgfplots@loc@TMPa{\pgfplots@loc@TMPa}%
		\edef\pgfplots@loc@TMPb{autocol\the\c@pgf@countd}%
		\ifx\pgfplots@loc@TMPa\pgfplots@loc@TMPb
			\advance\c@pgf@countd by1
		\fi
	}%
	\edef\pgfplotsretval{autocol\the\c@pgf@countd}%
	\pgfmath@smuggleone\pgfplotsretval
	\endgroup
}

% Returns a column vector in form of \pgfplotslist
% into #3.
%
% #1: the column name (not a macro)
% #2: the table structure
% #3: the output macro name.
%
% @throw `no such element' on error
\def\pgfplotstablegetcolumnbyname#1\of#2\to#3{%
	\let#3=\pgfutil@empty% just for sanity checking.
	\pgfplotstable@getcol{#1}\of{#2}\to{#3}\getcolmethod{STD}%
}

% Declares a new getcolmethod class #1.
% The argument '#1' can then be used as argument for 
% \pgfplotstable@getcol...\getcolmethod{#1}.
%
% #2: a set of keys which redefine the behavior.
% See below for available choices.
\def\pgfplotstable@definegetcolmethod#1#2{%
	\edef\pgfplotstable@definegetcolmethod@{#1}%
	\pgfqkeys{/pgfplots/table/@getcol}{%
		% Checks if column '#1' exists in table `#2' and invokes `#3' if that
		% is the case and `#4' if not.
		%
		% Note that the check is only for physically existent colums, no alias
		% or create on use is checked here.
		% #1: colname
		% #2: table struct
		% #3: true code
		% #4: false code
		ifexists	=\pgfutil@ifundefined{\string##2@##1}{##4}{##3},
		%
		% Low level method to return (the actual content) of column #1 of
		% table #2 to macro #3.
		% #1: colname
		% #2: table struct
		% #3: result macro
		getexisting	=\expandafter\let\expandafter##3\csname\string##2@##1\endcsname,
		%
		% no arguments, the value is a boolean (0 or 1)
		createonuse	=1,
		%
		% error handling.
		% #1: colname
		% #2: table struct
		% #3: result macro
		lazyforbidden=\pgfplotstablegetcolumnbyname@impl@createonuseforbidden{##1}\of{##2}\to{##3},
		noalias		=\pgfplotstablegetcolumnbyname@impl@nosuchalias{##1}\of{##2}\to{##3},
		nocol		=\pgfplotstablegetcolumnbyname@impl@nocolumn{##1}\of{##2}\to{##3},
		#2%
	}%
}%
\pgfqkeys{/pgfplots/table/@getcol}{%
	ifexists/.code		={\expandafter\def\csname pgfplotstable@getcol@ifexists@\pgfplotstable@definegetcolmethod@\endcsname##1\of##2##3##4{#1}},%
	getexisting/.code	={\expandafter\def\csname pgfplotstable@getcol@getexisting@\pgfplotstable@definegetcolmethod@\endcsname##1\of##2\to##3{#1}},%
	createonuse/.code	={\expandafter\def\csname pgfplotstable@getcol@createonuse@\pgfplotstable@definegetcolmethod@\endcsname{#1}},%
	lazyforbidden/.code	={\expandafter\def\csname pgfplotstable@getcol@lazyforbidden@\pgfplotstable@definegetcolmethod@\endcsname##1\of##2\to##3{#1}},%
	noalias/.code		={\expandafter\def\csname pgfplotstable@getcol@noalias@\pgfplotstable@definegetcolmethod@\endcsname##1\of##2\to##3{#1}},%
	nocol/.code			={\expandafter\def\csname pgfplotstable@getcol@nocol@\pgfplotstable@definegetcolmethod@\endcsname##1\of##2\to##3{#1}},%
}%

\pgfplotstable@definegetcolmethod{STD}{}%
\pgfplotstable@definegetcolmethod{getptr}{%
	ifexists	=\pgfutil@ifundefined{pgfplotstblread@colindex@for@name#1}{#4}{#3},%
	getexisting	=\edef#3{\csname pgfplotstblread@colindex@for@name#1\endcsname},%
}
\pgfplotstable@definegetcolmethod{resolvename}{%
	getexisting	=\edef#3{#1},%
}

% The low level implementation of \pgfplotstablegetcolumnbyname.
%
% #4: is a getcolmethod declared with \pgfplotstable@definegetcolmethod
%
% ATTENTION: this macro should be expandable! It is also used inside
% of \thisrow{}.
\def\pgfplotstable@getcol#1\of#2\to#3\getcolmethod#4{%
	\csname pgfplotstable@getcol@ifexists@#4\endcsname{#1}\of{#2}{%
		\csname pgfplotstable@getcol@getexisting@#4\endcsname{#1}\of{#2}\to#3%
	}{%
		% Oh, there is no column '#1' in table '#2'!
		%
		% Ok, then check for special features.
		%
		% 1. 'create col/...' 
		% 2. 'create on use'
		% 3. 'alias'
		%
		% WARNING : this code has been REPLICATED in
		% *** \pgfplotstablereadvalueofcolname ***
		\if1\csname pgfplotstable@getcol@createonuse@#4\endcsname%
			\pgfutil@in@{create col/}{#1}%
			\ifpgfutil@in@
				\pgfplotstablegenerateuniquecolnamefor{#2}%
				\def\pgfplotstable@loc@TMPa##1{%
					\pgfkeysdef{/pgfplots/table/create on use/##1}{\pgfkeysalso{#1}}%
					\pgfplotstablecreatecol[/pgfplots/table/create on use/##1]{##1}{#2}%
					% and return the newly generated col:
					\csname pgfplotstable@getcol@getexisting@#4\endcsname{##1}\of{#2}\to#3%
				}%
				\expandafter\pgfplotstable@loc@TMPa\expandafter{\pgfplotsretval}%
			\else
				\pgfplotstable@getcol@next{#1}\of{#2}\to{#3}\getcolmethod{#4}%
			\fi
		\else
			\pgfplotstable@getcol@next{#1}\of{#2}\to{#3}\getcolmethod{#4}%
		\fi
	}%
}%
\def\pgfplotstable@getcol@next#1\of#2\to#3\getcolmethod#4{%
	\pgfkeysifdefined{/pgfplots/table/create on use/#1/.@cmd}{%
		% aah - a 'create on use' style exists. Apply it!
		\if1\csname pgfplotstable@getcol@createonuse@#4\endcsname%
			\pgfplotstablecreatecol[/pgfplots/table/create on use/#1]{#1}{#2}%
			% and return the newly generated col:
			\csname pgfplotstable@getcol@getexisting@#4\endcsname{#1}\of{#2}\to#3%
		\else
			\csname pgfplotstable@getcol@lazyforbidden@#4\endcsname{#1}\of{#2}\to#3%
		\fi
	}{%
		% ok, then it is either an alias or it simply doesn't exist
		\pgfkeysifdefined{/pgfplots/table/alias/#1}{%
			\csname pgfplotstable@getcol@ifexists@#4\endcsname{\pgfkeysvalueof{/pgfplots/table/alias/#1}}\of{#2}{%
				\csname pgfplotstable@getcol@getexisting@#4\endcsname{\pgfkeysvalueof{/pgfplots/table/alias/#1}}\of{#2}\to#3%
			}{%
				\csname pgfplotstable@getcol@noalias@#4\endcsname{#1}\of{#2}\to{#3}%
			}%
		}{%
			\csname pgfplotstable@getcol@nocol@#4\endcsname{#1}\of{#2}\to{#3}%
		}%
	}%
}
\def\pgfplotstablegetcolumnbyname@impl@createonuseforbidden#1\of#2\to#3{%
	\pgfplotstablegetcolumnbyname@impl@nocolumn@{#1}\of{#2}\to{#3}\pgfplotstablegetcolumnbyname@impl@createonuseforbiddentext@
}

\def\pgfplotstablegetcolumnbyname@impl@createonuseforbiddentext@#1{%
	\space Note that I found a 'create on use/#1' style, but unfortunately, I can't evaluate it in this context. Please use the 'read completely' key such that it is processed earlier (see the manual for details)%
}%
\def\pgfplotstablegetcolumnbyname@impl@createonuseforbiddentext@disable#1{}%

\def\pgfplotstablegetcolumnbyname@impl@nosuchalias#1\of#2\to#3{
	\pgfplotstablegetcolumnbyname@impl@nosuchalias@{#1}\of{#2}\to{#3}\pgfplotstablegetcolumnbyname@impl@createonuseforbiddentext@disable
}%
	
\def\pgfplotstablegetcolumnbyname@impl@nosuchalias@#1\of#2\to#3#4{%
	\pgfplotsthrow{no such element}{#3}{Sorry, could not retrieve aliased column '\pgfkeysvalueof{/pgfplots/table/alias/#1}' from table '\pgfplotstablenameof{#2}'. The original request was for '#1', which does not exist either.#4{#1}}\pgfeov%
}%

\def\pgfplotstablegetcolumnbyname@impl@nocolumn#1\of#2\to#3{%
	\pgfplotstablegetcolumnbyname@impl@nocolumn@{#1}\of{#2}\to{#3}\pgfplotstablegetcolumnbyname@impl@createonuseforbiddentext@disable
}%
\def\pgfplotstablegetcolumnbyname@impl@nocolumn@#1\of#2\to#3#4{%
	\pgfplotsthrow{no such element}{#3}{Sorry, could not retrieve column '#1' from table '\pgfplotstablenameof{#2}'. Please check spelling (or introduce name aliases).#4{#1}}\pgfeov%
}%


% Retrieves the column *NAME* '#1' of table #2 and writes it into
% '#3'.
%
% If there is no such column, column aliases will be checked. Finally
% if there are no aliases, the command fails with an error.
%
% The 'create on use' statements can't be used in this context.
% @see \pgfplotstablegetcolumnbyname
%
% #1: a column name (not a macro)
% #2: the table structure
% #3: a macro name which will be filled with the (probably modified)
% column name into #2.
\def\pgfplotstableresolvecolname#1\of#2\to#3{%
	\let#3=\pgfutil@empty% just for sanity checking.
	\pgfplotstable@getcol{#1}\of{#2}\to{#3}\getcolmethod{resolvename}%
}

% Invokes either \pgfplotstablegetcolumnbyindex or
% \pgfplotstablegetcolumnbyname.
%
% #1: either a column name, alias or create on use specification, or
% [index]<integer> denoting a column index.
% #2: either a loaded table or a table macro.
% #3: a macro name which will be filled with the column, in the format
% accepted of \pgfplotslist...
\long\def\pgfplotstablegetcolumn#1\of#2\to#3{%
	\begingroup
	\pgfplotstable@isloadedtable{#2}{%
		\pgfplotstablegetcolumnfromstruct{#1}\of{#2}\to{#3}%
	}{%
		\pgfplotstableread{#2}\pgfplotstable@tmptbl
		\pgfplotstablegetcolumnfromstruct{#1}\of\pgfplotstable@tmptbl\to{#3}%
	}%
	\pgfmath@smuggleone#3%
	\endgroup
}%
\def\pgfplotstablegetcolumnfromstruct#1\of#2\to#3{%
	\def\pgfplotstable@loc@TMPc{#1}%
	\pgfplotstable@is@colname{\pgfplotstable@loc@TMPc}%%
	\ifpgfplotstableread@foundcolnames
	\else
		\expandafter\pgfplotstablegetcolumnnamebyindex\pgfplotstable@loc@TMPc\of{#2}\to\pgfplotstable@loc@TMPc
	\fi
	\expandafter\pgfplotstablegetcolumnbyname\pgfplotstable@loc@TMPc\of#2\to{#3}%
}%

% Defines #3 to be the column name of the column with index '#1'.
%
% #1: a column index (starting with 0)
% #2: a loaded table structure
% #3: an output macro name.
\def\pgfplotstablegetcolumnnamebyindex#1\of#2\to#3{%
	\pgfplotslistselect#1\of#2\to#3\relax
}%

% Defines #3 to be the column index of the column with name '#1'.
%
% #1: a column name
% #2: a loaded table structure
% #3: an output macro name which will be filled with the index (or -1
% if there is no such index)
\def\pgfplotstablegetcolumnindexforname#1\of#2\to#3{%
	\def#3{-1}%
	\edef\pgfplotstablegetcolumnindexforname@@{#1}%
	\pgfplotstableforeachcolumn{#2}\as\pgfplotstablegetcolumnindexforname@{%
	\relax
		\ifx\pgfplotstablegetcolumnindexforname@\pgfplotstablegetcolumnindexforname@@
			\edef#3{\pgfplotstablecol}%
		\fi
	}%
}%
\def\pgfplotstablegetcolumnbyindex#1\of#2\to#3{%
	\pgfplotslistselect#1\of#2\to#3\relax
	\expandafter\pgfplotstablegetcolumnbyname#3\of#2\to{#3}%
}

\def\pgfplotstablecopy#1\to#2{%
	\let#2=#1%
	\pgfplotstablegetname#1\pgfplotstable@loc@TMPa
	\expandafter\let\csname\string#2@@table@name\endcsname=\pgfplotstable@loc@TMPa
	\expandafter\edef\csname\string#2@@table@scanline\endcsname{\pgfplotstablescanlinelengthof{#1}}%
	\pgfplotslistforeachungrouped#1\as\pgfplotstable@loc@TMPa{%
		\def\pgfplotstable@loc@TMPb{%
			\expandafter\let\csname\string#2@\pgfplotstable@loc@TMPa\endcsname}%
		\expandafter\pgfplotstable@loc@TMPb\csname\string#1@\pgfplotstable@loc@TMPa\endcsname
	}%
}

% Returns the file name of table '#1' into macro #2.
\def\pgfplotstablegetname#1#2{%
	\expandafter\let\expandafter#2\csname\string#1@@table@name\endcsname
}
	
% expands to the table file name of table '#1'
\def\pgfplotstablenameof#1{%
	\csname\string#1@@table@name\endcsname
}

% Returns the value of \pgfplotsscanlinelength for table '#1' into
% macro #2.
\def\pgfplotstablegetscanlinelength#1#2{%
	\expandafter\let\expandafter#2\csname\string#1@@table@scanline\endcsname
}%

% Expands to the scan line length of table '#1'.
\def\pgfplotstablescanlinelengthof#1{\csname\string#1@@table@scanline\endcsname}%

%%%%%%%%%%%%%%%%%%%%%%%%%%%%%%%%%%%%%%%%%%%%%%%%%%%%%%%%%%%%%%%%%%%%%%%%%%%
%
% IMPLEMENTATION
%
%%%%%%%%%%%%%%%%%%%%%%%%%%%%%%%%%%%%%%%%%%%%%%%%%%%%%%%%%%%%%%%%%%%%%%%%%%%
\newif\ifpgfplotstableread@curline@contains@colnames
\newif\ifpgfplotstableread@foundcolnames
\newif\ifpgfplotstableread@skipline

% A method which is necessary to work with inline table data.
%
% It does nothing if format=file.
%
% It needs to be invoked BEFORE the inline table data has been
% seen the first time. "Seen" means collected as argument!
%
% The macro changes \catcodes in order to implement the 'row sep=newline',
% some special 'col sep' choices and the 'ignore chars' features.
%
% ATTENTION: this changes the processing of ALL following newline
% characters.
% @see \pgfplotstablecollectoneargwithpreparecatcodes
\def\pgfplotstablereadpreparecatcodes{%
	\ifx\pgfplotstablereadrestorecatcodes\relax
		\edef\pgfplotstablereadrestorecatcodes{%
			\noexpand\pgfplotstableuninstallignorechars
			\noexpand\catcode`\noexpand\^^M=\the\catcode`\^^M\noexpand\space
			\noexpand\catcode`\noexpand\ =\the\catcode`\ \noexpand\space
			\noexpand\catcode`\noexpand\;=\the\catcode`\;\noexpand\space
			\noexpand\catcode`\noexpand\:=\the\catcode`\:\noexpand\space
			\noexpand\catcode`\noexpand\^^I=\the\catcode`\^^I\noexpand\space
			\noexpand\let\noexpand\pgfplotstablereadrestorecatcodes=\noexpand\relax
		}%
		\ifcase\pgfplotstableread@FORMAT@CASE\relax
			% format=auto
			\pgfplotstablereadpreparecatcodes@
		\or
			% format=inline
			\pgfplotstablereadpreparecatcodes@
		\fi
	\fi
}
\let\pgfplotstablereadrestorecatcodes=\relax
\def\pgfplotstablereadpreparecatcodes@{%
	\if0\pgfplotstableread@ROWSEP@CASE\relax
		% row sep = newline
		\catcode`\^^M=12
	\fi
	\ifcase\pgfplotstableread@COLSEP@CASE\relax
		% col sep=space:
		\catcode`\ =10
	\or
		% col sep=comma:
	\or
		% col sep=semicolon:
		\catcode`\;=12
	\or
		% col sep=colon:
		\catcode`\:=12
	\or
		% col sep=brace:
	\or
		% col sep=tab:
		\catcode`\^^I=12
	\fi
	\pgfplotstableinstallignorechars
}%

% Logically, this routine does
% \pgfplotstablereadpreparecatcodes
% #1
% and that's all. However, It makes sure that any white space
% characters (especially newlines) between '#1' and the following
% argument are gobbled!
%
% To show the problem, consider
% \pgfplotstabletypeset[<options>]
% 	\loadedtable
% 
% and we want to set the catcodes *before* \loadedtable is seen.
% Well, there is a newline character after ']'! If we change the
% catcodes, this newline will be considered as character and will make
% up the first argument, the \loadedtable will be ignored.
%
% \pgfplotstablecollectoneargwithpreparecatcodes avoids this problem
% by gobbling all space (newline) characters *before* collecting the
% directly following argument token. Furthermore, it restores the
% catcodes directly after the argument has been collected.
%
% @see \pgfplotstablecollectoneargwithpreparecatcodesnorestore
\long\def\pgfplotstablecollectoneargwithpreparecatcodes#1{%
	\begingroup
	\t@pgfplotstable@a{{#1}}% this allows '#' characters in '#1'
	\def\pgfplotstablecollectoneargwithpreparecatcodes@@{%
		\pgfplotstablereadpreparecatcodes
		\expandafter\pgfplotstablecollectoneargwithpreparecatcodes@end\the\t@pgfplotstable@a%
	}%
	% equivalent to
	%\let\pgfplotstable@loc@TMPa=<next token>\pgfplotstablecollectoneargwithpreparecatcodes@@<next token>
	\futurelet\pgfplotstable@loc@TMPa\pgfplotstablecollectoneargwithpreparecatcodes@@
	% this employs a side effect: \futurelet gobbles any spaces
	% (including newlines).
}%
\def\pgfplotstablecollectoneargwithpreparecatcodes@end#1#2{%
	\endgroup
	#1{#2}%
}%

% An overloaded method of
% \pgfplotstablecollectoneargwithpreparecatcodes which does not
% restore the catcodes after collecting one arg.
\long\def\pgfplotstablecollectoneargwithpreparecatcodesnorestore#1{%
	\t@pgfplotstable@a{{#1}}%
	\def\pgfplotstablecollectoneargwithpreparecatcodes@@{%
		\expandafter\pgfplotstablereadpreparecatcodes\t@pgfplotstable@a
	}%
	% equivalent to
	%\let\pgfplotstable@loc@TMPa=<next token>\pgfplotstablecollectoneargwithpreparecatcodes@@<next token>
	\futurelet\pgfplotstable@loc@TMPa\pgfplotstablecollectoneargwithpreparecatcodes@@
}%

% Accept one of
% \pgfplotstableread[#1]{<file>}{<\macro>}
% \pgfplotstableread[#1]{<file>} to listener{<\macro>}
% or
% \pgfplotstableread[#1]{<file>} to {<\macro>} (DEPRECATED)
\long\def\pgfplotstableread@impl[#1]{%
	\ifpgfplots@tableread@use@begingroup
		\begingroup
		\def\endgroup@@pgfplotstableread{\endgroup}%
	\else
		\let\endgroup@@pgfplotstableread=\relax%
	\fi
	% set options here, although we can't evaluate 'every table' yet
	% (the file name is not yet known).
	% But #1 may contain input format specifiers which are important
	% for \catcodes, BEFORE we have read the second argument:
	\pgfplotstableset{#1}%
	%
	\pgfplotstablecollectoneargwithpreparecatcodes{%
		\pgfplotstableread@impl@collectfirstarg{#1}%
	}%
}%
\long\def\pgfplotstableread@impl@collectfirstarg#1#2{%
	\pgfplotsutil@ifnextchar t{%
		\pgfplotstableread@impl@@{#1}{#2}%
	}{%
		\pgfplotstableread@impl@{#1}{#2}%
	}%
}%


% I don't know why; but I started with 
% >> \pgfplotstableread[]{file} to \macro
% That ' to ' is really ugly. This here is for backwards
% compatibility:
\long\def\pgfplotstableread@impl@@#1#2to {%
	\pgfutil@ifnextchar l{%
		\pgfplotstableread@impl@@listener{#1}{#2}%
	}{%
		\pgfplotstableread@impl@{#1}{#2}%
	}%
}%
\long\def\pgfplotstableread@impl@@listener#1#2listener#3{%
	\pgfplots@tableread@to@listenertrue
	\pgfplotstableread@impl@{#1}{#2}{#3}%
	\pgfplots@tableread@to@listenerfalse
}%

\newif\ifpgfplotstableread@inline
\long\def\pgfplotstableread@checkinlineformat@CRCR#1\\#2\pgfplotstable@EOI{%
	\def\pgfplotstable@loc@TMPa{#2}%
	\ifx\pgfplotstable@loc@TMPa\pgfutil@empty
		\pgfplotstableread@inlinefalse
	\else
		\pgfplotstableread@inlinetrue
		\let\pgfplotstableread@loop@next=\pgfplotstableread@loop@next@CRCR
	\fi
}%
\long\def\pgfplotstableread@loop@next@CRCR#1\\{%
	\long\def\pgfplotstable@LINE{#1}%
	\pgfplotstableread@loop@processnextline
	\pgfplotstableread@loop@over@lines
}%
\begingroup
\catcode`\^^M=12\relax%
\long\gdef\pgfplotstableread@checkinlineformat@NL@#1^^M#2\pgfplotstable@EOI{%
	\def\pgfplotstable@loc@TMPa{#2}%
	\ifx\pgfplotstable@loc@TMPa\pgfutil@empty%
		\pgfplotstableread@inlinefalse%
	\else%
		\pgfplotstableread@inlinetrue%
		\let\pgfplotstableread@loop@next=\pgfplotstableread@loop@next@NL%
	\fi%
}%
\long\gdef\pgfplotstableread@checkinlineformat@NL#1{%
	\pgfplotstableread@checkinlineformat@NL@ #1^^M\pgfplotstable@EOI%
}%
\long\gdef\pgfplotstableread@loop@next@NL#1^^M{%
	\long\def\pgfplotstable@LINE{#1}%
	\pgfplotstableread@loop@processnextline%
	\pgfplotstableread@loop@over@lines%
}%
\endgroup

% #1: options.
% #2: the table content (file name or inline data)
% #3: the result macro.
\long\def\pgfplotstableread@impl@#1#2#3{%
	\ifcase\pgfplotstableread@FORMAT@CASE\relax
		% format=auto
		\ifcase\pgfplotstableread@ROWSEP@CASE\relax
			% row sep=newline
			\pgfplotstableread@checkinlineformat@NL{#2}%
		\or
			% row sep=crcr
			\pgfplotstableread@checkinlineformat@CRCR #2\\\pgfplotstable@EOI
		\fi
	\or
		% format=inline
		\pgfplotstableread@inlinetrue
		\ifcase\pgfplotstableread@ROWSEP@CASE\relax
			% row sep=newline
			\let\pgfplotstableread@loop@next=\pgfplotstableread@loop@next@NL
		\or
			% row sep=crcr
			\let\pgfplotstableread@loop@next=\pgfplotstableread@loop@next@CRCR
		\fi
	\or
		\pgfplotstableread@inlinefalse
	\fi
	%
	% optimizations:
	\pgfkeysgetvalue{/pgfplots/table/comment chars}\pgfplots@loc@TMPd
	\ifx\pgfplots@loc@TMPd\pgfutil@empty
		\let\pgfplotstableread@checkspecial@line@default=\relax
	\fi
	\pgfkeysgetvalue{/pgfplots/table/skip first n}\pgfplots@loc@TMPd
	\ifx\pgfplots@loc@TMPd\pgfutil@empty \let\pgfplotstableread@check@skipfirstn=\relax \fi
	\ifnum\pgfplots@loc@TMPd=0           \let\pgfplotstableread@check@skipfirstn=\relax \fi
	%
	\ifpgfplotstableread@inline
		\def\pgfplotstableread@impl@fCLOSE{}%
		\let\pgfplotstableread@loop@over@lines=\pgfplotstableread@loop@over@lines@frominline%
		\long\def\pgfplotstableread@start@loop{%
			\pgfplotstableread@loop@over@lines #2\pgfplotstable@EOI
		}%
		\def\pgfplotstableread@ready{1}%
		\def\pgfplotstableread@filename{<inline_table>}%
	\else
		\def\pgfplotstableread@ready{1}%
		\def\pgfplotstableread@filename{#2}%
		\pgfplotstableread@openfile
		\def\pgfplotstableread@impl@fCLOSE{%
			\closein\r@pgfplots@reada
		}%
		\let\pgfplotstableread@loop@over@lines=\pgfplotstableread@loop@over@lines@fromfile%
		\def\pgfplotstableread@start@loop{\pgfplotstableread@loop@over@lines}%
		\pgfplotstableinstallignorechars
	\fi
	\edef\pgfplotstableread@oldendlinechar{\the\endlinechar}%
	\endlinechar=-1 % do not append a white space after each line for file input
%\pgfplots@message{ATTEMPTING TO READ \pgfplotstableread@filename}%
	%
	\def\pgfplots@loc@TMPa{\pgfplotstableread@impl@prepare{#1}}%
	\expandafter\pgfplots@loc@TMPa\expandafter{\pgfplotstableread@filename}{#3}%
	%
	\if1\pgfplotstableread@ready
		\pgfplotstableread@start@loop
		%
		\pgfplotstableread@impl@fCLOSE%
		%
		\pgfplotstableread@finish
		%
	\fi
	\endlinechar=\pgfplotstableread@oldendlinechar
	\pgfplotsscanlinelengthcleanup
	\expandafter\pgfplotstable@copy@to@globalbuffers@simple\expandafter{\pgfplotstableread@filename}%
	\endgroup@@pgfplotstableread
	\ifpgfplots@tableread@to@listener
		% there are no data structures in this case.
		\let\pgfplotsscanlinelength=\pgfplotstable@glob@buf@scanline
	\else
		% Now, we can access the global variables!
		% copy them to #3.
		\pgfplotstable@copy@globalbuffers@to#3%
	\fi
}

{\catcode`\"=12 \gdef\pgfplots@dquote{"}}

% Opens \pgfplotstableread@filename.
\def\pgfplotstableread@openfile{%
	\def\pgfplotstable@loc@TMPa{\pgfutil@in@{ }}%
	\expandafter\pgfplotstable@loc@TMPa\expandafter{\pgfplotstableread@filename}%
	\ifpgfutil@in@
		\t@pgfplots@toka=\expandafter{\pgfplotstableread@filename}%
		\edef\pgfplotstableread@filename{\pgfplots@dquote\the\t@pgfplots@toka\pgfplots@dquote}%
	\fi
	\let\pgfplotstableread@old@crcr=\\%
	\def\\{\string\\}% just to make sure we don't try to open inline table data...
	\openin\r@pgfplots@reada=\pgfplotstableread@filename.tex
	\ifeof\r@pgfplots@reada
		\openin\r@pgfplots@reada=\pgfplotstableread@filename\relax
	\else
		\pgfplots@warning{%
			You requested to open table '\pgfplotstableread@filename', but there is also a '\pgfplotstableread@filename.tex'. 
			TeX will automatically append the suffix '.tex', so I will now open '\pgfplotstableread@filename.tex'.
			Please make sure you don't accidentally load TeX files - this may produce unrecoverable errors.}%
		\closein\r@pgfplots@reada
		\openin\r@pgfplots@reada=\pgfplotstableread@filename\relax
	\fi
	%
	\ifeof\r@pgfplots@reada
		\pgfplotsthrow{no such table file}{\pgfplots@loc@TMPa}{\pgfplotstableread@filename}{Could not read table file '\pgfplotstableread@filename'. In case you intended to provide inline data: maybe TeX screwed up your end-of-lines? Try `row sep=crcr' and terminate your lines with `\string\\' (refer to the pgfplotstable manual for details)}\pgfeov%
		\global\let\pgfplotstable@colnames@glob=\pgfplots@loc@TMPa
		\def\pgfplotstableread@ready{0}%
	\fi
	\pgfplots@logfileopen{\pgfplotstableread@filename}%
	\let\\=\pgfplotstableread@old@crcr
}

\def\pgfplotstableinstallignorechars{%
	\ifx\pgfplotstableuninstallignorechars\pgfutil@empty
		\pgfkeysgetvalue{/pgfplots/table/ignore chars}\pgfplotstable@loc@TMPa
		\ifx\pgfplotstable@loc@TMPa\pgfutil@empty
		\else
			\pgfplotstableinstallignorechars@\pgfplotstable@loc@TMPa{9}%
		\fi
		\pgfkeysgetvalue{/pgfplots/table/white space chars}\pgfplotstable@loc@TMPa
		\ifx\pgfplotstable@loc@TMPa\pgfutil@empty
		\else
			\pgfplotstableinstallignorechars@\pgfplotstable@loc@TMPa{10}%
		\fi
		\ifx\pgfplotstableuninstallignorechars\pgfutil@empty
		\else
			\expandafter\def\expandafter\pgfplotstableuninstallignorechars\expandafter{%
				\pgfplotstableuninstallignorechars
				\let\pgfplotstableuninstallignorechars=\pgfutil@empty
			}%
		\fi
	\fi
}%		
\let\pgfplotstableuninstallignorechars\pgfutil@empty%
% #1 macro containing the characters as comma-separated list
% #2 the catcode to assign
\def\pgfplotstableinstallignorechars@#1#2{%
	\expandafter\pgfplotsutilforeachcommasep\expandafter{#1}\as\pgfplotstable@loc@TMPa{%
		\t@pgfplots@toka=\expandafter{\pgfplotstableuninstallignorechars}%
		\edef\pgfplotstableuninstallignorechars{%
			\the\t@pgfplots@toka
			\noexpand\catcode`\expandafter\noexpand\pgfplotstable@loc@TMPa=\expandafter\the\expandafter\catcode\expandafter`\pgfplotstable@loc@TMPa\noexpand\space
		}%
		\expandafter\catcode\expandafter`\pgfplotstable@loc@TMPa=#2\relax
	}%
}%

% #1: any options to set (respect \ifpgfplots@table@options@areset )
% #2: the file name (if any)
% #3: the output macro (or listener)
\def\pgfplotstableread@impl@prepare#1#2#3{%
	\ifpgfplots@table@options@areset
	\else
		\pgfplotstableset@every@table{#2}{#1}%
	\fi
	%
	\def\pgfplotstablename{\pgfplotstable@colnames@glob}% provide the name of the actual table struct (during \pgfplotstableread, only partial functionality is available!)
	%
	% local counter definitions:
	\ifpgfplots@tableread@use@begingroup
		\let\pgfplotstableread@lineno=\c@pgf@counta
		\let\pgfplotstableread@numcols=\c@pgf@countb
		\let\pgfplotstableread@curcol=\c@pgf@countc
		\let\pgfplotstableread@usablelineno=\c@pgf@countd
		\def\thepgfplotstableread@lineno{\the\pgfplotstableread@lineno}%
		\def\thepgfplotstableread@usablelineno{\the\pgfplotstableread@usablelineno}%
		\def\thepgfplotstableread@curcol{\the\pgfplotstableread@curcol}%
		\def\thepgfplotstableread@numcols{\the\pgfplotstableread@numcols}%
		\def\pgfplotstableread@countreset##1{##1=0 }%
		\def\pgfplotstableread@countset##1##2{##1=##2\relax}%
		\def\pgfplotstableread@countadvance##1{\advance##1 by1 }%
	\else
		% don't re-use integers! We have no protecting scopes!
		\def\thepgfplotstableread@lineno{\pgfplotstableread@lineno}%
		\def\thepgfplotstableread@usablelineno{\pgfplotstableread@usablelineno}%
		\def\thepgfplotstableread@curcol{\pgfplotstableread@curcol}%
		\def\thepgfplotstableread@numcols{\pgfplotstableread@numcols}%
		\def\pgfplotstableread@countreset##1{\def##1{0}}%
		\def\pgfplotstableread@countset##1##2{\def##1{##2}}%
		\def\pgfplotstableread@countadvance##1{\pgfplotsutil@advancestringcounter{##1}}%
	\fi
	\pgfplotstableread@countreset\pgfplotstableread@lineno
	\pgfplotstableread@countreset\pgfplotstableread@usablelineno
	\pgfplotstableread@countreset\pgfplotstableread@numcols
	\pgfplotstableread@countreset\pgfplotstableread@curcol
	%
	\pgfplotstableread@impl@prepare@DO
	\def\pgfplotstableread@isgnuplotformat{0}%
	\global\pgfplotslistnewempty\pgfplotstable@colnames@glob
	\pgfplotsscanlinelengthinitzero
	\ifpgfplots@tableread@to@listener
		\def\pgfplotstablerow{\thepgfplotstableread@usablelineno}%
		\def\pgfplotstablelineno{\thepgfplotstableread@lineno}%
		\let\pgfplotstable@listener=#3%
		\let\pgfplotstableread@impl@nextrow@NEXT=\pgfplotstableread@impl@nextrow@NEXT@listener
		\let\pgfplotstablereadgetcolindex=\pgfplotstablereadgetcolindex@
		\let\pgfplotstablereadgetcolname=\pgfplotstablereadgetcolname@
		\let\pgfplotstablereadgetptrtocolname=\pgfplotstablereadgetptrtocolname@
		\let\pgfplotstablereadgetptrtocolindex=\pgfplotstablereadgetptrtocolindex@
		\let\pgfplotstablereadevalptr=\pgfplotstablereadevalptr@
		\let\pgfplotstablereadvalueofptr=\pgfplotstablereadvalueofptr@
		\let\pgfplotstableforeachcolumn=\pgfplotstableforeachcolumn@listener
		\let\pgfplotstablereadvalueofcolname=\pgfplotstablereadvalueofcolname@
		\let\pgfplotstablereadvalueofcolindex=\pgfplotstablereadvalueofcolindex@
		\let\getthisrow=\pgfplotstablereadgetcolname
		\let\thisrow=\pgfplotstablereadvalueofcolname
		\let\thisrowno=\pgfplotstablereadvalueofcolindex
		\let\getthisrowno=\pgfplotstablereadgetcolindex
	\fi
}%

% Copies the table column list and the column vectors of #1 to global buffers. 
% @see \pgfplotstable@copy@globalbuffers@to
%
% Use these two methods to avoid scoping problems.
%
% #1: the <\macro> name of the table which is to be copied to global buffers.
% #2: the table file name.
\def\pgfplotstable@copy@to@globalbuffers#1#2{%
	\global\let\pgfplotstable@colnames@glob=#1\relax
	\c@pgfplotstable@counta=0\relax%
	\pgfplotslistforeachungrouped\pgfplotstable@colnames@glob\as\pgfplotstable@loc@TMPa{%
		\def\pgfplotstable@loc@TMPb{%
			\expandafter\global\expandafter\let\csname pgfp@numtable@glob@col@\the\c@pgfplotstable@counta\endcsname%
		}%
		\expandafter\pgfplotstable@loc@TMPb\csname\string#1@\pgfplotstable@loc@TMPa\endcsname
		\advance\c@pgfplotstable@counta by1\relax
	}%
	\pgfplotstable@copy@to@globalbuffers@simple{#2}%
}%
% A variant of \pgfplotstable@copy@to@globalbuffers which copies only
% the member variables of a loaded table (the name and scanline length) to
% global buffers.
%
% #1: the table's file name.
%
% @PRECONDITION: any other variables and cell data of the table are already stored
% in global buffers.
\def\pgfplotstable@copy@to@globalbuffers@simple#1{%
	\gdef\pgfplotstable@glob@buf@name{#1}%
	\global\let\pgfplotstable@glob@buf@scanline=\pgfplotsscanlinelength
}%

% copies the global column list and the global column vectors to #1
% (NOT the table file name).
% @see \pgfplotstable@copy@to@globalbuffers
%
% @PRECONDITION the global buffers contain all members of a table.
% 
% @POSTCONDITION The table '#1' is be initialised to these members.
% 		Furthermore, \pgfplotsscanlinelength is set.
\def\pgfplotstable@copy@globalbuffers@to#1{%
	\let#1=\pgfplotstable@colnames@glob
	\c@pgfplotstable@counta=0\relax%
	\pgfplotslistforeachungrouped\pgfplotstable@colnames@glob\as\pgfplotstable@loc@TMPa{%
		\def\pgfplotstable@loc@TMPb{%
			\expandafter\let\csname\string#1@\pgfplotstable@loc@TMPa\endcsname}%
		\expandafter\pgfplotstable@loc@TMPb\csname pgfp@numtable@glob@col@\the\c@pgfplotstable@counta\endcsname
%\message{Column '\pgfplotstable@loc@TMPa' has entries: \expandafter\meaning\csname pgfp@numtable@glob@col@\the\c@pgfplotstable@counta\endcsname}%
		\expandafter\global\expandafter\let\csname pgfp@numtable@glob@col@\the\c@pgfplotstable@counta\endcsname=\pgfutil@empty
		\advance\c@pgfplotstable@counta by1\relax
	}%
	\global\let\pgfplotstable@colnames@glob=\pgfutil@empty
	\expandafter\let\csname\string#1@@table@scanline\endcsname=\pgfplotstable@glob@buf@scanline
	\expandafter\let\csname\string#1@@table@name\endcsname=\pgfplotstable@glob@buf@name
	\let\pgfplotsscanlinelength=\pgfplotstable@glob@buf@scanline
}%

\def\pgfplotstableread@finish{%
	\pgfplotsscanlinecomplete
	\ifpgfplots@tableread@to@listener
	\else
		\ifpgfplotstableread@foundcolnames
		\else
			\pgfplotstableread@create@column@names@with@numbers
		\fi
		\pgfplotstableread@countreset\pgfplotstableread@curcol%
		\pgfutil@loop
		\ifnum\pgfplotstableread@curcol<\pgfplotstableread@numcols\relax
			\expandafter\pgfplotsapplistXflushbuffers\csname pgfp@numtable@glob@col@\thepgfplotstableread@curcol\endcsname
			\pgfplotstableread@countadvance\pgfplotstableread@curcol
		\pgfutil@repeat
	\fi
}

\def\pgfplotstableread@loop@over@lines@fromfile{%
	\ifeof\r@pgfplots@reada
%\pgfplots@message{EOF}%
	\else
		\read\r@pgfplots@reada to\pgfplotstable@LINE
		\ifeof\r@pgfplots@reada
		\else
			\pgfplotstableread@loop@processnextline
		\fi
		\expandafter\pgfplotstableread@loop@over@lines
	\fi
}%
\def\pgfplotstableread@loop@over@lines@frominline{%
	\pgfutil@ifnextchar\pgfplotstable@EOI{%
		\pgfutil@gobble
	}{%	
		\pgfplotstableread@loop@next
	}%
}%

\def\pgfplotstableread@check@skipfirstn{%
	\ifnum\pgfplotstableread@lineno<\pgfkeysvalueof{/pgfplots/table/skip first n} %
		\pgfplotstableread@skiplinetrue
	\fi
}%

% PRECONDITION: 
% 	\pgfplotstable@LINE contains the current input line.
\def\pgfplotstableread@loop@processnextline{%
	\expandafter\pgfplotstableread@checkspecial@line\pgfplotstable@LINE\pgfplotstable@EOI
	\pgfplotstableread@check@skipfirstn
	\ifpgfplotstableread@skipline
		\def\pgfplotstableread@gnuplotcheck{####x y type}%
		\ifx\pgfplotstableread@gnuplotcheck\pgfplotstable@LINE
			\def\pgfplotstableread@isgnuplotformat{1}%
		\fi
		\def\pgfplotstableread@gnuplotcheck{####x y z type}%
		\ifx\pgfplotstableread@gnuplotcheck\pgfplotstable@LINE
			\def\pgfplotstableread@isgnuplotformat{1}%
		\fi
	\else
		%--------------------------------------------------
		% \ifnum\pgfplotstableread@lineno=0
		% 	\let\pgfplotstable@firstline=\pgfplotstable@LINE
		% \fi
		%-------------------------------------------------- 
%\pgfplots@message{READING LINE \thepgfplotstableread@lineno: '\meaning\pgfplotstable@LINE'.}%
		\pgfplotstableread@curline@contains@colnamesfalse
		\ifnum\pgfplotstableread@numcols=0\relax
			% STEP 0: initialise
			% 	- count columns
			% 	- find header data
			\pgfplotstableread@countreset\pgfplotstableread@curcol
			\ifpgfplotstable@firstline@is@header
				\pgfplotstableread@curline@contains@colnamestrue
			\fi
			\pgfplotstableread@impl@DO\pgfplotstableread@impl@countcols@and@identifynames@NEXT\pgfplotstable@LINE
			\pgfplotstableread@countset\pgfplotstableread@numcols{\pgfplotstableread@curcol}%
			\edef\pgfplotstablecols{\thepgfplotstableread@numcols}%
			\pgfplotstableread@countreset\pgfplotstableread@curcol
			% Create empty column lists:
			\pgfplotstableread@create@column@lists
			%
			\ifnum\pgfplotstableread@usablelineno=0\relax
			\if1\pgfplotstableread@isgnuplotformat%
				% The file started with
				% #...
				% #x y type
				% X Y i
				% -> thats a gnuplot file!
				\pgfplotstableread@curline@contains@colnamesfalse
			\fi
			\fi
			% Now, read the first line.
			% It contains either
			% - column names,
			% - numerical data,
			% - nothing (comments).
			\ifpgfplotstableread@curline@contains@colnames
				\pgfplotstableread@foundcolnamestrue
				\pgfplotstableread@countreset\pgfplotstableread@curcol
				\pgfplotstableread@impl@DO\pgfplotstableread@impl@collectcolnames@NEXT\pgfplotstable@LINE
			\else
				\pgfplotsscanlinelengthincrease
				\pgfplotstableread@foundcolnamesfalse
				\pgfplotstableread@countreset\pgfplotstableread@curcol
				% Leave column name lists empty...
				\pgfplotstableread@impl@DO\pgfplotstableread@impl@nextrow@NEXT\pgfplotstable@LINE
			\fi
%\pgfplots@message{After reading first row: found '\thepgfplotstableread@numcols' columns; column name list='\meaning\pgfplotstable@colnames@glob'}%
		\else
			\pgfplotsscanlinelengthincrease
			\pgfplotstableread@countreset\pgfplotstableread@curcol
			\pgfplotstableread@impl@DO\pgfplotstableread@impl@nextrow@NEXT\pgfplotstable@LINE
		\fi
		\ifnum\pgfplotstableread@curcol=\pgfplotstableread@numcols\relax
		\else
			\pgfplotstable@error{input table '\pgfplotstableread@filename' has an unbalanced number  of columns in row '\thepgfplotstableread@lineno' (expected '\thepgfplotstableread@numcols' cols; got '\thepgfplotstableread@curcol'). Maybe the input table is corrupted? If you need unbalanced data, consider using 'nan' in empty cells (perhaps combined with 'unbounded coords=jump')}%
		\fi
		\ifpgfplots@tableread@to@listener
			\ifpgfplotstableread@curline@contains@colnames
			\else
				% report row!
				\pgfplotstable@listener
			\fi
		\fi
		\pgfplotstableread@countadvance\pgfplotstableread@usablelineno
	\fi
	\pgfplotstableread@countadvance\pgfplotstableread@lineno
}%

% WARNING: this routine is also used in pgfplots.code.tex ...
\def\pgfplotstableread@checkspecial@line{%
	\futurelet\pgfplotstableread@tmp\pgfplotstableread@checkspecial@line@
}%
\def\pgfplotstableread@checkspecial@line@{%
	\pgfplotstableread@skiplinefalse
	\let\pgfplotstableread@checkspecial@line@@\pgfplotstableread@impl@gobble%
	%
	\let\pgfplotstableread@tmpb=##%
	\ifx\pgfplotstableread@tmp\pgfplotstableread@tmpb
		\pgfplotstableread@skiplinetrue
	\else
		\let\pgfplotstableread@tmpb=$%
		\ifx\pgfplotstableread@tmp\pgfplotstableread@tmpb
			\let\pgfplotstableread@checkspecial@line@@=\pgfplotstableread@process@flags@line
		\else
			\let\pgfplotstableread@tmpb=\pgfplotstable@EOI%
			\ifx\pgfplotstableread@tmp\pgfplotstableread@tmpb
				\pgfplotstableread@skiplinetrue
				\pgfplotsscanlinecomplete% the line is empty; same as \par!
			\else
				\let\pgfplotstableread@tmpb=\par%
				\ifx\pgfplotstableread@tmp\pgfplotstableread@tmpb
					\pgfplotstableread@skiplinetrue
					\pgfplotsscanlinecomplete% the line is empty;
				\else
					\pgfplotstableread@checkspecial@line@default%
				\fi
			\fi
		\fi
	\fi
	\pgfplotstableread@checkspecial@line@@
}%
\def\pgfplotstableread@checkspecial@line@default{%
	% this routine can be optimized away
	\pgfkeysgetvalue{/pgfplots/table/comment chars}\pgfplots@loc@TMPd
	\ifx\pgfplots@loc@TMPd\pgfutil@empty
	\else
		\expandafter\pgfplotsutilforeachcommasep\expandafter{\pgfplots@loc@TMPd}\as\pgfplots@loc@TMPd{%
			\expandafter\let\expandafter\pgfplotstableread@tmpb\pgfplots@loc@TMPd
			\ifx\pgfplotstableread@tmp\pgfplotstableread@tmpb
				\pgfplotstableread@skiplinetrue
			\fi
		}%
	\fi
}%

%--------------------------------------------------
% \def\pgfplotstableread@checkspecial@line{%
% 	\pgfutil@ifnextchar##{%
% 		\pgfplotstableread@skiplinetrue
% 		\pgfplotstableread@impl@gobble
% 	}{%
% 		\pgfutil@ifnextchar${%
% 			\pgfplotstableread@process@flags@line
% 		}{%
% 			\pgfutil@ifnextchar\pgfplotstable@EOI{%
% 				\pgfplotstableread@skiplinetrue
% 				\pgfplotsscanlinecomplete% the line is empty; same as \par!
% 				\pgfplotstableread@impl@gobble
% 			}{%
% 				\pgfutil@ifnextchar\par{%
% 					\pgfplotstableread@skiplinetrue
% 					\pgfplotsscanlinecomplete
% 					\pgfplotstableread@impl@gobble
% 				}{%
% 					\pgfplotstableread@skiplinefalse
% 					\pgfplotstableread@impl@gobble
% 				}%
% 			}%
% 		}%
% 	}%
% }
%-------------------------------------------------- 

\long\def\pgfplotstableread@process@flags@line$flags {%
%\pgfplots@message{Ignoring flags line ...}%
	\pgfplotstableread@skiplinetrue
	\pgfplotstableread@impl@gobble
}

\def\pgfplotstableread@create@column@lists{%
	\pgfutil@loop
	\ifnum\pgfplotstableread@curcol<\pgfplotstableread@numcols\relax
		\def\pgfplots@loc@TMPa{\pgfplotsapplistXnewempty[to global]}%
		\expandafter\pgfplots@loc@TMPa\csname pgfp@numtable@glob@col@\thepgfplotstableread@curcol\endcsname
		\pgfplotstableread@countadvance\pgfplotstableread@curcol
	\pgfutil@repeat
}

\def\pgfplotstableread@create@column@names@with@numbers{%
	\pgfplotstableread@countreset\pgfplotstableread@curcol
	\pgfutil@loop
	\ifnum\pgfplotstableread@curcol<\pgfplotstableread@numcols\relax
		\edef\pgfplotstable@loc@TMPb{\thepgfplotstableread@curcol}%
		\expandafter\pgfplotslistpushbackglobal\pgfplotstable@loc@TMPb\to\pgfplotstable@colnames@glob
		\pgfplotstableread@countadvance\pgfplotstableread@curcol
	\pgfutil@repeat
}

\long\def\pgfplotstableread@impl@gobble#1\pgfplotstable@EOI{}%

\def\pgfplotstable@EOI{\pgfplotstable@EOI}%

% A loop command which processes every single entry in a raw data row #2 
% and invokes the macro #1{<arg>}  for each found column entry.
%
% Columns are separated by the /pgfplots/table/col sep character.
%
% #1: a command which takes precisely one argument. It will be called
% for each found column entry
%
% #2: a macro containing a raw data line with <col sep> separated
% entries.
\long\def\pgfplotstableread@impl@DO#1#2{%
	\ifpgfplotstable@trimcells
		\def\pgfplotstableread@impl@ITERATE@NEXT@##1{%
			\pgfkeys@spdef\pgfplotstableread@impl@ITERATE@NEXT@@@{##1}%
			\expandafter#1\expandafter{\pgfplotstableread@impl@ITERATE@NEXT@@@}%
		}%
	\else
		\let\pgfplotstableread@impl@ITERATE@NEXT@=#1\relax
	\fi
	\expandafter\pgfplotstableread@impl@DO@\expandafter{#2}%
}
{%
	\catcode`\ =10
	\catcode`\;=12
	\catcode`\:=12
	\gdef\pgfplotstableread@impl@prepare@DO{%
		\ifcase\pgfplotstableread@COLSEP@CASE\relax
			% col sep=space:
			\catcode`\ =10
			\long\def\pgfplotstableread@impl@DO@##1{\pgfplotstableread@impl@ITERATE##1 \pgfplotstable@EOI}%
		\or
			% col sep=comma:
			\let\pgfplotstableread@impl@ITERATE@NEXT=\pgfplotstableread@impl@ITERATE@NEXT@COMMA
			\long\def\pgfplotstableread@impl@DO@##1{\pgfplotstableread@impl@ITERATE##1,\pgfplotstable@EOI}%
		\or
			% col sep=semicolon:
			\catcode`\;=12
			\let\pgfplotstableread@impl@ITERATE@NEXT=\pgfplotstableread@impl@ITERATE@NEXT@SEMICOLON
			\long\def\pgfplotstableread@impl@DO@##1{\pgfplotstableread@impl@ITERATE##1;\pgfplotstable@EOI}%
		\or
			% col sep=colon:
			\catcode`\:=12
			\let\pgfplotstableread@impl@ITERATE@NEXT=\pgfplotstableread@impl@ITERATE@NEXT@COLON
			\long\def\pgfplotstableread@impl@DO@##1{\pgfplotstableread@impl@ITERATE##1:\pgfplotstable@EOI}%
		\or
			% col sep=brace:
			% allow multi line cells:
			\endlinechar=\pgfplotstableread@oldendlinechar\relax
			\let\pgfplotstableread@impl@ITERATE@NEXT=\pgfplotstableread@impl@ITERATE@NEXT@BRACE
			\long\def\pgfplotstableread@impl@DO@##1{\pgfplotstableread@impl@ITERATE##1\pgfplotstable@EOI}%
		\or
			% col sep=tab:
			\catcode`\^^I=12
			\let\pgfplotstableread@impl@ITERATE@NEXT=\pgfplotstableread@impl@ITERATE@NEXT@TAB
			\long\edef\pgfplotstableread@impl@DO@##1{\noexpand\pgfplotstableread@impl@ITERATE##1\pgfplotstableread@tab\noexpand\pgfplotstable@EOI}%
		\or
			% col sep=&:
			\let\pgfplotstableread@impl@ITERATE@NEXT=\pgfplotstableread@impl@ITERATE@NEXT@AMPERSAND
			\long\def\pgfplotstableread@impl@DO@##1{\pgfplotstableread@impl@ITERATE##1&\pgfplotstable@EOI}%
		\fi
	}%
}%
\long\def\pgfplotstableread@impl@ITERATE{%
	\pgfutil@ifnextchar\pgfplotstable@EOI{%
		\pgfplotstableread@impl@gobble
	}{%
		\pgfplotstableread@impl@ITERATE@NEXT
	}%
}%
\long\def\pgfplotstableread@impl@ITERATE@NEXT#1 {%
	\pgfplotstableread@impl@ITERATE@NEXT@{#1}%
	\pgfplotstableread@impl@ITERATE
}%
\long\def\pgfplotstableread@impl@ITERATE@NEXT@COMMA#1,{%
	\pgfplotstableread@impl@ITERATE@NEXT@{#1}%
	\pgfplotstableread@impl@ITERATE
}%
\long\def\pgfplotstableread@impl@ITERATE@NEXT@SEMICOLON#1;{%
	\pgfplotstableread@impl@ITERATE@NEXT@{#1}%
	\pgfplotstableread@impl@ITERATE
}%
\long\def\pgfplotstableread@impl@ITERATE@NEXT@COLON#1:{%
	\pgfplotstableread@impl@ITERATE@NEXT@{#1}%
	\pgfplotstableread@impl@ITERATE
}%
\long\def\pgfplotstableread@impl@ITERATE@NEXT@BRACE#1{%
	\pgfplotstableread@impl@ITERATE@NEXT@{#1}%
	\pgfplotstableread@impl@ITERATE
}%
\long\def\pgfplotstableread@impl@ITERATE@NEXT@AMPERSAND#1&{%
	\pgfplotstableread@impl@ITERATE@NEXT@{#1}%
	\pgfplotstableread@impl@ITERATE
}%
\begingroup
\catcode`\^^I=12
\gdef\pgfplotstableread@tab{^^I}%
\long\gdef\pgfplotstableread@impl@ITERATE@NEXT@TAB#1^^I{% the following white spaces are SPACES, not tabs:
    \pgfplotstableread@impl@ITERATE@NEXT@{#1}%
    \pgfplotstableread@impl@ITERATE
}%
\endgroup
%%%%%%%%%%%%%%%%%%%%%%%%%%%%%%%%%%%%%%%%%%%%%%%%%%%%%%%%%%%%%

% these values are only usable for a read-listener, that means:
% when using
%
% \pgfplotstableread{<file>} to listener<\macro>
%
% -> <\macro> can than use the methods
%
% \pgfplotstablereadgetcolindex{<index>}{<\content>}
% performs \let\content=<content of column no <index> >
\def\pgfplotstablereadgetcolindex@#1#2{%
	\pgfutil@ifundefined{pgfplotstblread@colcontent@no#1}{%
		\pgfplotsthrow{no such element}{#2}{Sorry, the requested column number '#1' in table '\pgfplotstableread@filename' does not exist!? Please verify you used the correct index 0 <= i < N.}\pgfeov%
	}{%
		\expandafter\let\expandafter#2\csname pgfplotstblread@colcontent@no#1\endcsname
	}%
}%

\pgfkeysdef{/pgfplots/table/@undefined column text}{#1__column_not_found.}%
\def\pgfplotstable@undefinedtext#1{\pgfkeysvalueof{/pgfplots/table/@undefined column text/.@cmd}#1\pgfeov}%

% As \pgfplotstablereadgetcolindex, but
% \pgfplotstablereadvalueofcolindex{<index>}
% directly expands to the value stored in the desired column.
%
% Example:
% \pgfplotstablereadvalueofcolindex{3}  -> expands to '42' if '42' is
% written in column no 3.
%
% Column indexing starts at 0.
%
% @ATTENTION: since such a command may occur within an \edef or an
% \csname, it can't perform sanity checking. Proving an invalid index
% expands to \pgfkeysvalueof{/pgfplots/table/@undefined column text}.
\def\pgfplotstablereadvalueofcolindex@#1{%
	\pgfutil@ifundefined{pgfplotstblread@colcontent@no#1}{%
		\pgfplotstable@undefinedtext{colindex#1}%
	}{%
		\csname pgfplotstblread@colcontent@no#1\endcsname
	}%
}

% \pgfplotstablereadgetcolname{<name>}{<\content>}
% performs \let\content=<content of column named <name> >
\def\pgfplotstablereadgetcolname@#1#2{%
	\pgfplotstablereadgetptrtocolname{#1}{\pgfplots@loc@TMPa}%
	\pgfplotstablereadevalptr\pgfplots@loc@TMPa{#2}%
}%

% This here is the implementation of \pgfplotstablereadvalueofcolname
% (and \thisrow{<colname>}) for use inside of the 'read to listener'
% framework.
%
% Like \pgfplotstablereadgetcolname, but this one expands directly to
% the value of the desired column.
%
% #1: a column name or a column alias.
%
% @ATTENTION: since such a command may occur within an \edef or an
% \csname, it can't perform sanity checking. Proving an invalid index
% expands to \pgfkeysvalueof{/pgfplots/table/@undefined column text}.
\def\pgfplotstablereadvalueofcolname@#1{%
	\pgfplotstable@thisrow@impl{#1}{pgfplotstblread@colindex@for@name}{\pgfplotstable@thisrow@impl@read}%
}%
\def\pgfplotstable@thisrow@impl@read#1{\csname pgfplotstblread@colcontent@no#1\endcsname}%

% This implements \thisrow in different contexts.
%
% Usage:
% \def\thisrow#1{\pgfplotstable@thisrow@impl{#1}{macroprefix@}{\pgfplotstable@thisrow@impl@}}
%
% Then, \thisrow{existingcol}
% will expand to 
% -> \pgfplotstable@thisrow@impl@{\csname macroprefix@existingcol\endcsname}
% -> \csname macroprefix@existingcol\endcsname
%
% Furthermore, if '#1' is no existing col and there exists /pgfplots/table/alias/#1,
% \thisrow{aliased}
% will expand to
% -> \pgfplotstable@thisrow@impl@{\csname macroprefix@\pgfkeysvalueof{/pgfplots/table/alias/aliased}\endcsname}
% -> \csname macroprefix@\pgfkeysvalueof{/pgfplots/table/alias/aliased}\endcsname
%
% #1: the argument for \thisrow{#1}
% #2: a macro prefix such that \csname #2<colname>\endcsname contains
% the value of the current row for (physical) column <colname>
% #3: the name of a one-argument-macro which will get \csname #2<colname>\endcsname
% as argument. This is the last step of \thisrow. It allows indirect
% access by translating colnames to col indices in '\pgfplotstableread to listener'
\def\pgfplotstable@thisrow@impl#1#2#3{%
	\pgfutil@ifundefined{#2#1}%
	{%
		% WARNING : this code has been REPLICATED in
		% *** \pgfplotstablereadgetptrtocolname ***
		% *** \pgfplotstablegetcolumnbyname ***
		% *** \pgfplotstableresolvecolname ***
		% *** \pgfplotstablereadvalueofcolname ***
		\pgfkeysifdefined{/pgfplots/table/alias/#1}{%
			\pgfutil@ifundefined{#2\pgfkeysvalueof{/pgfplots/table/alias/#1}}{%
				\pgfplotstable@undefinedtext{\pgfkeysvalueof{/pgfplots/table/alias/#1}}%
			}{%
				#3{\csname #2\pgfkeysvalueof{/pgfplots/table/alias/#1}\endcsname}%
			}%
		}{%
			\pgfplotstable@undefinedtext{#1}%
		}%
	}%
	{%
		#3{\csname #2#1\endcsname}%
	}%
}%
\def\pgfplotstable@thisrow@impl@#1{#1}%

% \pgfplotstablereadgetptrtocolname{<name>}{\ptr}
% Creates some sort of "pointer" to the column named <name>. This
% pointer can than be used every time a new line has been reported to
% the listener. It works like this:
%
% \let\ptr=\pgfutil@empty
% \def\macro{%
% 	\ifx\ptr\empty
% 		\pgfplotstablereadgetptrtocolname{<my col>}{\ptr}%
% 	\fi
% 	\pgfplotstablereadevalptr{\ptr}{\content}%
% 	-> do something with \content!
% }
% 	
% \pgfplotstableread{<file>} to listener<\macro>
%
% -> will evaluate \macro foreach row.
%
% The advantage of such a prepared pointer over \thisrow{#1} or
% \getthisrow{#1} is simply efficiency and sanity checking: the checks
% are done at the time of pointer creation, dereferencing the pointer
% is fast.
\def\pgfplotstablereadgetptrtocolname@#1#2{%
	\def#2{0}%
	\pgfplotstable@getcol{#1}\of{\pgfplotstableread@filename}\to{#2}\getcolmethod{getptr}%
}%

% As \pgfplotstablereadgetptrtocolname, but this here access columns
% by index.
\def\pgfplotstablereadgetptrtocolindex@#1#2{\def#2{#1}}%

% \pgfplotstablereadevalptr{<\ptr>}{<\content}
% writes the current value of <\ptr> to <\content>. The pointer <\ptr>
% must be initialised with \pgfplotstablereadgetptrtocolname
\let\pgfplotstablereadevalptr@=\pgfplotstablereadgetcolindex@

% \pgfplotstablereadvalueofptr{<\ptr>} -> expands to the pointers value.
%
% The pointer \ptr must be initialised with
% \pgfplotstablereadgetptrtocolname.
\let\pgfplotstablereadvalueofptr@=\pgfplotstablereadvalueofcolindex@
%%%%%%%%%%%%%%%%%%%%%%%%%%%%%%%%%%%%%%%%%%%%%%%%%%%%%%%%%%%%%

\long\def\pgfplotstableread@impl@nextrow@NEXT@listener#1{%
	\expandafter\def\csname pgfplotstblread@colcontent@no\thepgfplotstableread@curcol\endcsname{#1}%
	\pgfplotstableread@countadvance\pgfplotstableread@curcol
}

\long\def\pgfplotstableread@impl@nextrow@NEXT#1{%
%\pgfplots@message{Inserting '#1' at (\thepgfplotstableread@lineno, \thepgfplotstableread@curcol).}%
	\ifnum\pgfplotstableread@curcol<\pgfplotstableread@numcols\relax
		\pgfplotslist@assembleentry{#1}\into\t@pgfplots@tokc
		\def\pgfplotstableread@TMP{\expandafter\pgfplotsapplistXpushback\expandafter{\the\t@pgfplots@tokc}\to}%
		\expandafter\pgfplotstableread@TMP\csname pgfp@numtable@glob@col@\thepgfplotstableread@curcol\endcsname
		\pgfplotstableread@countadvance\pgfplotstableread@curcol
	\else
		\begingroup
		\t@pgfplots@tokc={#1}%
		\pgfplotstable@error{Table '\pgfplotstableread@filename' appears to have too many columns in line \thepgfplotstableread@lineno: Ignoring '\the\t@pgfplots@tokc'. PGFPlots found that the number of columns is larger than the previously determined number of columns. Please verify that every cell entry is separated correctly (use braces {<cell entry>} if necessary. Also verify that column names are plain ASCII.). This error is not critical}%
		\endgroup
	\fi
}



\long\def\pgfplotstableread@impl@collectcolnames@NEXT#1{%
%\pgfplots@message{Got column name no \thepgfplotstableread@curcol\ as '#1'}%
	\pgfutil@ifundefined{pgfplotstableread@impl@COLNAME@#1}{%
		\def\pgfplotstable@loc@TMPa{#1}%
	}{% generate unique column names warning:
		\pgfplots@warning{Table '\pgfplotstableread@filename' has non-unique column name '#1'. Only the first occurence can be accessed via column names.}%
		\edef\pgfplotstable@loc@TMPa{#1--index\thepgfplotstableread@curcol}%
	}%
	\expandafter\def\csname pgfplotstableread@impl@COLNAME@#1\endcsname{foo}% remember this name.
	\expandafter\pgfplotslistpushbackglobal\expandafter{\pgfplotstable@loc@TMPa}\to\pgfplotstable@colnames@glob
	\ifpgfplots@tableread@to@listener
		% create an associative container colindex -> colname
		% for use in a listener.
		\expandafter\edef\csname pgfplotstblread@colindex@for@name#1\endcsname{\thepgfplotstableread@curcol}%
	\fi
	\pgfplotstableread@countadvance\pgfplotstableread@curcol
}




\long\def\pgfplotstableread@impl@countcols@and@identifynames@NEXT#1{%
	\pgfplotstableread@countadvance\pgfplotstableread@curcol
	\ifpgfplotstable@search@header
		\ifpgfplotstableread@curline@contains@colnames
		\else
			\pgfplotstableread@isnumber@ITERATE#1\pgfplotstable@EOI
%\ifpgfplotstableread@curline@contains@colnames\pgfplots@message{'#1' is a column name!}\else\pgfplots@message{'#1' is NO column name!}\fi
		\fi
	\fi
}
\def\pgfplotstableread@isnumber@plus{+}
\def\pgfplotstableread@isnumber@minus{-}
\def\pgfplotstableread@isnumber@zero{0}
\def\pgfplotstableread@isnumber@one{1}
\def\pgfplotstableread@isnumber@two{2}
\def\pgfplotstableread@isnumber@three{3}
\def\pgfplotstableread@isnumber@four{4}
\def\pgfplotstableread@isnumber@five{5}
\def\pgfplotstableread@isnumber@six{6}
\def\pgfplotstableread@isnumber@seven{7}
\def\pgfplotstableread@isnumber@eight{8}
\def\pgfplotstableread@isnumber@nine{9}
\def\pgfplotstableread@isnumber@e{e}
\def\pgfplotstableread@isnumber@E{E}
\def\pgfplotstableread@isnumber@period{.}

\def\pgfplotstableread@isnumber@ITERATE#1{%
	\def\pgfplotstableread@CURTOK{#1}%
	\ifx\pgfplotstableread@CURTOK\pgfplotstable@EOI
		\def\pgfplotstableread@NEXT{}%
	\else
		\def\pgfplotstableread@NEXT{\pgfplotstableread@isnumber@ITERATE}%
		\ifx\pgfplotstableread@CURTOK\pgfplotstableread@isnumber@plus
		\else
		\ifx\pgfplotstableread@CURTOK\pgfplotstableread@isnumber@minus
		\else
		\ifx\pgfplotstableread@CURTOK\pgfplotstableread@isnumber@zero
		\else
		\ifx\pgfplotstableread@CURTOK\pgfplotstableread@isnumber@one
		\else
		\ifx\pgfplotstableread@CURTOK\pgfplotstableread@isnumber@two
		\else
		\ifx\pgfplotstableread@CURTOK\pgfplotstableread@isnumber@three
		\else
		\ifx\pgfplotstableread@CURTOK\pgfplotstableread@isnumber@four
		\else
		\ifx\pgfplotstableread@CURTOK\pgfplotstableread@isnumber@five
		\else
		\ifx\pgfplotstableread@CURTOK\pgfplotstableread@isnumber@six
		\else
		\ifx\pgfplotstableread@CURTOK\pgfplotstableread@isnumber@seven
		\else
		\ifx\pgfplotstableread@CURTOK\pgfplotstableread@isnumber@eight
		\else
		\ifx\pgfplotstableread@CURTOK\pgfplotstableread@isnumber@nine
		\else
		\ifx\pgfplotstableread@CURTOK\pgfplotstableread@isnumber@e
		\else
		\ifx\pgfplotstableread@CURTOK\pgfplotstableread@isnumber@E
		\else
		\ifx\pgfplotstableread@CURTOK\pgfplotstableread@isnumber@period
		\else
%\message{NO ITS NOT!  Token: '\meaning\pgfplotstableread@CURTOK'}%
			% it's no number, so it is a column name.
			\pgfplotstableread@curline@contains@colnamestrue
			\def\pgfplotstableread@NEXT{\pgfplotstableread@impl@gobble}%
		\fi\fi\fi\fi\fi\fi\fi\fi\fi\fi\fi\fi\fi\fi\fi
	\fi
	\pgfplotstableread@NEXT
}

\def\pgfplotstable@error#1{\pgfplotsthrow{unsupported operation}{#1}\pgfeov}%


\def\pgfplotstableset{%
	\pgfqkeys{/pgfplots/table}%
}%

% Accepts a macro #1 which contains an argument denoting a column
% name.
%
% It checks whether #1 starts with '[index]', indicating that it is actually
% a column INDEX. If that is the case,
% \ifpgfplotstableread@foundcolnames is set to false and the index is
% returned into #1.
%
% Otherwise, \ifpgfplotstableread@foundcolnames is set to true.
\def\pgfplotstable@is@colname#1{%
	\expandafter\pgfplotstabletypeset@is@colname@#1\pgfplotstable@EOI
	\ifpgfplotstableread@foundcolnames
	\else
		\let#1=\pgfplotstable@loc@TMPa
	\fi
}%
\def\pgfplotstabletypeset@is@colname@{%
	\pgfutil@ifnextchar[{%
		\pgfplotstabletypeset@is@colname@index
	}{%
		\pgfplotstableread@foundcolnamestrue
		\pgfplotstabletypeset@is@colname@name
	}%
}
\def\pgfplotstabletypeset@is@colname@index@@{index}%
\def\pgfplotstabletypeset@is@colname@index[#1]#2\pgfplotstable@EOI{%
	\def\pgfplotstable@loc@TMPa{#1}%
	\ifx\pgfplotstable@loc@TMPa\pgfplotstabletypeset@is@colname@index@@
		\pgfplotstableread@foundcolnamesfalse
		\edef\pgfplotstable@loc@TMPa{#2}%
	\else
		\pgfplotstableread@foundcolnamestrue
	\fi
}%
\def\pgfplotstabletypeset@is@colname@name#1\pgfplotstable@EOI{}%

% calls \pgfplotstableset{every table={#1}} where '#1' is the table
% file name.
%
% Note that 'every table' is usually set *AFTER* any other option. In
% other words: settings in 'every table' have higher precedence than
% those provided, for example, at
% \pgfplotstabletypeset[<options>]{...}.
%
% Why? Well, the <options> need to be set *before* the file name
% argument is read. After all, it might contain an inline table for
% which \catcodes need to be adjusted -- and <options> might explain
% how.
%
% REMARK: An early attempt to change the precendence for 'every table'
% was to set <options> again after 'every table'. But that is
% incompatible with, for example '/.add={}{}' key handlers.
%
% This macro does nothing if 'every table' is empty.
%
% #1: the table name (argument to 'every table')
% #2: The <options> which have already been set.
\long\def\pgfplotstableset@every@table#1#2{%
	\pgfkeysgetvalue{/pgfplots/table/every table/.@cmd}\pgfplotstable@loc@TMPa
	\ifx\pgfplotstable@loc@TMPa\pgfplots@empty@style@key
	\else
		\pgfplotstableset{/pgfplots/table/every table={#1}}%,#2}%
	\fi
}%

\def\pgfplotstabletypeset{%
	\pgfutil@ifnextchar[{%	
		\pgfplotstabletypeset@opt
	}{%
		\pgfplotstabletypeset@opt[]%
	}%
}
\long\def\pgfplotstabletypeset@opt[#1]#2{%
	\pgfplotstable@error@pkg{Please load \string\usepackage{pgfplotstable} before using \string\pgfplotstabletypeset}%
}
\let\pgfplotstabletypesetfile=\pgfplotstabletypeset

\def\pgfplotstablecreatecol{%
	\pgfutil@ifnextchar[{%
		\pgfplotstablecreatecol@opt
	}{%
		\pgfplotstablecreatecol@opt[]%
	}%
}%
\def\pgfplotstablecreatecol@opt[#1]#2#3{%
	\pgfplotstable@error@pkg{Please load \string\usepackage{pgfplotstable} before using \string\pgfplotstablecreatecol}%
}%

% Iterates through every column of table '#1' and invokes the code
% '#3' for each column. The current column name will be available as
% '#2' and the current column index as |\pgfplotstablecol| (starting
% with 0).
%
% Example:
% \pgfplotstableforeachcolumn{\table}\as\colname{%
% 	The column name is `\colname'; its index \pgfplotstablecol.\par
% }%
%
% REMARK: this routine does NOT introduce TeX groups.
\long\def\pgfplotstableforeachcolumn#1\as#2#3{%
	\def\pgfplotstablecol{0}%
	\pgfplotslistforeachungrouped#1\as#2{%
		#3\relax%
		\pgfplotsutil@advancestringcounter\pgfplotstablecol
	}%
	\let\pgfplotstablecol=\relax
}%
\let\pgfplotstableforeachcolumn@orig=\pgfplotstableforeachcolumn

% A variant of \pgfplotstableforeachcolumn which is used inside of
% \pgfplotstableread to listener.
%
% It is used as \pgfplotstableforeachcolumn\as\cur{<do something with \cur>}
\long\def\pgfplotstableforeachcolumn@listener#1\as#2#3{%
	\pgfplotstableforeachcolumn@orig\pgfplotstable@colnames@glob\as{#2}{#3}%
}%

% Reports every element t_{ij} for a fixed column j (in read-only
% mode).
%
% For every cell, the code '#4' will be executed where '#3' will
% contain the cell's value. During code '#4', the macro
% \pgfplotstablerow will contain the current row index.
%
% #1: either a column name or the string '[index]' followed by a
% number denoting a column index Access by column name is much faster..
% #2: the table (macro or file name)
% #3: the macro in which the cell values shall be written
% #4: the code to execute.
%
% Example:
% \pgfplotstableforeachcolumnelement{colname}\of\table\as\cellelem{%
% 	I have now cell element `\cellelem' at row index
% 	`\pgfplotstablerow'.
% 	\par
% }
%
% REMARK: this routine does NOT introduce TeX groups.
\long\def\pgfplotstableforeachcolumnelement#1\of#2\as#3#4{%
	\def\pgfplotstablerow{0}%
	\pgfplotstablegetcolumn{#1}\of{#2}\to\pgfplotstableforeachcolumnelement@list
	\pgfplotslistforeachungrouped\pgfplotstableforeachcolumnelement@list\as#3{%
		% allow nesting by copying the old value of \pgfplotstablerow:
		\expandafter\pgfplotstableforeachcolumnelement@\expandafter{\pgfplotstablerow}{#4}%
	}%
	\let\pgfplotstablerow=\relax
}%
% helper method to allow nesting. It copies \pgfplotstablerow.
% #1: the expanded value of \pgfplotstablerow. 
\long\def\pgfplotstableforeachcolumnelement@#1#2{%
	#2\relax
	\def\pgfplotstablerow{#1}% restore to old value.
	% advance.
	\pgfplotsutil@advancestringcounter\pgfplotstablerow
}%

% A routine which is similar to \pgfplotstableforeachcolumnelement,
% but this here checks for changes in \pgfplotsretval and writes them
% back into the respected cell.
%
% The runtime is quadratic in the number of rows.
\long\def\pgfplotstablemodifyeachcolumnelement#1\of#2\as#3#4{%
	\def\pgfplotstablerow{0}%
	%
	% Step 0: get the REAL column name for '#1'.
	% This needs modifications if '#1' is [index]<integer>.
	% -> store the colname to \pgfplotstable@loc@TMPc:
	\def\pgfplotstable@loc@TMPc{#1}%
	\pgfplotstable@is@colname{\pgfplotstable@loc@TMPc}%%
	\ifpgfplotstableread@foundcolnames
	\else
		\expandafter\pgfplotstablegetcolumnnamebyindex\pgfplotstable@loc@TMPc\of{#2}\to\pgfplotstable@loc@TMPc
	\fi
	% Step 0.1: prepare a command which re-assembles column '#1'
	% (using the real column name).
	% The re-assemblation command will be invoked at the end of each
	% iteration. This complicated macro preparation allows nested
	% calls to \pgfplotstablemodifyeachcolumnelement.
	\t@pgfplots@toka={\expandafter\pgfplotslistpushback\expandafter{#3}\to}%
	\edef\pgfplotstable@loc@TMPd{%
		\the\t@pgfplots@toka{\expandafter\noexpand\csname\string#2@\pgfplotstable@loc@TMPc\endcsname}}%
	% Step 1: copy the column data to \pgfplotstable@loc@TMPb
	\expandafter\pgfplotstablegetcolumnbyname\pgfplotstable@loc@TMPc\of#2\to\pgfplotstable@loc@TMPb%
	%
	% clear the original column:
	\expandafter\pgfplotslistnewempty\csname\string#2@\pgfplotstable@loc@TMPc\endcsname
	%
	% Call loop. The prepared re-assemble macro will be provided as macro argument
	% to allow nested calls:
	\expandafter\pgfplotstablemodifyeachcolumnelement@\expandafter{\pgfplotstable@loc@TMPd}
		{\pgfplotstable@loc@TMPb}{#3}{#4}%
	\let\pgfplotstablerow=\relax
}%
% #1: post-iteration code
% #2: the row list
% #3: the loop macro to assign
% #4: the loop body
\long\def\pgfplotstablemodifyeachcolumnelement@#1#2#3#4{%
	\pgfplotslistforeachungrouped{#2}\as#3{%
		% allow nesting by copying the old value of \pgfplotstablerow:
		\expandafter\pgfplotstableforeachcolumnelement@\expandafter{\pgfplotstablerow}{#4}%
		#1%
	}%
}

% Selects a single table element at row #1 and column #2 from table
% #3.
%
% #1: a row index.
% #2: a column name or the string '[index]' followed by a number
% denoting a column index. Access by column name is much faster.
% #3: the table (macro or file name).
%
% The cell value will be written into the macro \pgfplotsretval.
%
% REMARK:
% this routine is supposed to be very slow: it needs time O(N) where N
% is the number of rows. This may change in future versions.
%
% Example:
% \pgfplotstablegetelem{0}{[index]2}\of\table
% The elem is `\pgfplotsretval'.
\def\pgfplotstablegetelem#1#2\of#3{%
	\begingroup
	\pgfplotstablegetcolumn{#2}\of{#3}\to\pgfplotstable@loc@TMPa
	\def\pgfplotsexceptionmsg{Sorry, row `#1' does not exist in table \pgfplotstablenameof{#3}}%
	\pgfplotslistselect#1\of\pgfplotstable@loc@TMPa\to\pgfplotsretval
	\pgfmath@smuggleone\pgfplotsretval
	\endgroup
}%

\def\pgfplotstablegetcolumnlist#1\to#2{\let#2=#1}

% Defines \pgfmathresult (and now also \pgfplotsretval) to be the number of rows in table #1.
%
% #1 may be either a loaded table structure (a macro name) or a table
% file name. In the latter case, the file will be loaded temporarily.
\long\def\pgfplotstablegetrowsof#1{%
	\pgfplotstable@isloadedtable{#1}{%
		% ah - it is an already loaded table!
		\begingroup
		\pgfplotslistfront#1\to\pgfplotstablegetrows@@
		\expandafter\pgfplotslistsize\csname\string#1@\pgfplotstablegetrows@@\endcsname\to\c@pgfplotstable@counta
		\edef\pgfmathresult{\the\c@pgfplotstable@counta}%
		\pgfmath@smuggleone\pgfmathresult
		\endgroup
	}{%
		% ah - it is a file name.
		\begingroup
			\pgfplotstableread{#1}\pgfplotstablegetrows@
			\pgfplotstablegetrowsof{\pgfplotstablegetrows@}%
			\pgfmath@smuggleone\pgfmathresult
		\endgroup
	}%
	\let\pgfplotsretval=\pgfmathresult
}%

% Defines \pgfplotsretval to contain the number of columns in table
% #1.
%
% #1 may be either a loaded table structure (a macro name) or a table
% file name. In the latter case, the file will be loaded temporarily.
\long\def\pgfplotstablegetcolsof#1{%
	\pgfplotstable@isloadedtable{#1}{%
		% ah - it is an already loaded table!
		\begingroup
		\pgfplotslistsize#1\to\c@pgfplotstable@counta
		\edef\pgfplotsretval{\the\c@pgfplotstable@counta}%
		\pgfmath@smuggleone\pgfplotsretval
		\endgroup
	}{%
		% ah - it is a file name.
		\begingroup
			\pgfplotstableread{#1}\pgfplots@table
			\pgfplotstablegetcolsof{\pgfplots@table}%
			\pgfmath@smuggleone\pgfplotsretval
		\endgroup
	}%
	\let\pgfmathresult=\pgfplotsretval
}
