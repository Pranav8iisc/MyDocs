\documentclass{article}

\usepackage{amsmath}

\begin{document}

\title{Constrained delaunay tetrahedralization}
\author{Pranav Kant Gaur}

\maketitle

\begin{abstract}
This document describes work of the author to develop and integrate module for \textit{\textbf{Constrained Delaunay tetrahedralization}} into an \textit{inhouse} developed data visualization framework \textbf{AnuVi}.
\end{abstract}

\section{Problem definition}
The problem can be decomposed into:
\begin{enumerate}
\item Understanding Constrained Delauney tetrahedralization problem
\item Studying other solutions
\item Selecting and implementing the solution most suited to the \textbf{\textit{real}} problem at hand
\end{enumerate}

\section{Analysing subparts of the problem}
In this section, individual components of the problem will be analysed in detail.
\subsection{Constrained Delaunay tetrahedralization}
Input: 
\begin{enumerate}
\item Set of points, $P\in R^{3}$ over \textit{euclidean} space
\item Set of segments, $S=\lbrace{(p_{1},p_{2})| p_{1},p_{2}\in P\rbrace}$ called \textit{constraint segments} 
\end{enumerate}
Problem:\newline
Compute connectivity over points in $P$ called \textit{volume mesh} say $M$, such that any tetrahedron $T$ in $M$ satisfies \textit{empty sphere} criterion and $M$ must \textit{preserve} constraint segments.

\subsubsection{Empty sphere criterion}
It simply states that for any tetrahedron $T$ in CDT, there must not be any vertex lying \textit{inside} the circumsphere of T. If there are more than 4 vertices sharing a circumsphere then CDT for those set of points and constraint segments will not be \textit{unique}.


\subsection{Studying proposed solution(s)}
In this work, author will implement and analyze CDT method described in paper `\textit{Meshing Piecewise Linear Complexes by Constrained Delaunay Tetrahedralizations}' by Hang Si and Klaus G{\"a}rtner.


\subsubsection{Algorithm}
Given \textit{piecewise linear complex(PLC)}, X_0, which encapslates the vertices & constraining segments.
\end{document}
